% mnras_template.tex 
%
% LaTeX template for creating an MNRAS paper
%
% v3.0 released 14 May 2015
% (version numbers match those of mnras.cls)
%
% Copyright (C) Royal Astronomical Society 2015
% Authors:
% Keith T. Smith (Royal Astronomical Society)

% Change log
%
% v3.0 May 2015
%    Renamed to match the new package name
%    Version number matches mnras.cls
%    A few minor tweaks to wording
% v1.0 September 2013
%    Beta testing only - never publicly released
%    First version: a simple (ish) template for creating an MNRAS paper

%%%%%%%%%%%%%%%%%%%%%%%%%%%%%%%%%%%%%%%%%%%%%%%%%%
% Basic setup. Most papers should leave these options alone.
\documentclass[fleqn,usenatbib]{mnras}

% MNRAS is set in Times font. If you don't have this installed (most LaTeX
% installations will be fine) or prefer the old Computer Modern fonts, comment
% out the following line
%\usepackage{newtxtext,newtxmath}
% Depending on your LaTeX fonts installation, you might get better results with one of these:
\usepackage{mathptmx}
\usepackage{txfonts}
\usepackage[T1]{fontenc}

% Allow "Thomas van Noord" and "Simon de Laguarde" and alike to be sorted by"N" and "L" etc. in the bibliography.
% Write the name in the bibliography as "\VAN{Noord}{Van}{van} Noord, Thomas"
\DeclareRobustCommand{\VAN}[3]{#2}
\let\VANthebibliography\thebibliography
\def\thebibliography{\DeclareRobustCommand{\VAN}[3]{##3}\VANthebibliography}

\usepackage{graphicx}	% Including figure files
\usepackage{amsmath}	% Advanced maths commands
\usepackage{amssymb}	% Extra maths symbols
\usepackage{longtable}


%%%%%%%%%%%%%%%%%%%%%%%%%%%%%%%%%%%%%%%%%%%%%%%%%%

%%%%% AUTHORS - PLACE YOUR OWN COMMANDS HERE %%%%%

% Please keep new commands to a minimum, and use \newcommand not \def to avoid
% overwriting existing commands. Example:
%\newcommand{\pcm}{\,cm$^{-2}$}	% per cm-squared

%%%%%%%%%%%%%%%%%%%%%%%%%%%%%%%%%%%%%%%%%%%%%%%%%%

%%%%%%%%%%%%%%%%%%% TITLE PAGE %%%%%%%%%%%%%%%%%%%

% Title of the paper, and the short title which is used in the headers.
% Keep the title short and informative.
\title[S-PLUS: Emission line objects]{Photometric identification of emission
  line sources in the southern photometric local Universe survey (S-PLUS)}

% The list of authors, and the short list which is used in the headers.
% If you need two or more lines of authors, add an extra line using \newauthor
\author[Guti\'{e}rrez-Soto et al.]{
L. A. Guti\'{e}rrez-Soto,$^{1}$\thanks{E-mail: gsoto.angel@gmail.com}
Second  Author,$^{2}$
Third Author$^{2,3}$
and Fourth Author$^{3}$
\\
% List of institutions
$^{1}$Departamento de Astronomia, IAG, Universidade de S\~{a}o Paulo, Rua do Mat\~{a}o,
1226, 05509-900, S\~{a}o Paulo, Brazil\\
$^{2}$Department, Institution, Street Address, City Postal Code, Country\\
$^{3}$Another Department, Different Institution, Street Address, City Postal Code, Country
}

% These dates will be filled out by the publisher
\date{Accepted XXX. Received YYY; in original form ZZZ}

% Enter the current year, for the copyright statements etc.
\pubyear{2021}

% Don't change these lines
\begin{document}
\label{firstpage}
\pagerange{\pageref{firstpage}--\pageref{lastpage}}
\maketitle

% Abstract of the paper
\begin{abstract}
  The emission line objects are very important objects in astronomy
  because reflects different class of objects that evolved physical
  mechanics that given counts of stellar formation process, presences
  the gas, shocks, star-burst in galaxies, the finals stage of stars
  among others process. For this reason we have created a list of
  H{$\alpha$} emitters selected from the S-PLUS data, which is mapping
  the southern hemisphere at relatively high latitudes. We implemented
  the (r - $J$0660) versus (r - i) color-color diagram for that task.
  We found 9,200 objects that exhibit um excess in emission in the J066o
  which we have traduced as the presence of the H{$\alpha$} emission line.
  In addition we have found that by combining the colors: (r - i) and (g - z)
  with unsupervised (clustering) machine learning it is possible separate
  the blue sources from red ones, then we ave divided our list of emitters
  in two sub-groups: one with intense blue continuum and another with
  intense red one. We compare hierarchical-clustering algorithm with the
  HDBSCAN. By adopting a ``soft'' clustering approach, we can assign
  each emission object a probability of belonging to a given cluster
  (blue or red group), allowing for more flexibility in the classification
  of objects according to these colours those objects with a low probability
  of belonging to any cluster. 
\end{abstract}
% Select between one and six entries from the list of approved keywords.
% Don't make up new ones.
\begin{keywords}
  surveys -- stars: emission-line, Be -- novae, cataclysmic variables
  -- galaxies: dwarf -- quasars: emission lines
\end{keywords}

%%%%%%%%%%%%%%%%%%%%%%%%%%%%%%%%%%%%%%%%%%%%%%%%%%

%%%%%%%%%%%%%%%%% BODY OF PAPER %%%%%%%%%%%%%%%%%%

\section{Introduction}

The existence of an ionizing radiation field can lead to Balmer hydrogen
emission lines. From the presence  of the H Balmer lines in the optical
spectra of some sources it is well known the possible presence of ionized
gas. Many important astronomical objects involve the physics of photo-ionized
gases and the interpretation of the emission-line spectra. Emission line
objects as the H II regions allow us to study the star formation history
of the far reaches of our Galaxy and of distant galaxies. Planetary nebulae
let us to see the remaining envelope of dying stars. Star-burst galaxies and
QSOs are one the most luminous objects and hence the most distant that can
be observed. Their spectra can reveal details about of the first generation
of star and the formation of heavy elements in the young universe. On the
other hand, emission lines can also infer the presence or lack the accretion
discs \citep{Schwope:2000, Ratti:2012}, the properties of single or double
picked line can allow us to infer geometrical characteristics \citep{Horne:1986},
the nature of  donor stars in binary system \citep{Steeghs:2002, Spaandonk:2010, Casares:2015}
and the compact objects as black holes \citep{Casares:2016}. 

Emission lines are also associated with stars in very early-type and/or
very late evolutionary stage which are short phase. As already mentioned
are also associated with binaries that experiencing mass transfer. These
group of emission line stars includes young stellar (YSOs) and Herbig-Haro
(HH) objects, post-asymptotic and some asymptotic giant branch (AGB), some
red giant stars (RGB), Wolf-Rayet (WR) stars, supernova remnants, classical
Be stars, active late-type dwarfs, interacting binary system like symbiotic
stars (SySt) and cataclysmic variables (CV). Most of these class of object
are in-homogeneous and some contains many few identified members, for
instance at the moment around 323 symbiotic system have been identified
from which 257 belong to the Galaxy and  $\sim$66 are extra-galactic
objects \citep{Akras:2019a}. The same occurs with PNe from witch around
3500 of them are been cataloged \citep{Parker:2016}, this current number
of PNe represents only about 15-30\% of the estimated total of Galactic
PNe (Frew, 2008; Jacoby et al., 2010) showing that a small fraction of the
PNe have been cataloged. Many galaxies, in addition to harbor Planetary
nebulae and H II regions, show characteristic nebular in their spectra.
In most of these objects, the gas is photoionized by hot stars in the nucleus,
which is thus much like giant H II region, or perhaps many H II regions.
The galactic nucleus with very strongest  emission lines of this type are
often called blue compact galaxies, extragaltic H II regions, star forming
or starburst galaxies \citep{Osterbrock:2006}. There are also spiral galaxies
that present emission lines.

In the past H$\lpha$ surveys with modest spatial resolutions have been used
to identified extended nebular emission to study supernova remnants, galaxy
groups and star forming regions (Davies, Elliott & Meaburn 1976). More recently,
higher resolution surveys such as the INT Photometric H$\alpha$ survey
(IPHAS; \citealt{Drew:2005, Barentsen:2014}) have focused in the study of
compact emission line sources on the Galactic plane, typically with objects
in different stage of stellar evolution. The Anglo-Australian Observatory UKS
chmidt Telescope Supercosmos H$\alpha$ Survey (Parker et al. 2005) is another
H{$\alpha$} survey of the Southern Galactic Plane and Magellanic Cloud which
has covered to b $\sim$ 10-13$^{\circ}$ (verificar esto). Currently ongoing is
the VST Photometric H$\alpha$ Survey of the Southern Galactic Plane and Bulge
(VPHAS+; \citealt{Drew:2014}) that will cover the Galactic bulge and plane in
five filters. 

Like VPHAS+, others ongoing surveys that are used to study the population of
emission line objects are the The Javalambre Photometric Local Universe Survey
(J-PLUS\footnote{\url{https://www.j-plus.es}}, \citealp{Cenarro:2018})
and the Southern-Photometric Local Universe Survey
(S-PLUS\footnote{\url{http://www.splus.iag.usp.br}}, \citealp{Mendes:2019})
are providing observations of the Galactic halo covering both northern and
southern celestial hemispheres in a systematic way with twin telescopes
using the same set of multi-band filters. In addition to the H$\alpha$ filter,
which is already vastly applied to systematically searching for H$\alpha$ emitters
the telescopes offer 11 more filters. And more ambitious yet the JPAS survey that
will the same area of J-PLUS in 56 narrow-band filters.

Traditionally, color-color diagrams based in H$\alpha$ filter are been used to
identify H$\alpha$ emitters.  The analysis the color-color diagram  (r - H$\alpha$)
versus (r - i) has resulted on the discoveried of new emission line objects, for
instance \citet{Witham:2006, Witham:2007}  used the (r - H$\alpha$) versus (r - i)
colour-colour diagram to find for new CV. On the other hand, \citet{Vink:2008}
reported the discovery of YSOs by using this same colour criteria. In this sense using
this methology a variety of classes of objects are been identified, which include
symbiotic stars \citep{Corradi:2008, Corradi:2010, Corradi:2011}, early type emission
line stars \citep{Drew:2008} and planetary nebulae \citep{Viirone:2009, Sabin:2010}.
Recently, by using this same color diagram were also identified compact PN candidates
in VPHAS+ catalog \citep{Akras:2019}. And the same diagram in conjunction with new
ones shows to be very efficient to find for PN candidates \citep{Gutierrez:2020}.
In general terms, \citet{Witham:2006} presented a methodology and first results
in looking for emission line sources in narrow-band surveys.

In this era of big data on astronomy, machine learning techniques are becoming
in important statistical tools for the analysis and find meaning from massive
data sets. Particularly, unsupervised machine approaches have showed a promised
in various applications, especially in automatic classification task. Including
object classification and selection, using galaxies with active galactic nuclei
as example \citep{Geach:2012}, morphological analysis of galaxies \citep{Martin:2020},
classification of variable stars, relying only on the similarity among light
curves. \citep{Valenzuela:2018}. Using unsupervised machine
learning can be very advantageous because they do not require a labeled data
training sets. Unlike of supervised methods like Random Forest algorithm.
Instead, unsupervised techniques are generally based in the data itself to
identify patrons, e. g. cluster of similar objects, in some pre-defined
feature space where the data are defined.

In this work, we used S-PLUS observations of the southern hemisphere to search
for objects with an excess of H{$\alpha$} using automatic methods based on the
(r - H$\alpha$) versus (r - i) color-color diagram we also used color criteria
based in (g - r) and (z - g) in conjunction to unsupervised machine learning
techniques to split the final list in those with blue and red continuum. The
paper is organized as follows...

\section{Observations}
\label{sec:obser}

Particularly, we are implemented data from S-PLUS DR3 (ref) to carried out our
study. S-PLUS is 12-band optical photometric survey, which are formed by using
seven narrow-band and five brow-band like SDSS filters. The narrow-band set
include the filter $J$0660 which detect the H{$\alpha$} emission line.
Figure~\ref{fig:curves} shows the Javalambre filter system (Marín-Franch et al. 2012)
overlapping are the optical spectra of several class emission line objects on which
it is possible to see that the H{$\alpha$} line falls into the $J$0660 filter, except
for the QSOs.   

The actual data release contains about 60 millions of objects covering a total
area of $\sim$8000 deg$^2$, at high Galactic latitudes ($ > 30$~deg) using a
dedicated 0.83m robotic telescope,the T80-South (T80S), located at Cerro Tololo,
Chile. S-PLUS will cover an additional 1300 deg$^2$ of the Galactic plane and bulge
toenable Galactic studies. In this work, we focus on the aspects thatare of
particular interest to the second data release of the S-PLUS main survey.
Additional information about S-PLUS can be found in \citet{Mendes:2019}. 

\begin{figure}
    \includegraphics[width=0.9\linewidth]{Figs/splus-filter.pdf}
    \caption{Transmission curves of the S-PLUS filters set. The narrow-band filter
      $J0660$ detects the H$\alpha$ emission line. Over-plotted are different
      classes of emission line objects, from upper to down PN, SySt ... }
    \label{fig:curves}
\end{figure}


\section{Methodology}
\label{sec:metho}

We first constructed a sub-sample from all S-PLUS DR3 from which we
applied an iterative and automatic technique to select objects with
an excess of H{$\alpha$} emission line, as we describe below:

\subsection{Initial selection sample}
\label{sec:}

The firts step in our selection procedure consist in the following
criteria to guarantee the quality of the observations of the objects:

\begin{enumerate}
\item The sources must have detection in the filters: $r$, $i$ and
  $J$0660. To assure that we select object must have error minor or
  equal to 0.2 in each of three filters.

\item Must have an $r$ magnitude until $r = 21$.
  
\end{enumerate}

\subsection{Finding the main stellar locus and selecting the H{$\alpha$ emitters} }
\label{sec:}

Once the initial cut were made, we proceed to select the objects with
an excess of H{$\alpha$} wich is represent with a relatively high value of
the filter $J$0660 in comparison with r-band filter. For that we
first divided our sub-sample in for magnitude bins using the $r$-band
magnitudes. The bins have the follow distribution:

\begin{itemize}
\item 1 bin- objects with magnitude in the $r$-band $r < 16$
\item 2 bin- objects with magnitude in the $r$-band $16 \leq r < 18$
\item 3 bin- objects with magnitude in the $r$-band $18 \leq r < 20$
\item 4 bin- objects with magnitude in the $r$-band $20 \leq r < 21$
  
\end{itemize}

To select the emission lines we used the same method created and
implemented by \citet{Witham:2008} its possible to do that because
the S-PLUS has similar filters that the IPHAS project, which are
$r$, $J$0660 and $i$. This technique was used by \citet{Scaringi:2013}
to identify blue objects with excess of H{$\alpha$} and after that
\citet{Wevers:2017} also applied this methodology to create catalogue
of candidate H{$\alpha$} emission showing an high effectiveness. In this
order of ideas we attempted this methodology in S-PLUS.

We first generated the ($r$ - $J$0660) versus ($r$ - $i$) color- color
diagram for each magnitude bins. We then carried a out an initial straight
line fit to all objects in each magnitude bin. This initial fit is an
attempt to find the loci of main-sequence and giant stars. We implemented
a iterative $\sigma$-clipping tecnique to find the best-fitting of the main
stellar locus. In this order of ideas, we made four interactive $\sigma$-clipped.
Once we have found the apropiate fitting for each maginitude bin, we identified
those objects significantly above of this final fit as likely sources with
a excesses of H{$\alpha$}. Objects with a contribution significative of H{$\alpha$}
meet the condition:

\begin{equation}
  (r - J0660)_{\mathrm{obs}} - (r - J0660)_{\mathrm{fit}} \geq C \times \sqrt{\sigma^2_{\mathrm{s}} - \sigma^2_{\mathrm{phot}}}
  \label{eq:criterion}
\end{equation}
 
 where $\sigma_{\mathrm{s}}$ is the root mean squared value of the residuals around
 the fit and $\sigma_{\mathrm{phot}}$ is the error on the observed $(r - J0660)$ colour idex.
 $C$ is a constant which has the value 4 following \citet{Wevers:2017}.
The fits are carried out with the aid of the python library \texttt{astropy.modeling}
\footnote{https://docs.astropy.org/en/stable/modeling/index.html}.

Figure~\ref{fig:criteria-color-plot} ilustrates the procedured used to slected the H{$\alpha$}
emitters in S-PLUS DR3 for each magnitude bin. The continuos black lines represent the initial
fit and  the dashed lines indicate the 4-$\sigma$ clipping fit lines. The dotted lines are
the cut selection criteria for the H{$\alpha$} emitters -- the 4-$\sigam$ above of the final
fit--. Note that theses cut lines are only an approximations because to trace these lines
it is only considered the residual around the fit. The actual selection criteria used here
also include the phothometric uncertainties in $r$ - $J$0660 for each individual data source as
shows on the Equation~\ref{eq:criterion}.

\begin{figure*}
  \setlength\tabcolsep{0pt}
  \setkeys{Gin}{width=0.5\linewidth}
  \begin{tabular}{ll}
    (a) & (b) \\
    \includegraphics[trim=10 0 65 20, clip]{Figs/diagram-DR3-errorFlag0-3f-16r}
    & \includegraphics[trim=10 0 65 20, clip]{Figs/diagram-DR3-errorFlag0-3f-16r18}\\
    (c) & (d) \\
    \includegraphics[trim=10 0 65 20, clip]{Figs/diagram-DR3-errorFlag0-3f-18r20}
    & \includegraphics[trim=10 0 65 20, clip]{Figs/diagram-DR3-errorFlag0-3f-20r21}\\
  \end{tabular}
  \caption{An illustration of the selection criteria used to identify strong
    emission-line objects via colour-colour plots. The data shown here are all from the
    S-PLUS DR3. The data are split up into four magnitude bins, as shown in the four
    panels. Objects with H{$\alpha$} excess should be located near the top of the
    colour-colour diagrams. The thin red lines illustrate the original linear
    fit to all the data (grey points). The dashed lines represent the final
    fits to the stellar locus of points which were obtained by applying an iterative
    $\sigma$-clipping technique to the initial fit. The actual cuts used to select H{$\alpha$}
    emitters are shown by the dotted lines. Objects selected as H{$\alpha$} emitters
    must be located above. Note that the cut lines (selection criteria) shown here
    are only approximate, as the actual selection criterion also considers the errors on
    each source. This means that an object could be in the bottom
    right-hand panel is not selected despite clearly lying above the cut line
    ({\sc explicar esta última frase mejor}).}
  \label{fig:criteria-color-plot}
\end{figure*}

After the algorithm was applied to all data, we visually inspection-ed the
resulting list seeing the S-spectra and corresponding colored image.
Figure~\ref{fig:Spectra} shows an example of how looks like a
S-spectra\footnote{S-spectra signify the S-PLUS emission in flux
or magnitude unities of an object in all twelve bands.} of sources in
magnitude unities, clearly this object exhibits strong H$alpha$ emitter.

\begin{figure}
\includegraphics[width=0.9\linewidth]{Figs/photopectrum_splus_SPLUS-n15s22-024043_Good-LD-Halpha-DR3_noFlag_merge-takeoutbad-Final_PStotal.pdf}
\centering
\llap{\shortstack{%
        \includegraphics[width=0.25\linewidth, trim=350 10 350 20, clip]{Figs/SPLUS-n15s22-024043_179.74240913634958--19.02992833661004_250_r.pdf}\\
        \rule{0ex}{0.91in}%
      }
  \rule{0.06in}{0ex}}
\caption{S-spectra of a random object observed by S-PLUS. Squares represent
  the SDSS-like broad-band filters. From left to right they are \(u, g, r,
  i~\text{and}~ z\). Circle symbols are the narrow-band filters, which from
  left to right represent \textit{J}0378, \textit{J}0395, \textit{J}0410,
  \textit{J}0515, \textit{J}0660 and \textit{J}0861.}
\label{fig:Spectra}
\end{figure}

Once, we felt confident of our sample of H{$\alpha$} emission lines sources,
we proceeded to classify the objects into two big groups; one group containing those
objects with a strong blue continuum and another with an intense emission of the
continuum on the blue part of the spectrum. 


\subsection{Clustering techniques}
\label{sec:clustering}

In this work we are using unsupervised machine learning approaches to divided our sample
in two groups: one represent the blue sources and the another one are the red sources.
The blue are those that have strong emission of the continuum on the part blue of the
spectrum and the another with strong emission continuum of the red one. 

\subsubsection{Hierarchical clustering}
\label{sec:Hierar}

Hierarchical clustering also known as hierarchical cluster analysis ({\sc hca}),
is an algorithm that groups similar objects into groups called clusters.
The endpoint is a set of clusters, where each cluster is distinct from each
other cluster, and the objects within each cluster are broadly similar to each other.

Choosing the number of cluster. Firstly, the Hierarchical cluster output dendrogram (tree)
can be implemented to obtain the desired clustering. Secondly, the dendrogram schema  allows
a convenient way to establish the entity relationship between at all levels of granularity.
In conclusion, a dendrogram is a visualization in form of a tree showing the order
and distances of merges during the hierarchical clustering.

\begin{itemize}
  
     \item On the $x$ axis you see labels. If you do not specify anything else they
       are the indices of your samples in X.
     \item On the $y$ axis you see the distances (of the ``ward'' method in our case).
       
\end{itemize}


\subsubsubsection{Dendrogram Truncation}

As you might have noticed, the above is pretty big for many samples already and you
probably have way more in real scenarios, so let me spend a few seconds on highlighting
some other features of the dendrogram() function:


Starting from each label at the bottom, you can see a vertical line up to a
horizontal line. The height of that horizontal line tells you about the distance
at which this label was merged into another label or cluster. You can find that
other cluster by following the other vertical line down again. If you don't
encounter another horizontal line, it was just merged with the other label
you reach, otherwise it was merged into another cluster that was formed earlier.

Hierarchical clustering can be performed with either a distance matrix or raw data.
When raw data is provided, the software will automatically compute a distance
matrix in the background. The distance matrix below shows the distance between
six objects.

Hierarchical clustering starts by treating each observation as a separate cluster.
Then, it repeatedly executes the following two steps: (1) identify the two
clusters that are closest together, and (2) merge the two most similar clusters.
This iterative process continues until all the clusters are merged together.
This is illustrated in the diagrams below.

Hierarchical clustering 2

\subsubsection{Hierarchical density-based cluster selection}
\label{sec:hdbscan}

Hierarchical density-based cluster selection (\texttt{HDBSCAN}; \citealp{McInnes:2017})
Hierarchical Density-Based Spatial Clustering of Applications with Noise (Campello,
Moulavi, and Sander 2013), (Campello et al. 2015). Performs \texttt{DBSCAN} overvarying
epsilon values and integrates the result to find a clustering that gives the beststability
over epsilon. This allows \texttt{HDBSCAN} to find clusters of varying densities
(unlike \texttt{DBSCAN}), and be more robust to parameter selection. The library also
includes supportfor Robust Single Linkage clustering (Chaudhuri et al. 2014),
(Chaudhuri and Dasgupta2010), GLOSH outlier detection (Campello et al. 2015), and
tools for visualizing and exploring cluster structures. Finally support for prediction
and soft clustering is also available.

There have been few times that it has been used \texttt{HDBSCAN} in astronomy.
Recently, \citet{Ntwaetsile:2021} used it to group radio sources into a sequence
of morphological classes, illustrating a simple methodology to classify and
label new, unseen galaxies in large samples.

\section{Results}
\label{sec:results}

\subsection{Simbad}

We made cross-match between our sample and Simbad. We found 1000
matches that include a big variety of emission line objects:

\subsubsection{Nebulae}

As was mentioned, objects with nebulousity include several
type of objects from H II region, planetary nebulae and galaxies.
The H II regions are objects with gas that being ionized by
O star on which are formed the emission lines. In this situation new
stars are formed. Unlike H II regions, planetary nebulae
represent the final stages of low- and intermediate-mass stars
where the gas previously ejected is ionized by their central star.


\subsection{SDSS}

\subsection{Lamost}

\begin{figure*}
	\includegraphics[width=0.9\linewidth]{Figs/final-emitters.pdf}
        \caption{Colour-colour diagram with all the emission line objects selected
          from S-PLUS DR3. Size of the symbols represent the measured FWHM assuming
          a Gaussian core (for more detail see \citealt{Fernandes:2021}). Colored
          bar indicates the magnitude values in the r-band. The contours represent
          the synthetic main-sequence and giant stars loci from the library of stellar
          spectral energy distributions of \citet{Pickles:1998}.}
    \label{fig:emission}
\end{figure*}

\begin{figure*}
\includegraphics[width=0.9\linewidth]{Figs/halpha-emitters-galactic-aitoff.pdf}
\centering
\llap{\shortstack{%
        \includegraphics[width=0.25\linewidth, trim=10 10 -40 0]{Figs/distribution-bgalactic.pdf}\\
        \rule{0ex}{1.8in}%
      }
  \rule{1.0in}{0ex}}
\caption{Distribution of H{$\alpha$} emitters in Galactic longitude and latitude.
  The emitters are shown as red points if brighter than r = 18, and black points if
  fainter. The S-PLUS direct fields are shown by green squares (offset fields are
  not shown). All emitters are shown here, including those with flagged with 'c' in
Table 1.}
\end{figure*}


\begin{figure*}
  \begin{tabular}{l l l}
  \includegraphics[width=0.7\columnwidth]{Figs/distribution-Halpha.pdf} 
    \includegraphics[width=0.7\columnwidth]{Figs/distribution-ri.pdf}
    \includegraphics[width=0.7\columnwidth]{Figs/distribution_r.pdf}
  \end{tabular}
    \caption{Emission lines selected...}
    \label{fig:diagram-distri}
\end{figure*}

\begin{figure}
	\includegraphics[width=0.9\linewidth]{Figs/Fig-SPLUS-gr-zg.pdf}
    \caption{Classifying...}
    \label{fig:synthetic}
\end{figure}

\begin{figure}
	\includegraphics[width=0.9\linewidth]{Figs/red-blue-colorObjects-gr-edit.jpg}
    \caption{Classifying...}
    \label{fig:new-color}
\end{figure}


\begin{figure}
	\includegraphics[width=0.9\linewidth]{Figs/Customer-Dendrograms.pdf}
    \caption{Costomer dendrogram...}
    \label{fig:dendrogram}
\end{figure}

\begin{figure}
	\includegraphics[width=0.9\linewidth]{Figs/blued-red-hierarchical.pdf}
    \caption{Costomer dendrogram...}
    \label{fig:hierar}
\end{figure}

\begin{figure*}
\centering
\begin{tabular}{l l}
  \includegraphics[width=0.5\linewidth, trim=10 10 5 8, clip]{Figs/blued-red-hdbscan.pdf}
   \includegraphics[width=0.5\linewidth, trim=10 10 5 8. clip]{Figs/blue-red-hdbscan-soft-alternative.pdf}
  \end{tabular}  
  \caption{New color-color diagram to separate the blue objects from the red ones.}
\label{fig:hdbscan}
\end{figure*}


\begin{figure*}
  \setlength\tabcolsep{0pt}
  \setkeys{Gin}{width=0.5\linewidth}
  \begin{tabular}{ll}
    (a) & (b) \\
    \includegraphics[trim=10 0 10 20, clip]{Figs/StenholmAcker_pn_g006_0-41_9_id176-SPLUS-s29s46-072842.pdf}
    & \includegraphics[width=0.4\linewidth, trim=10 0 10 20, clip]{Figs/PNG006_316-37_100_r.pdf} \\
    \includegraphics[trim=10 0 10 20, clip]{Figs/spec-0680-52200-0153-STRIPE82-0159-019049.pdf}
    & \includegraphics[width=0.4\linewidth, trim=10 0 10 20, clip]{Figs/GALEX24170_351-0_150_r.pdf} \\
    (c) & (d) \\
    \includegraphics[trim=10 0 10 20, clip]{Figs/spec-0376-52143-0631-STRIPE82-0142-027354.pdf}
    & \includegraphics[width=0.4\linewidth, trim=10 0 10 20, clip]{Figs/FASTT1560_338-0_100_r.pdf} \\
    \includegraphics[trim=10 0 10 20, clip]{Figs/spec-0397-51794-0336-STRIPE82-0026-058736.pdf}
    & \includegraphics[width=0.4\linewidth, trim=10 0 10 20, clip]{Figs/LEDA1185205_17-1_200_r.pdf} \\
    \includegraphics[trim=10 0 10 20, clip]{Figs/spec-9217-57934-0839-STRIPE82-0143-016137.pdf}
    & \includegraphics[width=0.4\linewidth, trim=10 0 10 20, clip]{Figs/PHL354_339-0_100_r.pdf} \\
  \end{tabular}
  \caption{Spectra of the known objects select with our algorithm }
  \label{fig:color-diagram}
\end{figure*}

\begin{figure*}
  \setlength\tabcolsep{0pt}
  \setkeys{Gin}{width=0.5\linewidth}
  \begin{tabular}{ll}
    (a) & (b) \\
    \includegraphics[trim=10 0 5 10, clip]{Figs/StenholmAcker_pn_g006_0-41_9_id176-SPLUS-s29s46-072842.pdf}
    & \includegraphics[width=0.2\linewidth, trim=10 0 5 5, clip]{Figs/PNG006_316--37_100_F660-RGB.pdf} \\
    \includegraphics[trim=10 0 5 10, clip]{Figs/spec-0680-52200-0153-STRIPE82-0159-019049.pdf}
    & \includegraphics[width=0.2\linewidth, trim=10 0 5 5, clip]{Figs/GALEX24170_351-0_200_F660-RGB.pdf} \\
    (c) & (d) \\
    \includegraphics[trim=10 0 5 10, clip]{Figs/spec-0376-52143-0631-STRIPE82-0142-027354.pdf}
    & \includegraphics[width=0.2\linewidth, trim=10 0 5 5, clip]{Figs/FASTT1560_338-0_80_F660-RGB.pdf} \\
    \includegraphics[trim=10 0 5 10, clip]{Figs/spec-0397-51794-0336-STRIPE82-0026-058736.pdf}
    & \includegraphics[width=0.2\linewidth, trim=10 0 5 5, clip]{Figs/LEDA1185205_17-1_200_F660-RGB.pdf} \\
    \includegraphics[trim=10 0 5 10, clip]{Figs/spec-9217-57934-0839-STRIPE82-0143-016137.pdf}
    & \includegraphics[width=0.2\linewidth, trim=10 0 5 5, clip]{Figs/PHL354_339-0_80_F660-RGB.pdf} \\
  \end{tabular}
  \caption{Spectra of the known objects select with our algorithm }
  \label{fig:color-diagram}
\end{figure*}

\begin{figure*}
  \setlength\tabcolsep{0pt}
  \setkeys{Gin}{width=0.5\linewidth}
  \begin{tabular}{ll}
    (a) & (b) \\
    \includegraphics[trim=10 0 10 20, clip]{Figs/spec-57313-EG220318S020919M01_sp10-173-SPLUS-s03s05-011958.pdf}
    & \includegraphics[width=0.4\linewidth, trim=10 0 10 20, clip]{Figs/SPLUS-s03s05-011958_329-3_100_r.pdf} \\
    \includegraphics[trim=10 0 10 20, clip]{Figs/spec-55893-F9304_sp15-198-STRIPE82-0057-001810.pdf}
    & \includegraphics[width=0.4\linewidth, trim=10 0 10 20, clip]{Figs/STRIPE82-0057-001810_39-1_100_r.pdf} \\
    (c) & (d) \\
    \includegraphics[trim=10 0 10 20, clip]{Figs/spec-57336-EG034838N001340M01_sp09-025-STRIPE82-0084-014280.pdf}
    & \includegraphics[width=0.4\linewidth, trim=10 0 10 20, clip]{Figs/STRIPE82-0084-014280_57-0_80_r.pdf} \\
  \end{tabular}
  \caption{Spectra of the Lamost }
  \label{fig:color-diagram}
\end{figure*}

\begin{figure*}
  \setlength\tabcolsep{0pt}
  \setkeys{Gin}{width=0.5\linewidth}
  \begin{tabular}{ll}
    (a) & (b) \\
    \includegraphics[trim=10 0 10 20, clip]{Figs/spec-0331-52368-0449-SPLUS-n02s23-034336.pdf}
    & \includegraphics[width=0.4\linewidth, trim=10 0 10 20, clip]{Figs/SPLUS-n02s23-034336_180-1_200_r.pdf} \\
    \includegraphics[trim=10 0 10 20, clip]{Figs/spec-9152-58041-0463-STRIPE82-0147-005730.pdf}
    & \includegraphics[width=0.4\linewidth, trim=10 0 10 20, clip]{Figs/STRIPE82-0147-005730_343-1_100_r.pdf} \\
    (c) & (d) \\
    \includegraphics[trim=10 0 10 20, clip]{Figs/spec-1089-52913-0196-STRIPE82-0007-024265.pdf}
    & \includegraphics[width=0.4\linewidth, trim=10 0 10 20, clip]{Figs/STRIPE82-0007-024265_3-0_100_r.pdf} \\
  \end{tabular}
  \caption{Spectra of the SDSS}
  \label{fig:color-diagram}
\end{figure*}

\section{Conclusions}

We have found a important sample of emission line objects.

\section*{Acknowledgements}


%%%%%%%%%%%%%%%%%%%%%%%%%%%%%%%%%%%%%%%%%%%%%%%%%%
\section*{Data Availability}


%%%%%%%%%%%%%%%%%%%% REFERENCES %%%%%%%%%%%%%%%%%%

% The best way to enter references is to use BibTeX:

\bibliographystyle{mnras}
\bibliography{ref} % if your bibtex file is called example.bib


%%%%%%%%%%%%%%%%%%%%%%%%%%%%%%%%%%%%%%%%%%%%%%%%%%

%%%%%%%%%%%%%%%%% APPENDICES %%%%%%%%%%%%%%%%%%%%%

\appendix
\section{Condensed Trees}
The question now is what does the cluster hierarchy look like - which
clusters are near each other, or could perhaps be merged, and which
are far apart. We can access the basic hierarchy via the \texttt{condensed\_tree\_}
attribute of the clusterer object. It is possible to see that \texttt{HDBSCAN} has found
two cluster that in agreement with previous results are the blue and red sources.

\begin{figure*}
	\includegraphics[width=0.9\linewidth]{Figs/cluster-hierarchy-hdbscan.pdf}
        \caption{Branches were selected by the HDBSCAN* algorithm.}
    \label{fig:emission}
\end{figure*}


\newcommand\TableHeader{
  \hline\hline
  Id Simbad & \(\mathrm{RA}\) & \(\mathrm{Dec}\) & Type & Group -- {\sc hca}
                          & P(Blue) -- {\sc hdbscan} & P(Red) -- {\sc hdbscan} \\
  %Object & \(D\) &   \(R_{\mathrm{out}}\) & \(R_{\mathrm{in}}\) \\
  \hline 
}

\clearpage
\section{Simbad objects}

\begin{center}
\begin{longtable}{llclccc}
  \caption{Simbad sources. \label{tab:simbad}}\\
  \TableHeader\endfirsthead 
  \caption[]{--continued}\\
  \TableHeader\endhead
  \hline \endfoot
%% \begin{table}
%% \begin{tabular}{cccc}
%% main_type & RA & DEC & Label_hier \\
Star & 0.4968921619628843 & -29.3112214428821 & Blue \\
QSO & 0.6279554824066291 & 0.8331230804411939 & Blue \\
ClG & 0.6982473212090481 & -0.373204977953541 & - \\
Star & 1.286523747867553 & -30.851162659705377 & Red \\
QSO & 1.658281373282977 & -0.6156060790712984 & - \\
QSO & 1.791687271704504 & 0.8914261244525057 & Blue \\
QSO & 2.038930048182944 & 0.8264638152844974 & - \\
Galaxy & 2.328179105177891 & -0.6519420575037337 & Blue \\
QSO & 2.666987531261032 & -29.740921877458867 & Blue \\
Star & 2.7030486187917577 & -29.791327943410987 & Red \\
Galaxy & 2.7306881758399904 & -30.739852073510534 & Blue \\
QSO & 3.1199364195988752 & -31.04443153319788 & Blue \\
Seyfert 1 & 3.3638162881857334 & 0.875614770300196 & Blue \\
EmG & 3.6199743565125737 & -0.7455029176005269 & Blue \\
Star & 3.7333117402685474 & 0.3176892728434901 & Blue \\
QSO & 3.860510825587982 & 0.3037236200425542 & Blue \\
QSO & 3.898110576074378 & 0.8989176411595012 & Blue \\
Galaxy & 4.117681731557023 & 1.1338942388667368 & Blue \\
QSO & 4.174451424491121 & -31.44905004903269 & Blue \\
QSO & 4.380258459418438 & -0.8164386494043362 & Blue \\
StarburstG & 4.416543859611166 & 0.5062627804218517 & Blue \\
QSO & 4.474256625814649 & 0.8493703288672522 & - \\
QSO & 4.801640792188885 & 0.0554916881824542 & Blue \\
QSO & 4.917646791908225 & -0.9099503373848068 & Blue \\
QSO & 4.958578954314056 & -0.6779700699260857 & - \\
QSO & 5.657902318221463 & 0.0886788337120439 & Blue \\
EmObj & 6.280815147745145 & 0.3125654118801338 & Blue \\
Seyfert 1 & 6.333011956389121 & 0.525479973176037 & Blue \\
Galaxy & 6.974318052697582 & -0.9667199252402 & Blue \\
Galaxy & 7.320040644293336 & -1.0064266381068807 & Blue \\
QSO & 7.416749471161749 & 1.0913027872498264 & Blue \\
AGN & 7.464354807727703 & 0.7000037288417243 & Red \\
QSO & 7.823701945596341 & 0.2847385223835051 & - \\
Unknown & 7.906256073243137 & -29.47091029710251 & Blue \\
Galaxy & 7.961265608895669 & -28.92685732854085 & Blue \\
Galaxy & 7.969005194202078 & -29.592587811984853 & Blue \\
QSO & 8.035561688713287 & -0.8843461346218345 & Blue \\
QSO & 8.144241522878199 & -0.2658504803749659 & Blue \\
Galaxy & 8.144555784075553 & -42.66955742286215 & Blue \\
QSO & 8.178072183921541 & 0.5197450140043085 & Blue \\
Galaxy & 8.477954545229275 & -29.936853996206867 & Blue \\
AGN & 8.821276175970617 & -42.088597877208215 & Red \\
QSO & 8.941080203860544 & 0.3849934246315855 & Blue \\
AGN & 9.10579672267925 & -0.4853030603908992 & Red \\
Galaxy & 9.160164736675396 & -32.5790767643802 & Blue \\
QSO & 9.308785545285224 & -0.9344374247364892 & - \\
QSO & 9.342361420176386 & -0.1946009608261733 & Blue \\
EmObj & 9.4213937498701 & 0.5555554739851708 & Blue \\
QSO & 9.747289111980264 & -0.714454350224955 & Blue \\
BlueCompG & 9.876175084523542 & 1.3391251468348433 & Blue \\
FIR & 9.89493608957293 & 0.8602458369384121 & Blue \\
FIR & 9.89493608957293 & 0.8602458369384121 & Blue \\
Galaxy & 10.565064891250165 & -32.21589781391012 & Blue \\
QSO & 10.682789961127495 & 1.283933341653539 & Blue \\
Galaxy & 10.84118200857788 & -33.31747498120273 & Blue \\
CataclyV* & 10.896509451927638 & -0.6248809118328724 & Blue \\
QSO & 11.065875879315966 & -0.717518646194803 & - \\
QSO & 11.434797546075147 & -31.95811304109312 & Blue \\
Galaxy & 11.609577132522691 & -1.237997024045413 & Blue \\
QSO & 12.112387692202818 & -34.22741726899439 & Blue \\
QSO & 12.32716144605164 & 1.2191920530524625 & Blue \\
Seyfert 1 & 12.955915679701484 & 0.5649407893753414 & Blue \\
QSO & 12.98183382570476 & -26.962028288894874 & Blue \\
PartofG & 12.99884215031428 & -0.489110520414667 & Blue \\
Galaxy & 13.215039941128667 & -27.32576442209961 & Blue \\
QSO & 13.43231563480846 & 1.363190953245345 & Blue \\
Galaxy & 13.564745864622545 & -1.0822213962934328 & Blue \\
EmG & 13.667158618404784 & -32.011715009609304 & Blue \\
QSO & 13.683104869130515 & -30.51502373758133 & Blue \\
QSO & 13.883649917774902 & -31.260498014964096 & Blue \\
Galaxy & 13.922147889436662 & -0.9418439978440708 & Blue \\
UV & 13.96395390310537 & -30.945219315827195 & Blue \\
Star & 13.97147997409192 & -28.91591361083288 & Blue \\
Galaxy & 13.97214405926334 & -33.65042770377668 & Blue \\
QSO & 14.04137683954713 & -31.36906110248861 & Blue \\
QSO & 14.162698397718229 & -31.96627152132016 & Blue \\
Galaxy & 14.30251703197933 & -0.3660285011264472 & Blue \\
QSO & 14.668422466041417 & -30.03336355079838 & Blue \\
GinCl & 14.767074131559603 & 1.0011685238164214 & Blue \\
Galaxy & 14.806556179582149 & -34.32102363637064 & Blue \\
Galaxy & 14.90035476839902 & -30.344152784983983 & Red \\
QSO & 14.953354509073169 & -1.3181152771969302 & Blue \\
QSO & 14.971706058633467 & -39.53259262490825 & Blue \\
GinCl & 14.99825027722706 & -0.8658482234469952 & Red \\
Star & 15.018499579744676 & -33.65902649700721 & Blue \\
Unknown & 15.041426928322918 & -32.02530671766837 & Blue \\
Galaxy & 15.067378592944028 & -34.96127978456718 & Blue \\
Galaxy & 15.340682319480235 & -0.050506307384101 & Blue \\
QSO & 15.561054131595537 & -30.13160407774772 & Blue \\
Seyfert 1 & 15.625098200551925 & -0.5352093094283729 & Blue \\
GinCl & 15.632402188184084 & 1.3433615405781432 & Blue \\
QSO & 15.901621012754958 & -0.9191165595206996 & Blue \\
QSO & 16.05775922175709 & -1.264431931617225 & Blue \\
QSO & 16.580131544341818 & 0.8064950808416276 & Red \\
AGN & 16.722666519750696 & -32.72831437441763 & Blue \\
AGN & 16.74555738353547 & 1.0772660282004771 & Red \\
QSO & 16.77312563028888 & 0.1024885367041135 & Blue \\
LSB G & 16.943641045715655 & 1.0639605673889347 & Blue \\
BClG & 16.952571807415076 & 0.7482365853106138 & Red \\
GinCl & 17.256603326003162 & 1.3781935749384482 & Blue \\
GinCl & 17.256603326003162 & 1.3781935749384482 & Blue \\
QSO & 17.28164156677598 & 0.1138851962867193 & Blue \\
HII G & 17.28310789223492 & 1.12097992550518 & Blue \\
QSO & 17.32733457723796 & 0.90539845012828 & Blue \\
QSO & 17.35815775854618 & -0.6275003706886488 & Blue \\
EmG & 17.558224802220092 & -30.4123526694582 & Blue \\
EmG & 17.829265187901036 & -30.00504645851688 & Blue \\
QSO & 17.86813658313074 & 0.0286974483465913 & Blue \\
Galaxy & 18.052665880956955 & -33.94196296129968 & Red \\
QSO & 18.12729657912228 & 0.2449304176402998 & Blue \\
Star & 18.24172293238632 & 0.9769381424794108 & Blue \\
Galaxy & 18.304286168638374 & -32.436079929066175 & Blue \\
LSB G & 18.418416425105388 & 0.8774947989896621 & - \\
Seyfert 1 & 18.509796280365048 & -0.7974488232759221 & Blue \\
QSO & 18.521860772867413 & -31.150780875277498 & Blue \\
Galaxy & 18.65047255772875 & -32.644711309144114 & Blue \\
HII G & 18.882863357422373 & -0.8622960010814391 & Blue \\
Galaxy & 18.887732613726776 & -0.8596562230501371 & Blue \\
HII G & 18.89077699675322 & -0.8587455347938298 & Blue \\
HII G & 18.89077699675322 & -0.8587455347938298 & Blue \\
QSO & 18.92575064401607 & 0.3834537397700357 & Blue \\
Galaxy & 19.15999417596711 & -32.92752270653384 & Blue \\
Galaxy & 19.417523978322624 & -33.07795844094275 & Blue \\
Galaxy & 19.485155478290817 & -30.44051337374056 & Blue \\
Galaxy & 19.523785998210776 & -33.05253885507861 & Blue \\
QSO & 19.57552004108609 & 0.2487480482840988 & Blue \\
Seyfert 1 & 19.623433609428748 & 0.7637166618270906 & Blue \\
Galaxy & 19.70478983753148 & -33.33696862040094 & Blue \\
EmG & 19.97594947041034 & -34.24999710931423 & Blue \\
Galaxy & 20.04161862712166 & -33.23630428587631 & Blue \\
QSO & 20.29473653398262 & -0.8436434200227299 & Blue \\
Star & 20.46746328553068 & -33.937725015878605 & Red \\
HII G & 20.55777706221541 & 0.9587780047376696 & Blue \\
Galaxy & 20.571187870663927 & -34.04488260918208 & Blue \\
QSO & 20.611445353426262 & 0.0577366753714225 & Blue \\
QSO & 20.75741214062869 & 0.0565525553589094 & Blue \\
Galaxy & 20.9619712625909 & -29.19623005164965 & Blue \\
GinCl & 20.97812192371461 & 0.2823414770093598 & Blue \\
Galaxy & 20.984790111603804 & 0.2086134560133449 & Blue \\
Galaxy & 20.989463788795724 & -33.80208215339882 & Blue \\
Galaxy & 21.02387513879961 & 0.984703935513288 & - \\
QSO & 21.067433955055776 & -32.20603385094096 & Blue \\
Galaxy & 21.12566535772545 & -33.645961451411665 & Red \\
QSO & 21.26913199118856 & -32.28740231179497 & Blue \\
EmG & 21.3593347521788 & -30.742435844021443 & Blue \\
Galaxy & 21.45538057489708 & -28.162097768619763 & - \\
Galaxy & 21.612608797820563 & 0.9810769034670448 & Blue \\
Galaxy & 21.657201901148973 & -34.58716304788603 & Blue \\
HII G & 21.69378980441009 & -0.6457451930656755 & Blue \\
GinCl & 21.99710751467788 & -29.086653030427502 & Blue \\
GinCl & 22.36059472065752 & -1.1997118023036937 & Red \\
QSO & 22.642393025135533 & -0.3518162354106257 & Blue \\
Galaxy & 22.84101560993188 & -33.101715461339545 & Blue \\
Galaxy & 22.940224488235035 & -32.94910608953699 & Blue \\
Galaxy & 22.94684892850287 & -33.18197514224045 & Blue \\
Galaxy & 23.22263487815018 & -33.4451915146773 & Blue \\
low-mass* & 23.268819219724307 & 0.0655786809704273 & Red \\
Galaxy & 23.50191659447233 & -1.0664487372750466 & Blue \\
Galaxy & 23.716830840704777 & -0.6486782673628352 & Blue \\
QSO & 23.753457358462345 & -0.6817256241073724 & Blue \\
Seyfert 1 & 23.823057046706182 & -0.3275050653110525 & Blue \\
Galaxy & 23.8798401596094 & -31.614165292143262 & Blue \\
QSO & 24.257111717644538 & -1.3497465276851617 & Blue \\
QSO & 24.3725070057774 & -32.12103455786598 & Blue \\
QSO & 24.655329021886622 & 0.4717925959967649 & Blue \\
QSO & 24.962782843879992 & 0.4272217892643302 & Blue \\
Seyfert 1 & 25.07109927100977 & -0.8341590148622415 & Blue \\
QSO & 25.35682525075279 & 0.1321647769641862 & Red \\
QSO & 25.60303358518956 & -32.07047314351019 & Blue \\
QSO & 25.764557663482428 & -29.88188949191848 & Blue \\
EmG & 25.82619164196742 & -34.206232083806285 & Red \\
QSO & 26.83799104587179 & -0.7514735524191466 & Red \\
QSO & 26.913386215352126 & -28.883097657544933 & Blue \\
QSO & 26.9133952038632 & -28.88312791748778 & Blue \\
Galaxy & 27.02603800602231 & -0.4782147764500624 & Blue \\
QSO & 27.05101817682357 & 0.0315348188803727 & Blue \\
Star & 27.18405084268284 & -27.936387555535745 & Blue \\
Galaxy & 27.320953851728728 & -32.74253858762196 & Blue \\
QSO & 27.33969542295298 & -0.5391525524002503 & Blue \\
Galaxy & 28.165578926823763 & 1.0994164155371124 & Red \\
Galaxy & 28.224387503312013 & 1.2043017183289575 & Red \\
Seyfert 1 & 28.24067423260991 & -28.810501310105817 & Blue \\
QSO & 28.38271202195532 & 0.3813754396509161 & Blue \\
HII G & 28.502069390098665 & -0.752764335854246 & Blue \\
QSO & 28.53861595248377 & 0.4459129033998668 & Blue \\
QSO & 28.545594305017943 & -28.870735753209058 & Blue \\
QSO & 28.564494450701485 & -28.8819264781154 & Blue \\
EmG & 28.66852283441473 & -0.1121248724133715 & Blue \\
Galaxy & 28.86195614598901 & 0.1043810217993179 & Blue \\
Galaxy & 28.871127003162204 & -0.6575329793880781 & Blue \\
RRLyr & 29.55730413894375 & 1.028736096882655 & Blue \\
QSO & 29.63394027051608 & -30.284120909894927 & Blue \\
QSO & 29.63398878244968 & -30.28409277206394 & Blue \\
QSO & 29.70926847240945 & -30.077239118059776 & - \\
QSO & 29.89784479360552 & 0.0670730705671857 & Blue \\
QSO & 30.105854148548683 & 0.4879895823228443 & Red \\
QSO & 30.22926062206253 & -29.59069575111854 & Blue \\
GinGroup & 30.31036584086444 & -31.728583712955565 & Blue \\
QSO & 30.31471271654344 & 0.5264157435180954 & Blue \\
QSO & 30.500247645814035 & -0.155885605135301 & Blue \\
QSO & 31.14774462904211 & -45.99000661095606 & Blue \\
Galaxy & 31.25347770425113 & 1.401015050803151 & Blue \\
Galaxy & 31.83471909535146 & -33.03174281330757 & - \\
QSO & 32.01870359151427 & -0.0063899756832095 & Blue \\
QSO & 32.11278143734105 & -0.8688706450161886 & Blue \\
QSO & 32.34166379711758 & -0.9153964683450352 & Blue \\
Galaxy & 33.10437995801359 & -33.083042512325804 & - \\
Galaxy & 33.60091505746918 & -33.24777452929094 & Blue \\
Galaxy & 33.69844878804215 & -32.709786589316614 & Blue \\
QSO & 33.87092614743144 & -0.8874609989930496 & Blue \\
Galaxy & 34.006750737249284 & -31.61438842877632 & Blue \\
Galaxy & 34.05728735762157 & -30.84910309956719 & Blue \\
QSO & 34.07162177605453 & -1.17963313592842 & Blue \\
QSO & 34.543835265892945 & -1.0297901208560476 & Blue \\
CataclyV* & 34.866661573073486 & -30.76277662129932 & Blue \\
EmG & 35.3022376237971 & -30.43306611004649 & Blue \\
Galaxy & 35.82052533020276 & -1.0137779997035794 & Blue \\
Seyfert 1 & 36.07140219729569 & 0.10724100290268 & - \\
Galaxy & 36.21975140764737 & -34.10951561423316 & Blue \\
Galaxy & 36.37884288953809 & -0.8353138394540365 & Blue \\
PartofG & 36.61785108970139 & 1.1604400554801353 & Blue \\
PartofG & 36.61785108970139 & 1.1604400554801353 & Blue \\
Galaxy & 36.69277213609015 & -43.59156135457005 & Blue \\
EmG & 36.81029852296449 & 1.0934074799976583 & Blue \\
Galaxy & 36.83039522616293 & 1.025612852538597 & Blue \\
QSO & 36.90951623742346 & -31.607345845404307 & Blue \\
QSO & 36.99250641431996 & 0.0404421118141437 & - \\
Galaxy & 37.06050668375983 & -39.267782818411305 & Blue \\
PartofG & 37.1197026918213 & -1.1496158836885042 & Blue \\
Star & 37.43891173256027 & 0.1489862508458985 & Blue \\
Seyfert 1 & 37.47787637713116 & -30.599559376424764 & Blue \\
Star & 37.48751137869475 & -1.0089856734353149 & Red \\
Galaxy & 37.54069701310436 & -31.60505123369656 & Blue \\
Seyfert 1 & 37.58720416243717 & 0.2321681675793745 & Blue \\
Galaxy & 38.03891271958179 & -1.3861949894293877 & Red \\
Galaxy & 38.07534927116573 & -33.84528829588545 & Blue \\
QSO & 38.12763748929023 & -1.2817952492366265 & Blue \\
Galaxy & 38.174504835656805 & -39.295229245873685 & Blue \\
Galaxy & 38.20293944460951 & 0.8607798584028256 & Blue \\
Nova & 38.34423535712647 & 0.8498264300321089 & Blue \\
EmG & 38.39285833594685 & -39.04470809996566 & Blue \\
QSO & 38.39735766972906 & -1.1290537121536242 & Blue \\
Galaxy & 39.11980508074 & -0.9749932035748634 & Blue \\
QSO & 39.14871940890983 & -0.5342706372644691 & - \\
QSO & 40.24643683967374 & 0.7627476177593495 & Blue \\
QSO & 40.6454475160122 & -1.0644227483678184 & Blue \\
HII & 40.69556506354039 & 0.0239358351857076 & Blue \\
HII & 40.89837979736232 & 1.3771938278420643 & Blue \\
HII & 40.90704549503712 & 1.3729265625996037 & Blue \\
HII & 40.92809118446975 & 1.3595621460159046 & Blue \\
HII & 40.933458965510255 & 1.3779211213547329 & Blue \\
Galaxy & 41.52198188668681 & -33.08317001603085 & Blue \\
Star & 41.60058125535593 & -0.4980253582007756 & Blue \\
QSO & 41.60311859466317 & -0.5045181639339942 & Blue \\
PartofG & 41.60593375574607 & -0.5027112363938046 & Blue \\
Star & 41.61070793418088 & -0.5000978872359795 & Blue \\
QSO & 42.75267977777988 & 0.2853735115843281 & Blue \\
Galaxy & 43.06969129708544 & 0.2947725157476631 & Blue \\
QSO & 43.21668023139433 & -0.3698979850578264 & Blue \\
Seyfert 1 & 43.60887244004112 & -0.6896473847720612 & Blue \\
RadioG & 43.98850261320957 & 0.692640180601257 & Red \\
QSO & 44.030210339218165 & 1.1774555299992029 & Blue \\
HII & 44.10549886585857 & 0.5748223254603584 & Blue \\
Galaxy & 44.118458575759135 & 0.6078355659352473 & Blue \\
Galaxy & 44.439740352809245 & -33.482079149106816 & Red \\
Seyfert 1 & 44.79324003091252 & -0.3777292200872463 & Blue \\
Galaxy & 45.79093859633274 & 0.2271602138922271 & - \\
Galaxy & 46.05196517556931 & -1.1927199888695097 & Red \\
Galaxy & 46.07402391239703 & -0.8254163698523541 & Blue \\
Seyfert 1 & 46.144829680398814 & -0.4751996068898478 & Blue \\
QSO & 46.20769411378219 & -0.1370507568661265 & Blue \\
Blue & 46.2415646756668 & 0.9538832761700016 & Blue \\
Galaxy & 46.32601371119922 & -0.1594812724372881 & Blue \\
QSO & 46.55298185202796 & 1.3659244829765935 & Blue \\
AGN & 46.6217332236032 & -33.89231237580181 & Blue \\
Galaxy & 46.62637094367816 & -0.1063532167324183 & Red \\
Galaxy & 46.81501686424852 & 0.7312750476021798 & Blue \\
QSO & 46.98979992986573 & 0.120017369478221 & Blue \\
QSO & 47.87373270347458 & -0.2837361426333072 & Blue \\
WD* & 47.87885012806618 & -31.880860044582445 & Blue \\
GinGroup & 48.202558396796846 & -31.48630904526674 & Blue \\
Galaxy & 48.24315359852245 & -0.0815656430783389 & Blue \\
Galaxy & 48.48367671208701 & -31.470164460703632 & Blue \\
Galaxy & 48.61771428658176 & 0.7519402356960914 & Blue \\
Galaxy & 49.06378553118447 & -31.209241913161897 & Blue \\
Galaxy & 49.0750504981236 & -0.5069279648561904 & Blue \\
Galaxy & 49.21104988541157 & -33.30106752208543 & Blue \\
Galaxy & 49.21108348383686 & -33.30109956831036 & Blue \\
Galaxy & 49.62107027321726 & -0.0112548928671764 & Blue \\
QSO & 49.68820265523018 & -0.3125867996899615 & Blue \\
QSO & 49.905357996326025 & -0.4447129467729893 & Blue \\
QSO Candidate & 50.68707745560916 & 0.7450965554947673 & Blue \\
GinPair & 51.26729795433997 & -36.92773995131392 & Blue \\
Galaxy & 51.3044429528618 & -36.369356180322086 & Blue \\
GlCl & 51.48028596578508 & -32.88302604889567 & Blue \\
QSO & 53.10952432881614 & -1.1905549158813071 & Blue \\
QSO & 53.10958810574174 & -1.190642242028551 & Blue \\
Seyfert 2 & 53.292001605589874 & 0.1469919509442148 & Red \\
QSO & 53.74365145217437 & -0.1288517115468432 & Blue \\
QSO & 54.58960430404596 & 0.5184971700047292 & Blue \\
BLLac & 54.67234367033864 & -35.52621813852562 & Blue \\
BLLac & 54.67235585673245 & -35.52616012986378 & Blue \\
QSO & 54.86439460038415 & -34.61862342769125 & Blue \\
HMXB & 55.05160899500496 & -35.62780439037955 & Blue \\
EmG & 55.08289490156473 & 1.0585242193939064 & Blue \\
QSO & 55.09580981936138 & -35.26861884407053 & Blue \\
AGN & 55.21032155384233 & -35.439371817285185 & Blue \\
GinGroup & 55.38562960618687 & -34.88861116743838 & Blue \\
Seyfert 1 & 55.69883615683876 & 1.1591704436722594 & Blue \\
Star & 55.76939354280558 & 0.4200849545421101 & Blue \\
Galaxy & 56.01948427011065 & -38.13665123007838 & Red \\
QSO & 56.03437204036832 & -0.5182824235374773 & Blue \\
Galaxy & 56.11555619710486 & -0.4611633605478731 & Red \\
QSO & 56.32088432677072 & -0.2638128432166242 & Blue \\
GinCl & 56.43910392669589 & -36.34612610981142 & Blue \\
Seyfert 2 & 56.51053562993339 & -0.0162825970895807 & Red \\
GinCl & 56.84143331660593 & -32.85143564003734 & Red \\
LSB G & 57.28305368748847 & 1.162003553131273 & Blue \\
LSB G & 57.28697520415392 & 1.1628501269812663 & Blue \\
EB* & 57.830647615852506 & 0.5379309203257563 & Red \\
Galaxy & 58.760643529078 & -38.594508054321246 & Red \\
Candidate WD* & 58.81671653823677 & -37.49575812066213 & - \\
Star & 58.91021078569528 & 0.4763724886809515 & Blue \\
Galaxy & 59.023230308727214 & -49.47798028782121 & Blue \\
Galaxy & 59.21160650977977 & -0.2430266735645456 & Red \\
Galaxy & 59.342101075938615 & -37.03168069989505 & Blue \\
Galaxy & 59.384437936839966 & -0.0132210920265977 & Blue \\
Galaxy & 59.76629933224044 & 1.359321839400579 & Blue \\
Galaxy & 59.76629933224044 & 1.359321839400579 & Blue \\
EmG & 60.12243387703378 & -49.03011512394491 & Blue \\
AGN Candidate & 60.19197563815276 & -34.40769068554132 & Blue \\
Galaxy & 60.22136110841383 & -35.23783495731069 & Blue \\
QSO & 60.794009739694005 & -34.94910807950281 & Blue \\
Galaxy & 60.98645768730256 & -38.23292063597632 & Blue \\
Galaxy & 61.17143615831319 & -34.96550088599031 & Blue \\
Galaxy & 61.33498273596544 & -36.81638793661925 & Blue \\
Galaxy & 61.33500967007201 & -36.81633617582762 & Blue \\
Star & 61.366561383475336 & -38.18946988054663 & Blue \\
QSO & 62.877144254160285 & -33.89197020875841 & Blue \\
Galaxy & 62.9931906508685 & -37.97889641219351 & Blue \\
GinGroup & 63.22209217387362 & -31.30837649675762 & Blue \\
Galaxy & 63.24997813266718 & -38.32846190144912 & Blue \\
Candidate WD* & 65.02824515881092 & -32.85555378019325 & - \\
Galaxy & 65.23682931827716 & -32.84522864914376 & - \\
Galaxy & 66.49750313852422 & -43.27294370064416 & Blue \\
Galaxy & 66.63573911050027 & -41.03239942176444 & Blue \\
Galaxy & 66.68476952384738 & -42.261434192503216 & Blue \\
Galaxy & 66.68695800025141 & -42.09458791504701 & Blue \\
Galaxy & 66.68714006680887 & -42.97731957959581 & Blue \\
Galaxy & 66.92599861968759 & -42.63898741357326 & Blue \\
Galaxy & 67.11963551092221 & -43.24142208891916 & Blue \\
CataclyV* & 67.91485862835738 & -30.25392017250568 & Blue \\
Galaxy & 68.18465267691735 & -32.02224565575888 & Red \\
Galaxy & 68.29080626237631 & -43.77044714232464 & Blue \\
Galaxy & 68.83252077391391 & -42.20328938270354 & Blue \\
Galaxy & 68.92411946177829 & -47.878850433212456 & Blue \\
Galaxy & 69.29655862698584 & -46.68521526277068 & Blue \\
Galaxy & 69.3258708273691 & -32.36953198691898 & Blue \\
Candidate RRLyr & 69.3959956471664 & -44.34132589285379 & Blue \\
Galaxy & 69.59805855964451 & -33.08572978600444 & Blue \\
Galaxy & 69.86435708444397 & -42.98661407860815 & Blue \\
Galaxy & 71.43743287875098 & -38.64685520996582 & Blue \\
Galaxy & 72.09605095751637 & -44.88268692292141 & Blue \\
Galaxy & 72.5097904389203 & -47.47749473675231 & Blue \\
Galaxy & 72.95356751115321 & -46.6268828264031 & Blue \\
Star & 73.12966757047025 & -44.18451458170195 & Blue \\
Galaxy & 73.33107814604597 & -32.775789122825564 & Blue \\
Seyfert 1 & 73.67932821336544 & -48.22227714319606 & Blue \\
SN & 73.71975877176362 & -37.32095899536727 & Blue \\
GinPair & 73.75077658898451 & -37.25984603891258 & Blue \\
Galaxy & 73.8604762285104 & -30.5912705964922 & Blue \\
Galaxy & 149.25704125132305 & -26.49125060546518 & Blue \\
Galaxy & 149.27756719714128 & -19.11835403915 & Blue \\
Galaxy & 149.34776368562885 & -7.214281786532718 & Blue \\
Galaxy & 149.65458057970807 & -47.0834549455375 & Blue \\
Galaxy & 149.94509282349162 & -19.46665636988552 & Blue \\
RRLyr & 149.9619948222865 & -38.50635457434919 & Blue \\
Galaxy & 150.02266834541734 & -38.791294342178766 & Blue \\
GinGroup & 150.02430652185143 & -31.553009436144805 & Red \\
Galaxy & 150.20467895584443 & -30.544964090090623 & Blue \\
Galaxy & 150.24614431502653 & 3.46430477903397 & Blue \\
LSB G & 150.28779184781126 & -19.441570601534337 & Blue \\
Galaxy & 150.39201389787095 & -7.882130572619383 & Blue \\
Star & 150.54887788731048 & -19.426983577758755 & Blue \\
QSO & 150.56596957389203 & -0.1821573352244082 & Blue \\
Galaxy & 150.6057355105053 & 1.3268834909408125 & Blue \\
GinGroup & 150.6613029459264 & -45.49829273146848 & Red \\
HII & 150.7159322821982 & -26.15665080119176 & Blue \\
BlueSG* & 150.7278728767585 & -26.149887286064303 & Blue \\
HII & 150.7346337031606 & -26.14957887246477 & Blue \\
HII & 150.7477842615568 & -26.14623490714214 & Blue \\
HII & 150.74781937989135 & -26.146224890426577 & Blue \\
AGN Candidate & 150.75873809239062 & -26.149651149010563 & Blue \\
Galaxy & 150.7687797537192 & -19.82725437530812 & Blue \\
Galaxy & 150.81269716967137 & -5.9091034216348906 & Blue \\
BlueSG* & 150.8235011834385 & -26.167151664932163 & Blue \\
QSO & 150.9246975604601 & -15.135802684821307 & Blue \\
Galaxy & 150.96798139179825 & -31.41348352964001 & Blue \\
Galaxy & 151.0827329372396 & -44.425732478471055 & Blue \\
Galaxy & 151.28199229547042 & -19.85840135525412 & Blue \\
Galaxy & 151.32213147773072 & 1.639363549720132 & Blue \\
Galaxy & 151.36878333879326 & -38.12502118371568 & Blue \\
QSO & 151.41616300935794 & 4.154090690739466 & Blue \\
RRLyr & 151.45414160514633 & -25.69641389129041 & Blue \\
Galaxy & 151.57165594109023 & -6.5743506042744455 & Blue \\
Galaxy & 151.59250580663203 & -26.83255786101104 & Blue \\
GinGroup & 151.63078465018722 & -32.04330204360609 & Red \\
HII G & 151.63902577766424 & -29.93550371363763 & Blue \\
X & 151.64125084856218 & -29.93654830008841 & Blue \\
Galaxy & 151.79613569006344 & -19.06794716771129 & Blue \\
Galaxy & 151.863516303881 & -31.9241460721396 & Blue \\
EB* & 151.89099681523237 & -30.32206607484225 & Blue \\
CataclyV* & 151.894380024813 & -20.292330786813192 & Blue \\
CataclyV* & 151.894397417937 & -20.29234976425421 & Blue \\
Galaxy & 152.04483687127907 & -33.517284694292314 & Red \\
Galaxy & 152.09172791204222 & -14.810029537324146 & Blue \\
Galaxy & 152.12536783724758 & -26.359156873827526 & Blue \\
GinGroup & 152.30756133387325 & -43.00250347539357 & Red \\
Galaxy & 152.4594322607411 & -35.462014678338335 & Blue \\
Galaxy & 152.49472827763157 & -20.516514889740176 & Blue \\
Galaxy & 152.71002820639652 & -30.423435447476333 & Blue \\
Galaxy & 152.71586515599603 & -35.00780396726373 & Blue \\
EmG & 152.79126867563463 & -20.870559523696024 & Blue \\
Galaxy & 152.80618764806343 & -29.45774182077688 & Blue \\
EB* & 153.0033682767619 & -36.95700736727947 & Red \\
Galaxy & 153.0140154103735 & -32.46855221932521 & Blue \\
Star & 153.1982442378898 & -47.56420453243028 & Blue \\
Galaxy & 153.24854544281163 & -34.935163258478006 & Blue \\
EmG & 153.42462857062154 & -34.85508456238106 & Blue \\
Galaxy & 153.475351986868 & -34.73973403487098 & Blue \\
Galaxy & 153.60648341878957 & -30.708348670780456 & Blue \\
Galaxy & 153.61172675715056 & -23.484691315285414 & Blue \\
EmG & 153.6739002429698 & -44.85392470634676 & Blue \\
EmG & 153.68898998838594 & -34.059173447990155 & Red \\
Galaxy & 153.7005254616359 & -43.53042741275906 & Blue \\
Galaxy & 153.738831902119 & -43.61921964149586 & Red \\
EmG & 153.93560354915405 & -20.295547464050404 & Red \\
Star & 153.99297118154576 & -47.969198865203865 & Blue \\
Seyfert 2 & 154.07782142695768 & -33.56382825508867 & Red \\
EmG & 154.3048015423118 & -21.066758849089943 & Blue \\
Galaxy & 154.52385478212994 & -31.646946374240247 & Blue \\
QSO & 154.59067560656675 & -21.66881849744636 & Blue \\
Candidate CV* & 154.72293935897585 & -40.11213368993416 & Blue \\
Galaxy & 154.755114203685 & -37.67199013139984 & Blue \\
HII G & 154.8382251423767 & -22.14259842833651 & Blue \\
HII G & 154.83865197802905 & -22.1433030664676 & Blue \\
EmG & 154.92214062464984 & -25.814701960679475 & Blue \\
Galaxy & 155.11884791676323 & -23.47924382987377 & Blue \\
Galaxy & 155.13632707464745 & -23.448329902566297 & Blue \\
Candidate CV* & 155.1756706372574 & -33.83399007627319 & Blue \\
Star & 155.18046602174618 & -20.798507707460985 & Blue \\
EmG & 155.20376267846092 & -23.46585528316608 & Red \\
Galaxy & 155.28862213986295 & -32.86107252975438 & Blue \\
Galaxy & 155.3376370346406 & -21.607687357095795 & Blue \\
Galaxy & 155.50926414134003 & -39.87941443605516 & Blue \\
AGN Candidate & 155.66642433355923 & -30.49182167237173 & Blue \\
Galaxy & 155.74810048793466 & -42.82746318744498 & Blue \\
GinGroup & 155.75973043034236 & -39.16661255574749 & Blue \\
EmG & 155.91777428751132 & -35.825975288268005 & Red \\
EmG & 155.91778896569792 & -35.825982705450976 & Red \\
Galaxy & 156.08946269809243 & -43.917108333597646 & Blue \\
Seyfert 2 & 156.1309531167453 & -23.55266399516915 & Red \\
EB* & 156.30609371925286 & -35.67130335611048 & Blue \\
Galaxy & 156.53085826350642 & -24.555606178694944 & Blue \\
GinCl & 156.5904324509266 & -29.19937594593883 & Blue \\
CataclyV* & 156.77427199442948 & -43.72812711312928 & Blue \\
Galaxy & 156.83444311537312 & -23.8054382317716 & Blue \\
SN & 156.9597791681585 & -43.90576634134269 & Red \\
SN & 156.95987056162335 & -43.90578498083468 & Red \\
X & 156.96365603572264 & -43.89957006382954 & Blue \\
IG & 156.96371293405733 & -43.903895415957074 & Red \\
IG & 156.96375808362262 & -43.90381928622784 & Red \\
X & 156.9655292631724 & -43.90377424167152 & Blue \\
X & 156.96556060694496 & -43.90370677872753 & Blue \\
HII & 156.97034612111383 & -43.90320661166486 & Blue \\
HII & 156.9703863167105 & -43.90311040224713 & Blue \\
HII & 156.9703863167105 & -43.90311040224713 & Blue \\
AGN & 157.17900380093772 & -31.0382414041846 & Red \\
CataclyV* & 157.1827613944662 & -16.21759565677345 & Blue \\
RRLyr & 157.24187820994632 & -30.14102445233805 & Blue \\
GinGroup & 157.25296711210694 & -40.08275785582687 & Blue \\
EmG & 157.29613556479967 & -30.342418658284583 & Blue \\
Galaxy & 157.6274499803631 & -36.47975718945754 & Blue \\
Galaxy & 157.6274682004481 & -36.479748605745336 & Blue \\
GinGroup & 157.7403937388915 & -34.70791610390661 & Blue \\
Galaxy & 157.75073627506603 & -40.17847373737768 & Red \\
EmG & 157.8745742217274 & -32.71308133498473 & Blue \\
Seyfert 2 & 157.96713209517503 & -34.853619773852564 & Red \\
Galaxy & 157.98905881760817 & -41.81141143980291 & Blue \\
GinCl & 158.24673600744714 & -27.54358314016397 & Blue \\
Galaxy & 158.32248528920698 & -43.07862690646628 & Blue \\
GinGroup & 158.5031268315751 & -35.282679860025794 & Blue \\
GinGroup & 158.50313124725002 & -35.28257740972755 & Blue \\
GinCl & 158.61142286949843 & -27.50109937658653 & Blue \\
EmG & 158.66144673587488 & -28.583352366148148 & Blue \\
EmG & 158.72747432181612 & -20.548791249883188 & Blue \\
Galaxy & 158.74382525909093 & -40.91202846007671 & Blue \\
GinCl & 158.76197984814078 & -29.506601635883555 & Blue \\
EmG & 158.78217874356557 & -27.991310579496265 & Blue \\
EmG & 158.82799732327132 & -36.87848044358572 & Blue \\
GinCl & 158.84031456075968 & -27.695706429361035 & Blue \\
Galaxy & 158.88207558555908 & -30.83334548440517 & Blue \\
Galaxy & 158.89231455964594 & -44.57808865211405 & Blue \\
Galaxy & 159.01109904047328 & -24.906702733164057 & Blue \\
Galaxy & 159.02890003630893 & -28.295841627615104 & Blue \\
EmG & 159.09213938275613 & -25.37651306213532 & Blue \\
GinCl & 159.12642175053503 & -27.90110338436836 & Blue \\
GinCl & 159.1895107330749 & -28.167422472389195 & Blue \\
HII G & 159.22859031523527 & -26.240560203605337 & Blue \\
Galaxy & 159.23368459218906 & -26.903779869033656 & Blue \\
GinCl & 159.25764732775707 & -28.3671309867131 & Blue \\
Galaxy & 159.2685358307238 & -31.365924306516472 & Blue \\
EmG & 159.30363493989583 & -27.683956942836808 & Blue \\
GinCl & 159.33289005555525 & -28.238872359967427 & Blue \\
Galaxy & 159.3425230747454 & -27.544963483924352 & Blue \\
MIR & 159.47885157911082 & -24.42906249628565 & Red \\
IG & 159.52433508127856 & -25.09446001133432 & Blue \\
HII & 159.55988117082038 & -38.09040420988944 & Blue \\
GinCl & 159.6195197846552 & -28.51528044835931 & Blue \\
Galaxy & 159.62643334547482 & -23.54853332522397 & Blue \\
EmG & 159.63924688805704 & -27.737156832802217 & Blue \\
EmG & 159.67291882499617 & -25.592286212707226 & Blue \\
Galaxy & 159.73852013561282 & -20.04493226744504 & Blue \\
Galaxy & 159.80424882724313 & -20.636809607995904 & Blue \\
EmG & 159.8583643577597 & -23.75467363684896 & Blue \\
CataclyV* & 159.9998687263755 & -47.023971746909325 & Blue \\
CataclyV* & 159.9998687263755 & -47.023971746909325 & Blue \\
IG & 160.12922543109804 & -29.26958257985193 & Blue \\
EmG & 160.24459285878757 & -21.78454117893259 & Red \\
Galaxy & 160.2588369311913 & -30.794483697929085 & Blue \\
RRLyr & 160.26609841870317 & -34.18983231402896 & Blue \\
Seyfert 1 & 160.31322227492322 & -21.02303660410036 & Blue \\
Galaxy & 160.36546852844964 & -31.780316177810057 & Blue \\
Galaxy & 160.36546852844964 & -31.780316177810057 & Blue \\
Galaxy & 160.3964705972055 & -37.46929287562097 & Blue \\
Galaxy & 160.41429810549982 & -27.777270815267524 & Blue \\
IG & 160.52757582075228 & -22.105582546244552 & Blue \\
Galaxy & 160.581242806875 & -36.320460681862535 & Blue \\
Galaxy & 160.65827140741635 & -23.935679330514013 & Blue \\
EmG & 160.8791677599853 & -30.77221131800008 & Blue \\
EmG & 160.8791677599853 & -30.77221131800008 & Blue \\
Galaxy & 161.0404678812419 & -20.81930794850729 & Blue \\
IG & 161.25085749109647 & -22.15227748471596 & Blue \\
Galaxy & 161.394795164108 & -24.28370035395108 & Blue \\
EmG & 161.57129581271013 & -28.42321162775831 & Blue \\
Galaxy & 161.62610118824867 & -30.321619430632296 & Blue \\
EmG & 161.6602141109401 & -36.353314928386006 & Blue \\
Galaxy & 161.69807525519326 & -23.32773309331226 & Blue \\
Star & 161.84960610101908 & -41.99703765702019 & Blue \\
EmG & 161.9348517270216 & -20.96356099589758 & Blue \\
HII G & 161.9671272252362 & -20.08148354643084 & Blue \\
HII G & 161.96719252130313 & -20.081516627814896 & Blue \\
Radio & 162.09779631183395 & -25.16210701557664 & Red \\
Galaxy & 162.10543718248564 & -21.850134134896635 & Blue \\
SN & 162.10604102445896 & -25.16002800864644 & Red \\
Galaxy & 162.17634873432067 & -32.6437113818742 & Blue \\
Galaxy & 162.44532875160957 & -28.67697878087972 & Blue \\
Galaxy & 162.6651817194897 & -18.542879179509185 & Red \\
Galaxy & 162.75154034827315 & -20.23925755451401 & Blue \\
Galaxy & 162.75753741468907 & -28.33790302356877 & Blue \\
Galaxy & 162.86415841998232 & -19.6936241482243 & Blue \\
Galaxy & 162.95445605894844 & -21.88819830344904 & Blue \\
Galaxy & 163.13766385805908 & -23.149877815633783 & Blue \\
Candidate SN* & 163.6701579306735 & -39.22193302241251 & Blue \\
Galaxy & 163.84009494937516 & -23.424249462174128 & Blue \\
Galaxy & 163.8401072399262 & -23.42423690390457 & Blue \\
Galaxy & 164.15997638632015 & -20.78672855473099 & Blue \\
Galaxy & 164.20211131014338 & -19.833433033461127 & Blue \\
Galaxy & 164.26801410841605 & -33.15564025093645 & Red \\
Galaxy & 164.3079665625576 & -47.66979296726016 & Red \\
Galaxy & 164.40217291905444 & -35.5043965318927 & Blue \\
Galaxy & 164.68435964333 & -19.158632283113736 & Blue \\
CataclyV* & 164.7459588212315 & -31.60947413507693 & Blue \\
Galaxy & 164.79107984418525 & -27.9998147852096 & Blue \\
Candidate RRLyr & 165.46357641452303 & -46.88458955272901 & Blue \\
PM* & 165.49154204855074 & -23.790923666401383 & Blue \\
CataclyV* & 165.90237819943616 & -21.629405520763783 & Blue \\
GinPair & 165.9423233893419 & -23.245019645533223 & Blue \\
UV & 165.99608941173514 & -18.77670167044572 & Blue \\
IG & 166.8045693341223 & -19.81867010242588 & Blue \\
EmG & 166.8297173445012 & -19.55555857294575 & Blue \\
EmG & 166.8297504542546 & -19.55541255601515 & Blue \\
IG & 166.9493403358462 & -20.02226907187982 & Blue \\
Galaxy & 167.71070433927775 & -21.97451931840779 & Blue \\
Galaxy & 167.73846916883397 & -21.9481524929022 & Blue \\
Galaxy & 168.4623790067395 & -21.44856383511027 & Blue \\
EmG & 168.67501934507874 & -23.727739165310048 & Blue \\
QSO & 169.1815896684582 & -17.194855188675877 & Blue \\
Galaxy & 169.31254592914237 & -18.973439165695304 & - \\
Galaxy & 169.39609173910995 & -22.751735680185 & Red \\
EB* & 170.73371282684457 & -24.47777875977149 & Blue \\
Galaxy & 172.39240723207388 & -24.777518733865605 & Blue \\
AGN & 172.868588307185 & -19.98410951303733 & Blue \\
Candidate CV* & 173.0792138701351 & -21.661910741435744 & - \\
Galaxy & 173.60216658108814 & 1.1543503176286625 & Blue \\
Star & 174.05290786239414 & 0.0819136002772735 & Blue \\
EmG & 174.15304036864492 & 0.8172611304944561 & Blue \\
Galaxy & 174.15329276947446 & 0.8154915247975553 & Blue \\
CataclyV* & 174.34240673564196 & 1.816372111607291 & Blue \\
Candidate WD* & 174.45820819375535 & -20.1269725960127 & Blue \\
Galaxy & 174.726375893798 & -1.6428088681911317 & Blue \\
RRLyr & 174.73164786697532 & -21.19658104320207 & Blue \\
Galaxy & 174.75578476142175 & 1.3382478314723678 & Blue \\
QSO & 174.7681233226218 & -1.4402891840929772 & Blue \\
Galaxy & 175.3959908754587 & -18.19457670541773 & Blue \\
HII G & 175.44030773451672 & -1.901335843741868 & Blue \\
Galaxy & 175.53808352236646 & -18.16909360257072 & Red \\
PartofG & 175.55137074064442 & 0.3342699608891871 & Blue \\
Galaxy & 175.5609804603061 & -2.53146576861475 & Blue \\
RRLyr & 175.65814831106258 & -20.45605247119997 & Blue \\
Seyfert 1 & 175.712294430323 & 1.516178983683592 & Blue \\
QSO & 175.8722683259682 & -2.0554055350002094 & - \\
QSO & 175.8722971552406 & -2.0555300405054746 & - \\
Galaxy & 175.94213914323512 & -1.2761143494938 & - \\
Galaxy & 175.9475214240503 & 1.5149705815849348 & Blue \\
RRLyr & 176.03675655570706 & 1.4057473653603636 & Blue \\
EmG & 176.21230592350696 & 1.72354409641951 & Blue \\
Candidate WD* & 176.23234742536548 & -17.944273338719373 & Blue \\
Galaxy & 176.28348295379152 & -0.9883941397936244 & Blue \\
Galaxy & 176.29884651458195 & -0.90069812354485 & Red \\
Galaxy & 176.3135693424706 & -20.746519437466933 & Blue \\
Galaxy & 176.35960159169613 & 0.0041113183584209 & Blue \\
Galaxy & 176.50187966637736 & 0.1769481366071585 & Red \\
Galaxy & 176.53216927105808 & -0.4579777695914926 & Blue \\
QSO & 176.6796613966468 & 1.1885571745316508 & Blue \\
EmG & 176.79778025270713 & -0.4516064637217532 & Blue \\
Seyfert 1 & 177.07589066324908 & -1.6399572049281346 & Blue \\
Galaxy & 177.07645388246996 & -1.6418060467883486 & Blue \\
QSO & 177.41500197480892 & 1.773741489322032 & Blue \\
HII G & 177.59908479508587 & -0.5282934246500881 & Blue \\
GinCl & 177.65124196182416 & -0.5685118230038877 & Blue \\
Galaxy & 177.65163991073865 & -0.5673845606788012 & Blue \\
QSO & 177.70537423275803 & -0.8636364611502556 & Blue \\
Galaxy & 177.8042441388274 & -22.890342062408017 & Red \\
Galaxy & 177.87268901124386 & -0.0593301076444981 & Red \\
Galaxy & 177.88734562043996 & -2.372761176366136 & Blue \\
Seyfert 1 & 177.88896201644386 & -2.3727022231183543 & Blue \\
Galaxy & 178.07031354806253 & 1.3909661333218957 & Red \\
EmG & 178.07224961864256 & -2.8840749153908187 & Blue \\
GinGroup & 178.11507669366992 & -20.10390480837867 & Blue \\
Seyfert 1 & 178.15543681764203 & -2.4692788013971625 & Blue \\
HII G & 178.15697510920216 & -2.468564555740332 & Blue \\
Seyfert 1 & 178.19800706306424 & -0.668799198555378 & Blue \\
Seyfert 1 & 178.19803381207012 & -0.6688280067157996 & Blue \\
Galaxy & 178.30862939134568 & -3.409047236958316 & Blue \\
Galaxy & 178.36942473779808 & -3.230237358905925 & Blue \\
QSO & 178.4393258721732 & -2.722344158881375 & Blue \\
GinPair & 178.55129628495746 & 0.1367777756088964 & Blue \\
Galaxy & 178.7357899597813 & 0.1848611250308225 & - \\
Galaxy & 178.79661341326374 & 0.4848658484267525 & Blue \\
Galaxy & 178.79873915999408 & 0.4902688388336615 & Blue \\
Galaxy & 179.29940481773815 & -2.686921090064285 & Blue \\
Galaxy & 179.3012016296504 & -2.686482916082229 & Blue \\
Galaxy & 179.36682377284328 & -19.624040211696983 & Blue \\
Galaxy & 179.40451295413072 & -2.027027964558633 & Blue \\
Galaxy & 179.4045305314517 & -2.026996683135112 & Blue \\
QSO & 179.4500979791226 & 1.722464633451602 & Blue \\
QSO & 179.47609913312544 & -1.6377672684560611 & - \\
Galaxy & 179.48620454609028 & -20.565663973239875 & Blue \\
Galaxy & 179.53333868601857 & -17.893384995930617 & Blue \\
Galaxy & 179.59916114110445 & -19.517546281041724 & Blue \\
EmG & 179.74240913634958 & -19.02992833661004 & Blue \\
Galaxy & 179.8478669187984 & -1.7228572218896734 & Blue \\
SN & 179.87055152576207 & -19.25634139982329 & Blue \\
Galaxy & 180.08253579679277 & -20.802092442909533 & Blue \\
Galaxy & 180.08415872887417 & -1.1066153905837022 & Blue \\
QSO & 180.0906960738449 & -2.72525835885637 & Blue \\
PartofG & 180.1095757612321 & -1.1019334954271696 & Blue \\
QSO & 180.15952618608105 & 1.2129112212203663 & Blue \\
QSO & 180.1873094362587 & -18.995686347147792 & Blue \\
Galaxy & 180.1978076966742 & -3.420035887658516 & Blue \\
HI & 180.1988428326189 & -0.02340688109002 & Blue \\
GinPair & 180.29522824339665 & -1.2972657148586302 & Blue \\
QSO & 180.34692325828544 & 0.4745788784794196 & Blue \\
Galaxy & 180.3769799367477 & -23.31854795335849 & Blue \\
Cl* & 180.46003622778284 & -18.870100269365853 & Blue \\
HMXB & 180.46006145306487 & -18.87215768996868 & Blue \\
HII & 180.4603613485204 & -18.86737026502464 & Blue \\
Radio & 180.46306050492643 & -18.87467270267031 & Blue \\
Radio & 180.4635139848257 & -18.8625819554819 & Blue \\
Cl* & 180.46623334670275 & -18.874490272565943 & Blue \\
HII & 180.4678191428082 & -18.87206620305753 & Blue \\
HII & 180.47066639368427 & -18.86764989585912 & Red \\
HII & 180.4707539538797 & -18.869071106666347 & Blue \\
HII & 180.4729992541083 & -18.86226786400905 & Blue \\
Radio & 180.47322728801936 & -18.88582768843834 & Red \\
Cl* & 180.47726007541223 & -18.868760846298173 & Blue \\
PartofG & 180.4772638227824 & -18.88438817240201 & Blue \\
MolCld & 180.4806328363593 & -18.880180958610644 & Blue \\
HII & 180.48142237348915 & -18.873019641423856 & Blue \\
MolCld & 180.481983657978 & -18.870542013361973 & Blue \\
Radio & 180.482084992802 & -18.87855472820429 & Blue \\
HII & 180.4845311148264 & -18.87743596736424 & Blue \\
EB* & 180.52810919054875 & -23.051671253398336 & Blue \\
Galaxy & 180.70997875542824 & 0.3254128494396162 & Red \\
WD* & 181.31583811226784 & -2.706276551980267 & Blue \\
Galaxy & 181.6572228298745 & -15.28811460859046 & Blue \\
Galaxy & 181.71110394382373 & -14.21553394093172 & Blue \\
QSO & 181.7517098625927 & 1.1990112046487067 & Blue \\
QSO & 182.33562177580043 & -0.4820361729402547 & Blue \\
QSO & 182.545072752764 & -0.6527062334835679 & Blue \\
Galaxy & 182.6100536766531 & -0.0870043020990194 & Blue \\
Seyfert 1 & 182.68163468989783 & -0.6523719250391294 & Red \\
EmG & 182.7543746779469 & 1.3402916186010196 & Blue \\
Seyfert 1 & 183.0613723384793 & 0.0723961267201827 & Blue \\
Galaxy & 183.06624544494187 & -0.5647749352872001 & Blue \\
EB* & 183.24909669974548 & 1.823100765666048 & Red \\
Galaxy & 183.27052753312853 & -0.6503293355092394 & Red \\
Galaxy & 183.41164031698213 & -1.2934275172309344 & Blue \\
Galaxy & 183.4506660746881 & -14.52772420569266 & Blue \\
QSO & 183.6469096827458 & -1.990114618560348 & Blue \\
QSO & 183.81343790615867 & -1.5946745490378629 & Blue \\
Galaxy & 183.91446236872125 & -2.363112482714665 & Blue \\
Galaxy & 184.0314339047394 & -2.4326620218980715 & Blue \\
Galaxy & 184.5002257605925 & 0.4326907734274652 & - \\
Galaxy & 184.52944788519463 & -3.1080109031254355 & Red \\
Galaxy & 184.5794111776165 & -14.20553320828172 & Blue \\
Galaxy & 184.579419031614 & -14.205547557474516 & Blue \\
QSO & 184.73251321873693 & 2.0005938638657152 & Blue \\
QSO & 184.9269721119905 & -0.3059539785935566 & Blue \\
HII G & 184.97137095309444 & 1.77334461639504 & Blue \\
Galaxy & 185.0155358422371 & 1.1089971952641655 & Red \\
LSB G & 185.04802604861416 & 1.9586348136308644 & Blue \\
Galaxy & 185.11998162334305 & -1.8391750369277204 & Blue \\
Galaxy & 185.12660572308948 & -0.4508268628498345 & - \\
QSO & 185.37906214223312 & 1.124474721489336 & Blue \\
Star & 185.39336391937871 & -14.964030135530916 & Blue \\
Galaxy & 185.48261972911092 & -1.593344147053986 & Blue \\
Star & 185.66389783419783 & -15.486697903533098 & Blue \\
Galaxy & 185.84330648828225 & -0.1338075560321897 & Blue \\
Galaxy & 186.051943489446 & 0.5669364629276971 & Blue \\
QSO & 186.08805453702857 & 0.3984481006440221 & Blue \\
AGN & 186.4280748118269 & 0.5727372676461775 & Blue \\
Galaxy & 186.4474227096049 & -1.3349274091067191 & Blue \\
HII G & 186.5943422902263 & -1.2547941267807978 & Blue \\
HII G & 186.5946956629417 & -1.2534088007482242 & Blue \\
QSO & 186.6069592196776 & 1.267941658397558 & Blue \\
Seyfert 1 & 186.67269731032096 & -0.3347511966887692 & Blue \\
GinPair & 186.7689240600145 & -0.9059699735306485 & Blue \\
GinPair & 186.7690053713781 & -0.9061085611109012 & Blue \\
QSO & 186.77970923418985 & 1.1364780564297168 & Blue \\
Cl* & 186.9419456708534 & 1.6004126119631985 & Blue \\
Galaxy & 187.0663127889704 & 1.828815876976476 & Blue \\
EmG & 187.21391718345924 & -2.441436093650128 & Blue \\
Galaxy & 187.311049563872 & -1.365324370079482 & Blue \\
LSB G & 187.44305324875543 & -1.2950038406448354 & Blue \\
RadioG & 187.4953452683044 & 0.0272052015597826 & Red \\
Galaxy & 187.72629290525728 & 0.9640254113499284 & Blue \\
PartofG & 187.95005920551247 & -2.970279786813015 & Blue \\
EmG & 188.01126587842555 & 0.5235179491769296 & Blue \\
HII G & 188.0984798952765 & -1.7400910842643869 & Blue \\
Blue & 188.1507145595841 & -3.310957494367809 & Blue \\
Star & 188.17323029456747 & 0.0573372798919609 & Blue \\
RRLyr & 189.26006228209985 & -15.278740375443848 & Blue \\
Star & 189.83103561160047 & -14.791965932344512 & - \\
Galaxy & 191.1121599071608 & -12.876636746547932 & Blue \\
Galaxy & 192.69598912587915 & -14.483746732287928 & Blue \\
Galaxy & 194.7464380329297 & -13.86174217154326 & Blue \\
Candidate CV* & 194.75339729526405 & -13.57832034488012 & Blue \\
Galaxy & 194.76947807533804 & -14.773193673700838 & Blue \\
Galaxy & 194.88643900048527 & -15.23870530424191 & Blue \\
Galaxy & 195.01265981525043 & -15.204780538418568 & Blue \\
GinPair & 195.1637302491048 & -14.666465631319904 & Blue \\
Galaxy & 195.27952833886835 & -13.517323249547193 & Blue \\
QSO & 195.68164279128675 & -13.931326571318332 & Blue \\
EmObj & 195.88935067936143 & -14.323090018355355 & Blue \\
Galaxy & 196.2068792963303 & -13.191177561311532 & Blue \\
Galaxy & 196.2182957489161 & -12.37186182183362 & Blue \\
Galaxy & 196.49382904825976 & -12.669019185775264 & Blue \\
Galaxy & 197.5365518439323 & -12.20565697288585 & Blue \\
EmG & 197.9928613567656 & -12.064271847970153 & Blue \\
Galaxy & 198.1183640318735 & -10.5901002725042 & Blue \\
EmObj & 198.40060156082 & -12.470900254658584 & Blue \\
Galaxy & 198.7832210949353 & -12.518085580918816 & Blue \\
Galaxy & 199.4203839129064 & -10.183193478269848 & Blue \\
Galaxy & 199.42655914037883 & -0.3376844496286302 & - \\
AGN & 199.4331632480484 & -1.000304791159635 & Blue \\
Galaxy & 199.8428608076528 & -15.156551152235588 & Blue \\
GinGroup & 199.9279871875005 & -11.474580164900791 & Blue \\
Star & 199.9899853471545 & -0.5796217824328943 & Blue \\
QSO & 199.99667654428632 & -12.487991060254044 & Blue \\
Galaxy & 200.0847263890424 & -12.571681196376176 & Blue \\
QSO & 200.0977807916915 & -0.7918937289501875 & - \\
Galaxy & 200.39451837704632 & -15.18209388051804 & Blue \\
Galaxy & 200.40763851841496 & -14.855478567286989 & Blue \\
EmG & 200.5712834913193 & -0.5484394981488907 & - \\
RRLyr & 200.93371811191045 & -12.053272612599368 & Blue \\
BlueCompG & 201.45279121661528 & -11.610508041620404 & Blue \\
BlueCompG & 201.4528190508685 & -11.610542402646049 & Blue \\
Galaxy & 201.5212567425004 & -9.370169275455012 & Blue \\
RRLyr & 202.99778408537745 & -9.884062720187078 & Blue \\
Radio & 204.0776714159391 & -7.3810632714249085 & Blue \\
Galaxy & 204.1377042494816 & -6.47921900688106 & Red \\
Blue & 204.7882876783339 & -8.327997709939716 & Blue \\
Galaxy & 205.2707900815086 & -7.018268794365892 & Blue \\
RRLyr & 205.8792268775973 & -15.316371740752832 & Blue \\
CataclyV* & 205.91016519674625 & -8.234371065828142 & Blue \\
SN & 206.66333089551333 & -9.643338332630568 & Blue \\
Blue & 206.95757353263463 & -4.169611778952146 & Blue \\
Galaxy & 207.4258304698629 & -2.199743232705236 & Blue \\
Blue & 207.63885638270955 & -12.278578148358983 & Blue \\
Galaxy & 207.8487547626973 & -6.069918761179799 & Blue \\
Galaxy & 207.89854491435565 & -2.554155873440916 & Blue \\
Galaxy & 208.0162602836453 & -2.122849268399833 & Blue \\
Galaxy & 208.0176716348379 & -2.1302616063806163 & Blue \\
Galaxy & 208.5470343354216 & -3.4408967610650127 & Blue \\
Galaxy & 208.5475170682714 & -3.4408463277549943 & Blue \\
QSO & 208.69386729420503 & -10.684057890909807 & Blue \\
GinPair & 208.89159973272297 & -5.971399339776448 & Blue \\
Galaxy & 208.907000978763 & -4.195508295444852 & Blue \\
Galaxy & 208.93941167819 & -6.0044269690496535 & Blue \\
Galaxy & 208.94449637980523 & -6.011323983078441 & Blue \\
QSO & 209.0116324685152 & -2.4398085262225853 & - \\
Galaxy & 209.2228497784912 & -2.647844687664313 & Blue \\
Galaxy & 209.5359168247792 & -4.145441774787235 & Blue \\
Galaxy & 209.67286668662445 & -1.5210215591711298 & Red \\
Galaxy & 210.6878474329628 & -7.373825503885681 & Blue \\
QSO & 216.7245347178442 & 5.421495966728137 & Blue \\
Galaxy & 216.795092182775 & 5.133141307382639 & Blue \\
Possible lensImage & 217.2307884592734 & 5.0060874486010265 & Blue \\
Blue & 217.2310767118411 & 5.0055279363461525 & Blue \\
GinCl & 217.430980101215 & 5.356359632037527 & Red \\
BClG & 217.4943714069119 & 4.769808218972206 & Red \\
Galaxy & 220.3668212426573 & 5.864515543127945 & Blue \\
QSO & 223.43549771425327 & 4.946107899138475 & Blue \\
AGN & 223.89042173389376 & 4.778676422249849 & Red \\
QSO & 300.4321955778412 & 0.8217833390436116 & Blue \\
low-mass* & 301.1349485385638 & 0.1781570740733244 & - \\
RRLyr & 302.7008333786705 & -0.2177414034676869 & Blue \\
RRLyr & 302.7008333786705 & -0.2177414034676869 & Blue \\
RRLyr & 305.64797588529285 & -0.6694194677482896 & Blue \\
RRLyr & 305.6574820578185 & -0.0473696356750876 & Blue \\
Galaxy & 307.05008700522615 & 0.288400031713299 & Blue \\
QSO & 307.2783691077352 & 0.9148870344062292 & Blue \\
Seyfert 1 & 310.91682422695794 & 0.481551561964735 & Blue \\
QSO & 311.6087878150387 & 0.3938280343112397 & Red \\
CataclyV* & 311.8364843167464 & 0.0021704620321628 & Blue \\
CataclyV* & 311.8365168802931 & 0.0021268467006492 & Blue \\
Seyfert 1 & 312.2956269426095 & 0.265970422493022 & Blue \\
QSO & 312.4859068430339 & -0.2004724242450093 & Blue \\
QSO & 312.48592414481044 & -0.200477643373699 & Blue \\
QSO & 313.3198559349733 & 0.9892076810353208 & - \\
QSO & 313.466816984702 & -0.2670765601766264 & Blue \\
Star & 314.0606317101171 & -0.680733084738148 & - \\
QSO & 314.302863294349 & 0.2031735829473498 & Blue \\
QSO & 314.4198074848838 & 0.9052857685742688 & Red \\
Candidate WD* & 314.5268677152485 & -30.1383582004064 & Blue \\
Galaxy & 314.6022450825876 & -32.722916673933284 & Blue \\
Galaxy & 314.70702257259086 & -44.34018284222228 & Blue \\
Galaxy & 314.9896989117314 & -21.65968492556584 & Blue \\
CataclyV* & 315.05879832773167 & 0.7460919799647099 & Blue \\
Galaxy & 315.48312134293826 & -0.5235874873138652 & Red \\
Galaxy & 315.48512664422447 & -39.394516374889456 & Blue \\
QSO & 315.6737967478414 & -32.87890999401036 & Blue \\
QSO & 315.67381555407934 & -32.87898718209263 & Blue \\
Galaxy & 315.7592781700893 & -45.24473926016992 & Blue \\
Star & 315.98609213331895 & -21.790860533436685 & Blue \\
IG & 316.03549398691706 & -43.53423041944633 & Blue \\
GinGroup & 316.04653712211586 & -43.592725469851146 & Blue \\
Galaxy & 316.08943735381246 & -30.197214902706765 & Red \\
EmG & 316.2304449432887 & -0.5893867983932235 & Blue \\
Galaxy & 316.3361978919587 & -45.98869609290722 & Blue \\
Galaxy & 316.41115448294687 & -42.7812242143475 & Blue \\
PN & 316.47319569385303 & -37.14456181858315 & Blue \\
Star & 316.7000786165858 & -40.334348615950496 & Blue \\
GinPair & 316.79991779423176 & -47.55699311008755 & Blue \\
GinPair & 316.80774783068466 & -47.55702616044946 & Blue \\
EmG & 318.00385158352736 & -0.280345841969967 & Blue \\
AGN & 318.10247638771125 & -41.48147711626288 & Blue \\
EB* & 318.3673788604048 & 0.0590671476084678 & Blue \\
Galaxy & 318.8297135149132 & -33.22628044991965 & Blue \\
CataclyV* & 319.517818265895 & -34.22874036730848 & Blue \\
Radio(cm) & 320.7617323963711 & -29.251115062981185 & Blue \\
Candidate CV* & 321.72725697504 & -1.348366630745761 & Blue \\
Galaxy & 321.7844092060996 & -30.9523274970352 & Blue \\
EmG & 322.4759106821289 & -1.0564795011399617 & Blue \\
QSO & 322.8730471709395 & -45.697352338099286 & Blue \\
Galaxy & 323.1762194624441 & -1.0525538970193808 & Blue \\
Seyfert 1 & 323.18856454382785 & 0.0296602067769526 & Blue \\
QSO & 323.408936906668 & 1.4413770493499043 & Blue \\
QSO & 323.72952597635833 & 0.1824349898807966 & Blue \\
QSO & 324.20730055187585 & -1.481159088195661 & Blue \\
QSO & 324.53121345495185 & -45.13834354299933 & - \\
Seyfert 1 & 324.57901744173665 & 1.2062510372667887 & Blue \\
Candidate CV* & 324.9065914699761 & -2.6536034951853305 & Blue \\
QSO & 325.2768379376971 & 0.7926481060166569 & Blue \\
Galaxy & 325.4280346483323 & 0.7597010317264318 & Blue \\
QSO & 325.47932878054104 & -1.292825833976646 & Blue \\
QSO & 325.479329853818 & -1.292849791540993 & Blue \\
SN & 326.0956022427804 & -29.91639450734865 & Red \\
EmG & 326.2329872209474 & 0.3850203444962297 & Blue \\
Galaxy & 326.4167001057865 & -29.32690805408077 & Red \\
EmG & 327.1275218035224 & -0.7979329669827752 & Blue \\
EmG & 327.127535195112 & -0.797904422039106 & - \\
QSO & 327.5112425223878 & 1.2288440704579693 & Blue \\
QSO & 327.54385873207247 & -0.1668265647596742 & Blue \\
Galaxy & 328.2731418249827 & -31.471640527742185 & Blue \\
Galaxy & 328.2731418249827 & -31.471640527742185 & Blue \\
Galaxy & 328.2731418249827 & -31.471640527742185 & Blue \\
Galaxy & 328.2731418249827 & -31.471640527742185 & Blue \\
Galaxy & 328.57500430647343 & 0.9422073085758816 & Blue \\
Galaxy & 329.0576398034554 & -1.1618941444509447 & Blue \\
Galaxy & 329.0576982984844 & -1.1619911379394898 & Blue \\
Galaxy & 329.08245077543467 & -1.1676749952103018 & Blue \\
Galaxy & 329.0825336876782 & -1.167693306004015 & Blue \\
Galaxy & 329.3370362394045 & -25.133993290826115 & Blue \\
HII G & 329.6011751965016 & -0.737145415174484 & Blue \\
Galaxy & 329.76204963872004 & -0.5550013061974669 & Blue \\
Galaxy & 329.7629603765794 & -1.9550737068589068 & Blue \\
Galaxy & 330.4586482938667 & -0.7074127748175637 & Blue \\
Galaxy & 330.5294901692158 & -26.44390115708122 & Blue \\
Star & 330.8130711615091 & 1.2891533854468642 & Blue \\
Galaxy & 331.22366426801165 & -25.051456106463764 & Blue \\
QSO & 331.37225331383246 & -0.5196269942518518 & Blue \\
IG & 331.7304425939274 & -31.052944378400603 & Red \\
Galaxy & 331.89396381531327 & -28.65815178172343 & Blue \\
GinGroup & 332.1018041812308 & -29.05100572704748 & Blue \\
Galaxy & 332.21648908406854 & -30.64966912723208 & - \\
QSO & 332.2165318654766 & -1.1010232176217527 & Blue \\
QSO & 332.21654479357403 & -1.1010212789856892 & Blue \\
GinPair & 332.2329099311068 & -27.22278199790022 & Blue \\
QSO & 332.3293752575985 & -24.12012102380704 & Red \\
Galaxy & 332.34547450375544 & -25.417946056108004 & Blue \\
WD* & 332.45263484436884 & -30.23217545253064 & Blue \\
Galaxy & 332.4639565862639 & 1.1500007990678438 & Red \\
QSO & 332.4773666618215 & -1.454892271729591 & Blue \\
EmG & 332.50311376214466 & -31.23332648999496 & Blue \\
Galaxy & 332.51558209616104 & -25.335365131244853 & Blue \\
Galaxy & 332.5240068270393 & -27.91075708397664 & Blue \\
Galaxy & 332.7416878016213 & -25.07531023085714 & Blue \\
Galaxy & 332.7430386884259 & -27.658157768445044 & Blue \\
SN & 333.17320112679323 & 0.5119642601996512 & Blue \\
QSO & 333.3985380422876 & -28.42824210005132 & Blue \\
Galaxy & 333.51191857192214 & -27.53928600571248 & Blue \\
Galaxy & 333.5119857829216 & -27.53930260557384 & Blue \\
EmG & 333.5134692326156 & -29.38272127795172 & Blue \\
Galaxy & 333.5212380480937 & -29.381350935720462 & Blue \\
EmG & 333.600966286032 & -29.98098137914205 & Blue \\
Galaxy & 333.6746944454729 & -28.44433992901234 & Blue \\
Galaxy & 333.69650411507115 & -29.68677871511806 & Blue \\
EmG & 333.8212699349557 & -28.89929996827705 & Blue \\
QSO & 333.88574209843875 & -28.301075647502863 & Blue \\
EmG & 334.1253072836468 & -29.01480498338059 & Red \\
Galaxy & 334.2771669200802 & -30.579476065999444 & Red \\
QSO & 334.3435194955581 & 1.0767543256819685 & Blue \\
Galaxy & 334.421449319186 & -27.365163382540192 & Blue \\
Galaxy & 334.5579398251045 & 0.2736951228444171 & - \\
Galaxy & 334.56262843755 & 1.254694273842988 & Blue \\
AGN & 334.57190645739905 & 0.6065838369084469 & Red \\
Seyfert 1 & 334.58077291303573 & -27.26228422509564 & Blue \\
Galaxy & 334.6947482712885 & -1.188558994516001 & Blue \\
Galaxy & 334.69475415064494 & -1.1885577406117744 & Blue \\
Galaxy & 334.71939499895126 & -1.0529142510236456 & Blue \\
EmG & 334.8579955716732 & -30.852129377014343 & Blue \\
Galaxy & 334.9364411418472 & -0.2444533826638444 & Red \\
EmG & 334.93782729725075 & -29.57039409080591 & Blue \\
Galaxy & 334.9744213770292 & 0.4846689985595231 & Blue \\
Star & 335.08900553863754 & 0.6779085075975249 & Blue \\
Seyfert 1 & 335.3067522132758 & -28.07246239885318 & Red \\
Galaxy & 335.72883636182854 & -30.70788858469623 & Blue \\
Galaxy & 335.8070780085201 & -28.979058837654552 & Blue \\
QSO & 335.88681821013813 & -1.10412204251795 & Blue \\
EmG & 335.90012802575814 & -28.527653500762096 & Blue \\
QSO & 336.01397097341777 & -0.9567096099850134 & Blue \\
QSO & 336.01401051604205 & -0.9566969699045356 & Blue \\
Candidate CV* & 336.0677440421202 & -29.40602881395364 & Blue \\
Galaxy & 336.912099323913 & -31.136207086254505 & Blue \\
Galaxy & 337.1046549602956 & -0.3714489675823685 & Red \\
Galaxy & 337.19918624431745 & -30.91199843220645 & Blue \\
Galaxy & 337.1992648299098 & -30.911989220539105 & Blue \\
Galaxy & 337.22360618425245 & -30.980939741660382 & Blue \\
Galaxy & 337.22360894565986 & -30.98095563687505 & Blue \\
QSO & 337.3458366571309 & -2.011785900315515 & - \\
QSO & 337.48557309949206 & 0.5240235390794395 & Blue \\
Galaxy & 337.5076760184757 & -29.597960152831284 & Blue \\
Galaxy & 337.50768925239095 & -29.59794652431351 & Blue \\
BlueCompG & 337.653463904362 & -0.1099571744851075 & Blue \\
Galaxy & 337.7750079939556 & -0.1955294013694825 & Blue \\
Star & 337.80812689962085 & -31.33455219401752 & Blue \\
QSO & 338.2155980841278 & -30.54711277145381 & Blue \\
Galaxy & 338.4273343986453 & -30.32651312377564 & Blue \\
Star & 338.5277783244117 & 0.0224447956398556 & Red \\
CataclyV* & 338.66637521384126 & 0.6909664418071105 & Blue \\
Galaxy & 338.73567173096313 & -31.145574300500385 & Blue \\
Galaxy & 338.73569433672816 & -31.1455850268799 & Blue \\
Seyfert 2 & 338.78506511786827 & -0.8998369935534453 & Red \\
Seyfert 2 & 338.785109135079 & -0.8998316294838345 & Red \\
EmG & 338.88432907328246 & -29.77583940043047 & Blue \\
Galaxy & 338.92933238127915 & -0.9101766325827358 & Blue \\
Galaxy & 338.92938653521645 & -0.9101558150095174 & Blue \\
Galaxy & 338.9294202112855 & -0.9100982563435506 & Blue \\
low-mass* & 338.9330431386031 & -0.6589082844776932 & Red \\
HII & 338.9369542908973 & -26.040945424398046 & Blue \\
HII & 338.9391682825249 & -26.03634410855905 & Blue \\
HII & 338.942864761339 & -26.074740893365583 & Blue \\
HII & 338.9509617589531 & -26.02333197997694 & Blue \\
QSO & 339.139729783869 & 0.4480037094611175 & Blue \\
QSO & 339.20666171031564 & 0.9038310163367896 & - \\
Galaxy & 339.3495140772359 & -1.0219920514291434 & Blue \\
HB* & 339.3744168628378 & -1.09702098520342 & - \\
QSO & 339.5968891921001 & -0.952268013442764 & Blue \\
QSO & 339.5969264365535 & -0.9522370681869864 & Blue \\
QSO & 339.6845594683822 & -0.948694319373839 & Blue \\
Galaxy & 340.3312015654902 & -39.97314857397814 & Blue \\
Galaxy & 340.4562435334129 & -30.329015431975083 & Blue \\
Galaxy & 340.81825118013086 & -39.86108916393093 & Blue \\
Galaxy & 340.8308173956169 & -39.87900062583571 & Blue \\
Galaxy & 340.854608426132 & -39.92223577354626 & Red \\
Galaxy & 340.96755531580084 & -0.3832981583866523 & Blue \\
QSO & 341.3799824113044 & -0.7526108924743485 & Blue \\
QSO & 341.3799836615205 & -0.7525947660724432 & Blue \\
QSO & 341.4164203070588 & -0.4054482118161328 & Blue \\
QSO & 342.4836663184702 & 0.0384316128868782 & Blue \\
Galaxy & 342.55381866523845 & -0.666369376718931 & Blue \\
QSO & 342.95730339479854 & -0.4698259059306293 & Blue \\
EmG & 342.9879147578966 & -29.414118119813367 & Blue \\
Star & 343.2393387782164 & 0.4587823345606925 & Blue \\
Seyfert 1 & 343.4706720102698 & -30.162151096626257 & Blue \\
QSO & 343.54646400880233 & -31.45314972730561 & Blue \\
QSO & 343.5498267824822 & -0.8303920878390473 & Blue \\
QSO & 343.5498379046172 & -0.8303498619565828 & Blue \\
EmG & 343.76622684097015 & -30.32036536315227 & Blue \\
Star & 344.7838160610223 & -31.454652481082007 & Blue \\
Galaxy & 345.12539458156243 & -0.5016202410575213 & Blue \\
QSO & 345.5049924642404 & 0.5131468261640252 & Blue \\
QSO & 345.6476494287596 & -28.941580257566937 & Blue \\
QSO & 345.8183544369748 & -0.2031687049072709 & Blue \\
CataclyV* & 345.9651406629765 & 1.11426658072561 & Blue \\
QSO & 346.1180884490532 & 0.950341888510796 & Blue \\
QSO & 346.1839995707146 & -1.047637756170335 & Blue \\
Galaxy & 347.04492000641267 & -1.299588506149179 & Blue \\
QSO & 347.231188850614 & 0.6182549476788003 & Blue \\
GinGroup & 347.2316580483845 & -30.85783430936128 & Blue \\
QSO & 347.3096188733344 & -30.986794418426523 & Blue \\
Seyfert 1 & 347.44230168607083 & 0.0136411947639146 & Blue \\
Seyfert 1 & 347.4423475040321 & 0.0136153759642055 & Blue \\
EmG & 347.67494008100954 & -1.1633342696409468 & Blue \\
QSO & 347.8963319528839 & -31.445592377850765 & Blue \\
Galaxy & 348.0373247446722 & -31.070356404741503 & Red \\
QSO & 348.13067741527794 & -1.1937043288787734 & Blue \\
QSO & 348.2461021601764 & 1.1349593899913708 & Blue \\
QSO & 348.299611065616 & -0.7605497687369258 & Blue \\
HII G & 348.46608265183744 & -1.1752128153912331 & Blue \\
LSB G & 348.7105150733019 & 1.390757174203209 & Blue \\
QSO & 348.8308033561476 & -30.64921915522369 & Blue \\
CataclyV* & 348.88240046675367 & -30.813531857016685 & Blue \\
QSO & 349.2168428172675 & 0.8571987065141106 & Blue \\
Seyfert 1 & 349.42752127421284 & 0.093129280239253 & Blue \\
QSO & 349.92814830462225 & -30.44152729995124 & Blue \\
Galaxy & 350.1467244968801 & -0.8808002402296434 & Blue \\
Galaxy & 350.1467502457788 & -0.8807767191983554 & Blue \\
Galaxy & 350.3604588557264 & -31.12485243647311 & Blue \\
RRLyr & 350.88051361237183 & 1.134997376685103 & Blue \\
HII & 351.0847562454346 & -0.1069557818481737 & Blue \\
HII G & 351.08902800249376 & -0.1081688053941925 & Blue \\
QSO & 351.2406451424701 & 0.3648363114586009 & Blue \\
Galaxy & 351.3517724774783 & 0.7700617331859075 & Blue \\
CataclyV* & 351.46448670155314 & -1.6732732825345382 & Blue \\
Seyfert 1 & 351.4812848799453 & -0.619649969309404 & - \\
BlueCompG & 351.9320769678968 & -2.0154836384737007 & Blue \\
Galaxy & 351.9349111408217 & -2.0130122785984863 & Blue \\
HII G & 352.05124626986577 & -1.0624481917322088 & Blue \\
CataclyV* & 352.25183229093045 & -29.77943600230389 & Blue \\
QSO & 352.768315319454 & -0.7102948577467144 & Red \\
Galaxy & 352.99905235631786 & -0.8051318589747921 & Blue \\
EmG & 353.22826977772627 & -30.978836100206514 & Blue \\
BClG & 353.2361807443991 & 1.1897411683153614 & - \\
QSO & 353.25093145251367 & -0.3418033383059611 & Blue \\
QSO & 353.660612982189 & 0.3949855898346673 & Blue \\
Galaxy & 353.8374272193407 & 1.1742822684735112 & Red \\
QSO & 353.8445277828368 & -0.1097797881436761 & Blue \\
LSB G & 354.19570227897805 & 0.6232818326354911 & Blue \\
QSO & 354.3417350883094 & 0.3775467635964666 & Blue \\
AGN & 354.38244578497734 & 0.4333010734691366 & Blue \\
GinCl & 354.4482155147033 & 0.295228446902126 & - \\
Galaxy & 355.16011198023494 & -0.8918305906086714 & Blue \\
QSO & 355.8714951658231 & -30.03336765255501 & Blue \\
CataclyV* & 356.1688766402904 & -0.2016823731104521 & Blue \\
Galaxy & 357.0999479884554 & -1.7919613357037034 & Blue \\
Galaxy & 357.50647665269133 & -30.18530531379982 & Blue \\
Galaxy & 357.8152522370048 & -1.0744996159314814e-05 & Blue \\
Galaxy & 357.8153294830434 & -5.8998646198314144e-05 & Blue \\
QSO & 358.9422665997112 & -0.3952102105534735 & Blue \\
QSO & 359.32653511559846 & 0.7307049350125405 & Red \\
QSO & 359.5218549462989 & -1.3649719892078715 & Blue \\
%% \end{tabular}
%% \end{table}

\end{longtable} 
\end{center}
%%%%%%%%%%%%%%%%%%%%%%%%%%%%%%%%%%%%%%%%%%%%%%%%%%


% Don't change these lines
\bsp	% typesetting comment
\label{lastpage}
\end{document}

% End of mnras_template.tex
