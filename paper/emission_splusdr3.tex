% mnras_template.tex 
%
% LaTeX template for creating an MNRAS paper
%
% v3.0 released 14 May 2015
% (version numbers match those of mnras.cls)
%
% Authors:
% Keith T. Smith (Royal Astronomical Society)

% Change log
%
% v3.0 May 2015
%    Renamed to match the new package name
%    Version number matches mnras.cls
%    A few minor tweaks to wording
% v1.0 September 2013
%    Beta testing only - never publicly released
%    First version: a simple (ish) template for creating an MNRAS paper

%%%%%%%%%%%%%%%%%%%%%%%%%%%%%%%%%%%%%%%%%%%%%%%%%%
% Basic setup. Most papers should leave these options alone.
\documentclass[fleqn,usenatbib]{mnras}

% MNRAS is set in Times font. If you don't have this installed (most LaTeX
% installations will be fine) or prefer the old Computer Modern fonts, comment
% out the following line
\usepackage{newtxtext,newtxmath}
% Depending on your LaTeX fonts installation, you might get better results with one of these:
%\usepackage{mathptmx}
%\usepackage{txfonts}
\usepackage[T1]{fontenc}

% Allow "Thomas van Noord" and "Simon de Laguarde" and alike to be sorted by"N" and "L" etc. in the bibliography.
% Write the name in the bibliography as "\VAN{Noord}{Van}{van} Noord, Thomas"
\DeclareRobustCommand{\VAN}[3]{#2}
\let\VANthebibliography\thebibliography
\def\thebibliography{\DeclareRobustCommand{\VAN}[3]{##3}\VANthebibliography}

\usepackage{graphicx}	% Including figure files
\usepackage{amsmath}	% Advanced maths commands
%\usepackage{amssymb}	% Extra maths symbols
\usepackage{longtable}
\usepackage{ulem}

\newcommand{\rlopes}[1]{\textcolor{blue}{#1}}
\newcommand{\cut}[1]{\textcolor{red}{#1}}
\newcommand{\comment}[1]{\textcolor{red}{#1}}
\newcommand{\luis}[1]{\textcolor{magenta}{#1}}

%%%%%%%%%%%%%%%%%%%%%%%%%%%%%%%%%%%%%%%%%%%%%%%%%%

%%%%% AUTHORS - PLACE YOUR OWN COMMANDS HERE %%%%%

% Please keep new commands to a minimum, and use \newcommand not \def to avoid
% overwriting existing commands. Example:
%\newcommand{\pcm}{\,cm$^{-2}$}	% per cm-squared

%%%%%%%%%%%%%%%%%%%%%%%%%%%%%%%%%%%%%%%%%%%%%%%%%%

%%%%%%%%%%%%%%%%%%% TITLE PAGE %%%%%%%%%%%%%%%%%%%

% Title of the paper, and the short title which is used in the headers.
% Keep the title short and informative.
\title[S-PLUS: H$\alpha$ emitters]{H$\alpha$ emitters from the Southern Photometric Local Universe Survey (S-PLUS)}

\author[Guti\'{e}rrez-Soto et al.]{
L. A. Guti\'{e}rrez-Soto,$^{1}$\thanks{E-mail: gsoto.angel@gmail.com}
R. Lopes de Oliveira$^{1,2,3}$
\\
% List of institutions
$^{1}$Departamento de Astronomia, IAG, Universidade de S\~{a}o Paulo, Rua do Mat\~{a}o,
1226, 05509-900, S\~{a}o Paulo, Brazil\\
$^{2}$Departamento de F\'isica, Universidade Federal de Sergipe, Av. Marechal Rondon, S/N, 49100-000, S\~ao Crist\'ov\~ao, SE, Brazil\\
$^{3}$Observat\'orio Nacional, Rua Gal. Jos\'e Cristino 77, 20921-400, Rio~de~Janeiro, RJ, Brazil\\
}

% These dates will be filled out by the publisher
\date{Accepted XXX. Received YYY; in original form ZZZ}

% Enter the current year, for the copyright statements etc.
\pubyear{2021}

% Don't change these lines
\begin{document}
\label{firstpage}
\pagerange{\pageref{firstpage}--\pageref{lastpage}}
\maketitle

% Abstract of the paper
\begin{abstract}

In the way to map 9000 deg$^2$ of the Southern hemisphere, the S-PLUS project is also surveying the sky in a proxy of a myriad of astrophysical processes: the H$\alpha$ transition. Here we explore such a capability from its DR3 to make H$\alpha$ emitters in evidence from 
the ($r$ - $J$0660) versus ($r$ - $i$) color-color diagram and distinguish the red from the blue ones by exploring the ($r$ - $i$) and ($g$ - $z$) diagram. Our catalog is composed of 9,200 objects that exhibit excess in the narrow $J$0660 band which is consistent with H{$\alpha$} in emission. Unsupervised, clustering machine learning approach revealed two distinct populations: one with an intense blue continuum and another with a red one. The hierarchical clustering algorithm was compared with the HDBSCAN. By adopting a ``soft'' clustering approach, we assigned the probability of each emitter belonging to a given population, blue or red clusters. We use synthetic and observed (SDSS) spectra to emphasize the potential of color-color diagrams to distinguish several classes of emission line emitters that include planetary nebulae, H II regions, young stellar objects, symbiotic stellar systems, cataclysmic variables, blue compact galaxies, star-forming galaxies, and quasars, and trace the way to reveal new ones with S-PLUS data.

\end{abstract}
% Select between one and six entries from the list of approved keywords.
% Don't make up new ones.
\begin{keywords}
  surveys -- stars: emission-line, Be -- novae, cataclysmic variables
  -- galaxies: dwarf -- quasars: emission lines
\end{keywords}

%%%%%%%%%%%%%%%%%%%%%%%%%%%%%%%%%%%%%%%%%%%%%%%%%%

%%%%%%%%%%%%%%%%% BODY OF PAPER %%%%%%%%%%%%%%%%%%

\section{Introduction}

Atomic excitation followed by recombination in Balmer hydrogen emission lines may be ignited in many different ways, from thermal and non-thermal collisional excitation in shock-heated gas, and from energetic photons acting over a diffuse gas. As a practical result, and the Universe being Hydrogen abundant, the observation of those electronic transitions offer an important window into the study of astrophysical objects. Among them, the Balmer lines represent extremely useful tools in Astronomy and in particular the red H$\alpha$ line (which has rest-frame wavelength of $@$6564.614\,\AA\ at vacuum), that corresponds to the electron transition from the $ n $ = 3 to the $ n $ = 2 energy level, is the most widely used to identify astronomical systems, given that it is the strongest. They are common in the spectra of star-forming regions and around extended planetary nebulae and supernova remnants, allowing the investigation of these and a number of other systems.  

%**[ Neste parágrafo abaixo tem que descrever todos os objetos que tem linha de emissão que depois serão apontados no trabalho.

Emission lines trace  discs \citep{Schwope:2000, Ratti:2012} in stellar and extragalactic systems, including their geometrical characteristics from line profiles \citep{Horne:1986}. Some short-lived evolutionary stages of stars may also be marked by emission lines. This category includes young stellar (YSOs) and Herbig-Haro (HH) objects surrounded by a 
nebula-like structure, evolved systems like 
post-asymptotic and some asymptotic giant branch (AGB), some
red giant stars (RGB), active late-type dwarfs, and supernova remnants. 
Amongst massive stars, emission lines are observed, for example, in Be stars when 
circumstellar decretion disks are present and in Wolf-Rayet (WR) stars, which are marked by strong winds.  
Interacting binary systems experiencing mass exchange, 
%the nature of donor stars 
%\rlopes{and accreting compact objects} 
%in 
% \citep{Steeghs:2002, Spaandonk:2010,
%Casares:2015,Casares:2016}.
%\sout{the compact objects as black holes \citep{Casares:2016}.} 
like symbiotic stars (SySt) in which a white dwarf (WD) or a neutron star accretes matter from a supergiant star, and Cataclysmic Variables (CVs), pairs of  white dwarf plus a low mass star, may be added to this category. Emission lines in Planetary Nebulae (PNe) allow the study of the remnant envelope of dying stars, illuminated by their residual white dwarf. 

At much larger scales, emission line objects such as H II regions, around regions with young stars, allow us to map and study the star formation history of the far reaches of our Galaxy and of distant galaxies. Emission lines in general (not necessarily Halpha) in starburst galaxies and QSOs, the most luminous objects, allow us to probe conditions when the Universe was young, in particular the first generations of stars and the primordial formation of heavy elements. 

Most of the classes mentioned above are not homogeneous 
and far from complete even in the local Universe, with some being highly populated while others less so. For example, 
there are $\sim$\,320 SySt known, with only $\sim$\,65 out of them harbored in galaxies other than the Milky Way \citep{Akras:2019a}.
%\sout{around 323 symbiotic system have been identified
%from which 257 belong to the Galaxy and  $\sim$66 are extra-galactic
%objects \citep{Akras:2019a}.} 
The number of known PNe in our Galaxy is higher, with 3500 objects listed in catalogs \citep{Parker:2016}, but it is argued that it represents only about 15-30\% of the total expected in the Galaxy (Frew, 2008; Jacoby et al., 2010).
%\sout{The same occurs with PNe from witch around
%3500 of them are been cataloged \citep{Parker:2016}, this current number
%of PNe represents only about 15-30\% of the estimated total of Galactic
%PNe (Frew, 2008; Jacoby et al., 2010) showing that a small fraction of the
%PNe have been cataloged.} 

Many H$\alpha$ surveys in a variety of angular resolution, sky coverage, 
and sensitivity have been carried out.
Some of them, with modest spatial resolutions, revealed 
spatially resolved, extended nebular emission to study supernova remnants, galaxy
groups, and star forming regions \citep[e.g.][]{1976MmRAS..81...89D}. 
Others with 
higher 
spatial resolution 
%\sout{surveys such as the INT Photometric H$\alpha$ survey
%(IPHAS; \citealt{Drew:2005, Barentsen:2014}) have focused in the study of}
disclosed compact emission-line sources
in the Galaxy and sources in nearby galaxies.
%\sout{, typically with objects
%in different stage of stellar evolution}. 
Examples are the INT Photometric H$\alpha$ survey
(IPHAS; \citealt{Drew:2005, Barentsen:2014}), the  SuperCOSMOS H$\alpha$ Survey with the UK Schmidt Telescope (UKST) of the Anglo-Australian Observatory \citep{2005MNRAS.362..689P}, and the ongoing VST Photometric H$\alpha$ Survey (VPHAS+; \citealt{Drew:2014}).

Traditionally, H$\alpha$ emitters are revealed directly from images and in color-color diagrams from photometric surveys observing the sky with at most five - generally broad-band or H$\alpha$ - filters.
%, and over small areas of the sky [** Luís, é verdade que são sobre pequenas areas, se sim colocar isto depois de - filters.]
For example, the (r - H$\alpha$) versus (r - i) colour-colour diagram or similar was used
to find new 
CVs \citep{Witham:2006,Witham:2007}, 
YSOs \citep{Vink:2008}, 
SySt \citep{Corradi:2008, Corradi:2010, Corradi:2011}, 
early-type emission-line stars \citep{Drew:2008}, 
and PNes \citep{Viirone:2009, Sabin:2010}. 

There are two new ongoing multi-band surveys observing the sky in a systematic, complementary way, with 5 broad and 7 narrow-band filters, including H$\alpha$: the Javalambre Photometric Local Universe Survey
(J-PLUS\footnote{\url{https://www.j-plus.es}}; \citealp{Cenarro:2018}), covering the Northern celestial hemisphere, and the Southern-Photometric Local Universe Survey
(S-PLUS\footnote{\url{http://www.splus.iag.usp.br}}; \citealp{Mendes:2019}), doing the same to the Southern sky with a twin 80\,cm telescope. They are paving the way for an even more ambitious survey, the Javalambre Physics of the Accelerating Universe Astrophysical Survey (J-PAS), which will observe the Northern sky with 56 narrow-band filters. As source hunters, the spectral energy distribution provided by these surveys also provide an unprecedented classification from photometry. However, as part of the new era of big data, efficient investigation tools are required to deal with their massive data sets and machine learning techniques have been increasingly used.

%Akras:2019}. And the same diagram in conjunction with new
%ones shows to be very efficient to find for PN candidates \citep{Gutierrez:2020}.
%In general terms, \citet{Witham:2006} presented a methodology and first results
%in looking for emission line sources in narrow-band surveys.

%\sout{The Anglo-Australian Observatory UKS
%chmidt Telescope Supercosmos H$\alpha$ Survey (Parker et al. 2005) is another
%H{$\alpha$} survey of the Southern Galactic Plane and Magellanic Cloud which
%has covered to b $\sim$ 10-13$^{\circ}$ (verificar esto). 
%Currently ongoing is
%the VST Photometric H$\alpha$ Survey of the Southern Galactic Plane and Bulge
%(VPHAS+; \citealt{Drew:2014}) that will cover the Galactic bulge and plane in
%five filters.} 

%\sout{REWRITTEN AND REPLACED ABOVE: Like VPHAS+, others ongoing surveys that are used to study %the population of
%emission line objects are the The Javalambre Photometric Local Universe Survey
%(J-PLUS\footnote{\url{https://www.j-plus.es}}, \citealp{Cenarro:2018})
%and the Southern-Photometric Local Universe Survey
%(S-PLUS\footnote{\url{http://www.splus.iag.usp.br}}, \citealp{Mendes:2019})
%are providing observations of the Galactic halo covering both northern and
%southern celestial hemispheres in a systematic way with twin telescopes
%using the same set of multi-band filters. In addition to the H$\alpha$ filter,
%which is already vastly applied to systematically searching for H$\alpha$ emitters
%the telescopes offer 11 more filters. And more ambitious yet the JPAS survey that
%will the same area of J-PLUS in 56 narrow-band filters.}

%\sout{ADDRESSED ABOVE: Traditionally, color-color diagrams based in H$\alpha$ filter are been %used to
%identify H$\alpha$ emitters.  The analysis the color-color diagram  (r - H$\alpha$)
%versus (r - i) has resulted on the discovered of new emission line objects, for
%instance \citet{Witham:2006, Witham:2007}  used the (r - H$\alpha$) versus (r - i)
%colour-colour diagram to find for new CV. On the other hand, \citet{Vink:2008}
%reported the discovery of YSOs by using this same colour criteria. In this sense using
%this methology a variety of classes of objects are been identified, which include
%symbiotic stars \citep{Corradi:2008, Corradi:2010, Corradi:2011}, early type emission
%line stars \citep{Drew:2008} and planetary nebulae \citep{Viirone:2009, Sabin:2010}.
%Recently, by using this same color diagram were also identified compact PN candidates
%in VPHAS+ catalog \citep{Akras:2019}. And the same diagram in conjunction with new
%ones shows to be very efficient to find for PN candidates \citep{Gutierrez:2020}.
%In general terms, \citet{Witham:2006} presented a methodology and first results
%in looking for emission line sources in narrow-band surveys.}

%\comment{I RECOMMEND TO REMOVE THIS BUT NOT ERASING IT FOR NOW; JUST KEEP IT HERE COMMENTED SO THAT WE MAY PLACE THE INFORMATION SOMEWHERE AT THE END - OR REMOVE IT AT ALL. I AM TRYING JUST TO KEEP THE INTRODUCTION AS SHORT AND DIRECT TO THE POINT AS POSSIBLE:} 
%\sout{In this era of big data on astronomy, machine learning techniques are becoming in important statistical tools for the analysis and find meaning from massive data sets. Particularly, unsupervised machine approaches have showed a promised in various applications, especially in automatic classification task. Including object classification and selection, using galaxies with active galactic nuclei as example \citep{Geach:2012}, morphological analysis of galaxies \citep{Martin:2020}, classification of variable stars, relying only on the similarity among light curves. \citep{Valenzuela:2018}. Using unsupervised machine learning can be very advantageous because they do not require a labeled data training sets. Unlike of supervised methods like Random Forest algorithm. Instead, unsupervised techniques are generally based in the data itself to identify patrons, e. g. cluster of similar objects, in some pre-defined feature space where the data are defined.}

Here we present H$\alpha$ emitters from the S-PLUS iDR3 by using color-color diagrams and unsupervised machine learning techniques, classifying them as blue or red fonts and also proposing a class to which they belong. Section~\ref{sec:obser} describes
the observations related to S-PLUS project, as well as important information of 
the third data release, section~\ref{sec:metho} presents the technique implemented 
to select the H{$\alpha$} emitters and machine learning approaches used to divide 
the sample in two populations based in their colors, in section~\ref{sec:results}  
the results and our findings are discussed and finally section~\ref{sec:conclu} 
discusses our study and conclusions.

%\sout{In this work, we used S-PLUS observations of the southern hemisphere to search
%for objects with an excess of H{$\alpha$} using automatic methods based on the
%(r - H$\alpha$) versus (r - i) color-color diagram. We have also used color criteria
%based in (g - r) and (z - g) in conjunction to unsupervised machine learning
%techniques to split the final list in those with blue and red continuum. The
%paper is organized as follows...}

\section{Methodology}
\label{sec:metho}
\subsection{Observations: The S-PLUS project}
\label{sec:obser}

This paper uses data from S-PLUS, a photometric sky map being carried 
out since end of 2017 using a 0.83m robotic telescope located at Cerro 
Tololo, Chile \citep{Mendes:2019}.
The project is surveying the Southern sky with 12 filters from the Javalambre 
filter system \citep{Martin-Franch:2012} spanning the wavelength range from 3000\AA\ 
to 10000\AA. The filters used are seven narrow-band filters
(\textit{J}0378, \textit{J}0395, \textit{J}0410, \textit{J}0430,
\textit{J}0515, \textit{J}0660,  and \textit{J}0861) 
and
five broad-band sloan-like \citep{Fukugita:1996} filters (\(u, g, r, i~\mathrm{and}~z\)).
The narrow-band $J0660$ filter  covers 
the H{$\alpha$} plus the [N II] spectral lines for sources up to redshift of approximately 0.02 \citep[see Fig. 1 after][]{Martin-Franch:2012}. 
%\sout{For more detailed about the configuration of S-PLUS filter set see %Figure~\ref{fig:curves}, which shows the Javalambre filter
%system \citep{Martin-Franch:2012} overlapping are the optical spectra of
%several class of emission line objects on which it is possible to see that the
%H{$\alpha$} line falls into the $J0660$ filter, except for the QSOs. Other strong emission %lines of the QSOs are detected on the $J0660$-band filter due to their red-shift, in the
%following sections, this topic will be addressed in more detail}. 

The S-PLUS data used here are those of the third data release (DR3). It is composed by about 60 million objects distributed over $\sim$2,000 deg$^2$ (out of the $\sim$8,000 deg$^2$ of sky, at high Galactic latitudes, $ > 30$~deg, that are planned to be done by S-PLUS by its completion). No area from the Galactic disk and bulge were included in this study (S-PLUS plans to cover $\sim$\,1,300 deg$^2$ of such areas), given that these will be available only in Dr4.
%which is expected to complete the main survey - plus $\sim$\,1,300 deg$^2$ of the 
%Galactic plane and bulge. 
In order to ensure that the best data are used, we consider only objects which were detected at least in the $r$, $i$ and
  $J0660$ bands, 
  %and having for each of them uncertainties at most of 
  with errors less than 0.2 mag.

%\sout{The current data release contains about 60 millions of objects covering a total
%area of $\sim$2,000 deg$^2$, at high Galactic latitudes ($ > 30$~deg) using a
%dedicated 0.83m robotic telescope,the T80-South (T80S), located at Cerro Tololo,
%Chile. In general terms, S-PLUS will cover $\sim$ 9,000 deg$^2$ and   
%an additional 1,300 deg$^2$ of the 
%Galactic plane and bulge to enable Galactic studies.
%In this work, we focus on the aspects that are of
%particular interest to the third data release of the S-PLUS main survey. Additional %information about S-PLUS can be found in \citet{Mendes:2019}.} 

\begin{figure}
    \includegraphics[width=0.9\linewidth]{Figs/splus-filter.pdf}
    \caption{Transmission curves of the S-PLUS filter set. The narrow-band filter
      $J0660$ includes the H$\alpha$ emission line. Over-plotted are spectra of different
      classes of emission line objects. From top to bottom:  a PN, a symbiotic star, an extragalactic H II region, a blue compact/H II galaxy, a YSO, a CV star, a B[e] star, a star forming galaxy and a QSO with a redshift of XX.}
    \label{fig:curves}
\end{figure}

The first goal of this paper is to identify objects that may be bona-fide H$\alpha$ emitters among sources detected in the S-PLUS DR2.
 For this, we
applied an iterative and automatic technique to select objects with
an excess in the $J0660$ band, which is consistent with the detection of the H{$\alpha$} line in emission. Next, the selected objects were separated
into two groups, one having redder and the other bluer emission. 
This classification was done 
by using optical colors in combination with unsupervised machine learning/statistical tools. These procedures are described 
in the following sections.

%\subsection{Initial selection sample}
%\label{sec:}

%\comment{Que tal excluir essa se\c c\~ao e colocar todo o conte\'udo na se\c c\c \~ao 2?}
%\luis{Voce quer dizer unir seçao 2 e 3 em uma só. A seçao 2 e muito curta, ne}


%\begin{enumerate}
%\item \sout{The sources must have detection in the filters: $r$, $i$ and
% $J0660$. To assure that, we select objects that must have an error minor or
%  equal to 0.2 in each of these three filters.}

%\item \sout{Must have magnitude in the $r$-band up to 21.}
  
%\end{enumerate}

\subsection{Finding the main stellar locus and selecting the H{$\alpha$ emitters}}
\label{sec:find-halpha}

Before applying the method, which is based on the excess in the $J0660$-band magnitude with respect to that in the $r$-band, we first divided our sample into four sub-samples, splitting the systems according to their magnitudes in the $r$-band. In this way, we avoid mixing up bright and faint objects that tend to have low and high uncertainties, respectively, in the same color-color diagram. Otherwise, the selection criteria would be potentially affected by the intrinsic scatter in the measurement of faint objects. We considered the following four sub-samples: (i) $r$-band $ < 16$, (ii) $16 \leq r < 18$, (iii) $18 \leq r < 20$, and $20 \leq r < 21$.

The identification of H$\alpha$ emitters was based on the method applied successfully by \citep[IPHAS][]{Witham:2008} to create the INT/WFC Photometric H$\alpha$ Survey, given that they use filters that are also available in our survey: $r$, $J$0660, and $i$ filters. The same technique was also used by \citet{Scaringi:2013} and \citet{Wevers:2017} to reveal H{$\alpha$} emitters.

%\comment{QUE TAL MOVER PARA O INICIO? ME PARECEU QUE ESTANDO AQUI, ESSA PARTE QUEBRA O FLUXO.} %\sout{To select the emission line objects we used a similar method created and
%implemented by \citet{Witham:2008}. It is possible to do that because
% S-PLUS has similar filters to the IPHAS project, which are the
% $r$, $J$0660 and $i$. This technique was used by \citet{Scaringi:2013}
% to identify blue objects with H{$\alpha$} emission line.
% \citet{Wevers:2017} also applied this methodology to create a catalogue
% of H{$\alpha$} emission line candidates. These previous results show an high 
% effectiveness of the technique. In this
% order of ideas, we attempted this methodology in S-PLUS.}

We first generated the ($r$ - $J0660$) versus ($r$ - $i$)
diagram for each magnitude bin and attempted to put in evidence the loci mainly 
occupied by main-sequence and giant stars with a linear regression fit. 
We then implemented an iterative $\sigma$-clipping technique so that, by construction, H$\alpha$ emitters should satisfy the condition: %\comment{(CORRETO?)} \luis{Acho sim, o objetivo de fazer um iterativo sigma clipping e encontrar o melhor ajuste posivel}

\begin{equation}
  (r - J0660)_{\mathrm{obs}} - (r - J0660)_{\mathrm{fit}} \geq C \times \sqrt{\sigma^2_{\mathrm{s}} - \sigma^2_{\mathrm{phot}}}
  \label{eq:criterion}
\end{equation}
 where $\sigma_{\mathrm{s}}$ is the root mean squared value of the residuals around
 the fit and $\sigma_{\mathrm{phot}}$ is the error on the observed $(r - J0660)$ colour idex.
 $C$ is a constant which has the value 4 following \citet{Wevers:2017}.
The fits were carried out with the aid of the python library. \texttt{astropy.modeling}
\footnote{\url{https://docs.astropy.org/en/stable/modeling/index.html}}

Figure~\ref{fig:criteria-color-plot} illustrates the procedure applied. The continuous black lines represent the initial
fit and  the dashed lines indicate the 4-$\sigma$ clipping fit lines. The dotted lines are
the cut selection criteria for the H{$\alpha$} emitters -- the 4-$\sigma$ above of the final
fit. Note that these cut lines are only an approximations because only the residual around the fit is taken into account. 
Note that the approach includes the phothometric uncertainties of the ($r$ - $J0660$) color for each
individual data source (see Equation~\ref{eq:criterion}).

\begin{figure*}
  \setlength\tabcolsep{0pt}
  \setkeys{Gin}{width=0.5\linewidth}
  \begin{tabular}{ll}
    (a) & (b) \\
    \includegraphics[trim=10 0 65 20, clip]{Figs/diagram-DR3-errorFlag0-3f-16r}
    & \includegraphics[trim=10 0 65 20, clip]{Figs/diagram-DR3-errorFlag0-3f-16r18}\\
    (c) & (d) \\
    \includegraphics[trim=10 0 65 20, clip]{Figs/diagram-DR3-errorFlag0-3f-18r20}
    & \includegraphics[trim=10 0 65 20, clip]{Figs/diagram-DR3-errorFlag0-3f-20r21}\\
  \end{tabular}
  \caption{An illustration of the selection criteria used to identify strong
    emission-line objects via colour-colour plots. The data shown here are all from the
    S-PLUS DR3. The data are split up into four magnitude bins, as shown in the four
    panels. Objects with H{$\alpha$} excess should be located near the top of the
    colour-colour diagrams. The thin continuous lines illustrate the original linear
    fit to all the data (grey points). The dashed lines represent the final
    fits to the stellar locus of points which were obtained by applying an iterative
    $\sigma$-clipping technique to the initial fit. The actual cuts used to select
    H{$\alpha$} emitters are shown by the dotted lines. Objects selected as H{$\alpha$}
    emitters must be located above. Note that the cut lines (selection criteria) shown
    here are only approximate, as the actual selection criterion also considers the
    errors on each source. This means that an object could be in the bottom
    right-hand panel is not selected despite clearly lying above the cut line
    ({\sc explicar esta última frase mejor}).}
  \label{fig:criteria-color-plot}
\end{figure*}

Once having selected H$\alpha$ emitters, we proceed with a visual inspection of 
their spectral energy distributions as seen from the whole S-PLUS coverage, namely 
S-spectra, and their false-color images. 
%\sout{After the algorithm was applied to all data, we visually inspection-ed the
%resulting list by seeing the S-spectra and corresponding colored image.} 
For illustration, Figure~\ref{fig:Spectra} shows how an H$\alpha$ emitter looks like from the S-PLUS data.
%\sout{Figure~\ref{fig:Spectra} shows an example of how looks like the
%S-spectra\footnote{S-spectra signify the S-PLUS emission in flux
%or magnitude unities of an object in all twelve bands.} of the sources in
%magnitude unities selected with the method explained above. This object
%clearly exhibits strong H$\alpha$ emission line.}

\begin{figure}
\includegraphics[width=0.9\linewidth]{Figs/photopectrum_splus_HYDRA-0026-052331_Good-LD-Halpha-DR3_noFlag_merge-takeoutbad-Final_PStotal.pdf}
\centering
\llap{\shortstack{%
        \includegraphics[width=0.25\linewidth, trim=350 10 350 20, clip]{Figs/HYDRA-0026-052331_158.85933642579576--24.753136157195524_80_r.pdf}\\
        \rule{0ex}{0.87in}%
        }
  \rule{0.06in}{0ex}}
\caption{S-spectra of a random object with emission lines select with the algorithm explained
  above. Squares represent the SDSS-like broad-band filters. From left to right 
  they are \(u, g, r, i~\text{and}~ z\). Circle symbols are the narrow-band filters,
  which from left to right represent \textit{J}0378, \textit{J}0395, \textit{J}0410,
  \textit{J}0515, \textit{J}0660 and \textit{J}0861. The inset figure is the coloured image 
  of the object which was produced by combining all twelves bands.}
\label{fig:Spectra}
\end{figure}

\begin{figure}
	\includegraphics[width=1.1\linewidth]{Figs/final-emitters.pdf}
        \caption{Colour-colour diagram with all the emission line objects selected
          from S-PLUS iDR3. Size of the symbols represent the measured FWHM assuming
          a Gaussian core (for more detail see \citealt{Fernandes:2021}). Coloured
          bar indicates the magnitude values of the r-band. The contours represent
          the S-PLUS synthetic photometry of main-sequence and giant stars loci from 
          the library of stellar spectral energy distributions of \citet{Pickles:1998}.}
    \label{fig:emission}
\end{figure} 

Figure~\ref{fig:emission} exhibits the distribution of the H{$\alpha$} emitters on the
($r - J0600$) versus ($r - i$) color-color plane. 
It also present the main-sequence and giant stars loci as derived from synthetic spectra collected of \citet{Pickles:1998} convolved with the transmission of S-PLUS filters, 
and defined in the AB magnitude system \citep{Oke:1983}.
%\sout{
%,
%stellar and the the S-PLUS synthetic photometry
%of the stellar locus. This stellar locus is represented by the contours in the diagram and was %obtained by using the filter transmission profiles, shown in Fig.~\ref{fig:curves}, and the %library of stellar spectral energy distributions of \citet{Pickles:1998}. This synthetic %magnitudes are defined in the AB magnitude system \citep{Oke:1983}.}

At this point, we can say that
the algorithm implemented here worked well because the sources
are located  above of the loci of the main and giant stars. The wide distribution of fonts across
the ($r - J060$) and ($r - i$) colours indicates that several types of objects were selected.
For instance, higher values on the  ($r - J0660$) color of some sources could be indicating
that they are H II regions or/and blue compact galaxies and PNe. On the other hand, the 
($r$ - $i$) color indicates the  reddened nature of sources such as SySt and young stellar objects
or in the opposite case, sources with a strong blue continuum as cataclysm variables and QSOs.

\begin{figure*}
\includegraphics[width=0.9\linewidth]{Figs/halpha-emitters-galactic-aitoff.pdf}
\centering
\llap{\shortstack{%
        \includegraphics[width=0.25\linewidth, trim=10 10 -40 0]{Figs/distribution-bgalactic.pdf}\\
        \rule{0ex}{1.8in}%
      }
  \rule{1.0in}{0ex}}
\caption{Distribution of H{$\alpha$} emitters in Galactic longitude and latitude
  coordinate. Inset figure represents distribution of the objects in Galactic latitude.}
\label{fig:aitoff-distribution}
\end{figure*}

Figure~\ref{fig:aitoff-distribution} displays the distribution of all emitters in Galactic 
latitude and longitude. The density map regions represent the spatial positions of the objects
on the sky. The surface density of $J$0660-excess objects is highest near the Galactic plane.

Once, we felt confident of our sample of H{$\alpha$} emission lines sources,
we proceeded to classify the objects into two groups; one group containing those
objects with a strong blue continuum and another formed by sources with  intense emission of the
continuum on the red part of the spectra. 

\subsection{Unsupervised machine learning/clustering techniques}
\label{sec:clustering}

Objects of our sample were
divided 
into two groups,
distinguishing the bluer from the redder population.
To do this, we apply an unsupervised machine learning approach to implement
 two clustering techniques: hierarchical clustering and hierarchical
density-based cluster selection, both based on the $(g - r)$ and $(z - g)$ colors, 
whose results were mutually compared. 

\subsubsection{Hierarchical agglomerative clustering}
\label{sec:Hierar}

``Hierarchical clustering'' (HC) belongs to the family of clustering algorithms of which clusters are constructed by recursively grouping and splitting the sources. 
It is an unsupervised algorithm in the sense that it does not require a training sample or pre-conceived hypotheses. 
From it, individuals are grouped 
based on patterns in a given space of parameters and on the levels of similarity
at which the groupings change \citep{Jain:1999}. 
In the end, HC allows a diagrammatic representation of groups as  a tree diagram, a dendrogram, and thus following a hierarchical structure.  
%\texttt{dendrogram}\footnote{Dendrogram is a diagram representing tree, which shows hierarchical relationship between objects.}

There are two types of hierarchical
clustering: 
the \textit{hierarchical agglomerative clustering} (\texttt{HAC}; the one; used in this work), 
which is ``bottom-up'' approach, and the \textit{hierarchical divisive clustering}, 
with a ``top-down'' approach. 
The \texttt{HAC} consists of building a binary merge tree, starting from 
each data element stored at the leaves (interpreted as individual clusters)
and proceed by merging two by two the ``closest'' sub-sets (stored at nodes)
until it reaches the root -- unique cluster -- of the tree that contains all the elements
of the data set. The agglomerative term is used because individual data
 are successively agglomerated into higher-level. In each iteration, two ``nearby'' clusters
are selected 
%\sout{that are considered as close as possible}. \sout{These clusters are merged and
%replaced with a newly created merged cluster with more data points}
and collapsed them into a new, more populated group \citep{Mann:2013, Aggarwal:2015}.
 Thus, each merging step reduces the number of clusters. 
 % by a factor of ``1'' \comment{(Eu não entendi esse 1)}. 
 %\sout{Therefore, the method needs to be 
%designated for measuring proximity between cluster containing multiple data points,
%so that they may be merged \citep{Mann:2013, Aggarwal:2015}}. \sout{We could describe how to
%work the algorithm in three simple steps:}
The procedure may be summarized in three steps:

\begin{enumerate}
\item Initially, each individual data point represents one cluster, i.e. ``leaves of the tree''. 
This means that at the beginning, the number of total the clusters is the number 
of elements of the data set.
\item Through a looping process, the clusters or nodes are merged into groups that 
have the maximum similarity between them.
\item At the end of the process, all the nodes belong to an unique cluster, ``the root of the tree'' structure.
 \end{enumerate}

In the opposite direction is the other type of hierarchical clustering, the \textit{hierarchical divisive clustering}. 
Following its ``top-down'' approach, the investigation starts from data as being in one only cluster then moving down recursively
in the hierarchy to small groups.
%\sout{the data start in one cluster (the root of the tree), and splits step by step
%are performed recursively as one moves down the hierarchy.} 
In simple words, hierarchical (agglomerative and divisive) clustering 
algorithms intend to 
gather similar objects into groups called clusters or nodes in the space of parameters which is investigated. 
%\sout{The endpoint is a set of clusters, where each cluster 
%is distinct from each other cluster,
%and the objects within each cluster are broadly similar to each other.}


%Starting from each label at the bottom, it is possible to see a vertical line up to a
%horizontal line. The height of that horizontal line tells you about the distance
%at which this label was merged into another label or cluster. You can find that
%other cluster by following the other vertical line down again. If you don't
%encounter another horizontal line, it was just merged with the other label
%you reach, otherwise it was merged into another cluster that was formed earlier.

%Hierarchical clustering can be performed with either a distance matrix or raw data.
%When raw data is provided, the software will automatically compute a distance
%matrix in the background. The distance matrix below shows the distance between
%six objects.

%Hierarchical clustering starts by treating each observation as a separate cluster.
%Then, it repeatedly executes the following two steps: (1) identify the two
%clusters that are closest together, and (2) merge the two most similar clusters.
%This iterative process continues until all the clusters are merged together.
%This is illustrated in the diagrams below.

\subsubsection{Hierarchical density-based cluster selection}
\label{sec:hdbscan}

Hierarchical density-based cluster selection (\texttt{HDBSCAN}; \citealp{Campello:2013})
is another unsupervised machine learning algorithm that relies on clustering. It is based on a slightly
modified version of density-based spatial
clustering of applications \sout{with noise} (\texttt{DBSCAN}; \citealp{Ester:1996}) which
declares points as noise. It is an algorithm that assumes clusters that are characterized
by ``islands'' of high density in the sea of the parameter space. Such clusters constitute
data partitions that have a higher density
than their neighbors  \citep{Ntwaetsile:2021}. \texttt{HDBSCAN} takes forward the
DBSCAN concept by introducing a hierarchy to the clustering, with ``persistent''
clusters finally extracted from the hierarchical tree. The main advantage of
\texttt{HDBSCAN} in comparison with the predecessor consists in the possibility in finding
clusters of variables densities and different shapes. Following \citet{Malzer:2021}
and \citet{Ntwaetsile:2021}\rlopes{, it} works as follows:

\begin{enumerate}
\item \texttt{HDBSCAN} define the ``core'' distance for a point $x$, core$_k(x)$,
  as the distance of an object to its $k$th nearest neighbor. This mean that lower values of core$_k(x)$ 
  represent higher densities and vice-versa.

\item The ``mutual readability distance'' between two points $a$ and $b$ is defined as
  $d_{\mathrm{m}}(a, b) = \mathrm{min}{\mathrm\{core}_k(a), \mathrm{core}_k(b), d(a, b)\}$,
  where $d(a, b)$ is the distance between $a$ and $b$ according, for instance,
  Euclidean metric. The mutual readability distance allows points in dense regions
  to stay close together and those that are in less dense regions to move away.% \comment{(N\~ao ficou claro para mim.)}.

\item The mutual readability plot is used to construct the minimum spanning tree,
  and sorting its edges by the mutual readability distance resulting in a
  hierarchical tree structure. The hierarchy of connected components is
  defined by sorting the edges of the tree by distance in reverse order,
  describing a dendrogram (this diagram explained in \ref{sec:Hierar}). 
  This is the structure from which the cluster will be identified.

\item \texttt{HDBSCAN} allows to extract clusters of variable density, effectively,
  by cutting the dendrogram at different levels of grouping. 

\item The cluster tree is condensed into a simpler structure (see, for instance, 
Figure~\ref{fig:condensed-trees} of Appendix~\ref{sec:trees}). 
Considering the single main trunk which contains all the data points, the tree splits 
into branches. A condensed cluster hierarchy can be described by considering the 
number of points
  that are kept in each branch as it splits. It is important to mention that there 
  is a key parameter called minimum
cluster size. If a given branch divides into two, with a branch
  containing fewer points than the minimum cluster size means, the large branch
  ``persists'' and the smaller split branch ``falls out'' of the  cluster. If
  a branch splits into two with both branches exceeding the minimum cluster size,
  both new branches are conserved.

\item The clusters are extracted on the notion of persistence in the hierarchy.
  The parameter $\lambda = d_{\mathrm{m}}^{-1}$ is defined, and each cluster has a
  $\lambda_{\mathrm{birth}}$ (the point at which the cluster split off) and
  $\lambda_{\mathrm{death}}$ (the point when the cluster split into other clusters).
  In each cluster, we have $\lambda_p$ describing when each point fell out of the
  cluster (or was split off into a new cluster), so that $\lambda_{\mathrm{birth}} \leq \lambda_{p} \leq \lambda_{\mathrm{death}}$.
  Cluster stability $S$ is defined as the sum of $\lambda_{p} - \lambda_{\mathrm{birth}}$
  for all points in the cluster. To extract clusters, the following procedure is
  implemented. First, %\sout{select all leaves as cluster} 
  each leave constitutes a cluster. Then, moving through the
  hierarchy, it is considered the stability of a parent cluster $S_p$ and its $n$ descendants
  $S_d^{0,1,2,...,n}$: if $S_p > \sum_{i=0}^{n} S_d^i$, we unselect all the descendants; otherwise,
  %\sout{If $S_p < \sum_{i=0}^{n} S_d^i$\rlopes{,} then} 
  the cluster stability is set as 
  $S_p = \sum_{i=0}^{n} S_d^i$. At the root node we have our set of selected 
  cluster. Any point in the sample that does not fall into one of the selected
  clusters is defined as noise.

%\item \sout{(Por favor tente expressar esse item de uma meneira melhor; eu estou com receio de %mudar e alterar o sentido; depois marque-o, e eu retornarei a esse ponto para ver)} 
%The selected clusters can be used to label points. The concept 
  %of $\lambda_p$ within a cluster, when normalized between O and 1 provides a means
  %of characterization a probability that a given point belongs to the cluster or
  %alternative a measure of the strength of membership. 
  
  \item \luis{Adopting the \texbf{soft clustering} or \texbf{fuzzy clustering} technique is possible to
  mitigate the need to establish or define had cluster membership limit. In fact, each source
  has finite probability of belonging to every selected cluster. In this approach all points 
  (including noises) are not assigned to cluster label, but are instead assigned to a vector of probabilities, which its length is equal to the number of clusters found. With this can be solved the noise classification.}
\end{enumerate}

The {\sc HDBSCAN} algorithm starts off much the same as {\sc DBSCAN}: transforming the space
according to density, exactly as DBSCAN does, and perform single linkage
clustering on the transformed space. Instead of taking an epsilon\footnote{Epsilon parameter in {\sc DBSCAN} 
represents the maximum distance between two samples for one to be considered as in the neighborhood 
of the other. This is the most important {\sc DBSCAN} parameter to choose appropriately for 
the data set and distance function.} value
as a cut level for the dendrogram, however, a different approach is taken:
the dendrogram is condensed by viewing splits that result in a small number
of points splitting off as points ``falling out of a cluster''. This results
in a smaller tree with fewer clusters that ``lose points''. That tree can then
be used to select the most stable or persistent clusters. This process allows
the tree to be cut at varying height, picking our varying density clusters
based on cluster stability. The immediate advantage of this is that we can
have varying density clusters; the second benefit is that we have eliminated
the epsilon parameter as we no longer need it to choose a cut of the dendrogram.
Instead we have a new parameter \texttt{min\_cluster\_size} which is used
to determine whether points are ``falling out of a cluster'' or splitting
to form two new clusters. 
%This trades an unintuitive parameter for one that
%is not so hard to choose for EDA (what is the minimum size cluster I am
%willing to care about?).


%Hierarchical Density-Based Spatial Clustering of Applications with Noise (Campello,
%Moulavi, and Sander 2013), (Campello et al. 2015). Performs \texttt{DBSCAN} overvarying
%epsilon values and integrates the result to find a clustering that gives the best
%stability over epsilon. 
%This allows \texttt{HDBSCAN} to find clusters of varying densities
%(unlike \texttt{DBSCAN}), and be more robust to parameter selection. The library also
%includes support for Robust Single Linkage clustering (Chaudhuri et al. 2014),
%(Chaudhuri and Dasgupta2010), GLOSH outlier detection (Campello et al. 2015), and
%tools for visualizing and exploring cluster structures. Finally support for prediction
%and soft clustering is also available.

In the last years, \texttt{HDBSCAN} have been used in astronomy for different tasks.
\citet{Jayasinghe:2019} presented the second data release Milky Way Project (MWP), a citizen
science initiative on the Zooniverse platform, presents internet users with infrared (IR)
images from Spitzer on which were aggregate $\sim$3 million classifications made
by volunteers during the years 2012-2017 to produce the DR2 catalogue, which
contains 2600 IR bubbles and 599 candidate bow shock driving stars.
The reliability of bubble identifications was made by using \texttt{HDBSCAN}.
On the other hand, \citet{Webb:2020} used \texttt{HDBSCAN} for transient discovery.
Recently, \citet{Ntwaetsile:2021} used it to group radio sources into a sequence
of morphological classes, illustrating a simple methodology to classify and
label new, unseen galaxies in large samples. This approach was also implemented to 
identify stellar groups in Canis Major OB1 \citep{Santos-Silva:2021}.

\subsection{Grouping the H{$\alpha$} emitters into blue and red colour-types}
\label{sec:apply-hac-hdbscan}

We adopt the $g$, $r$, and $z$ broad-bands spreading from the blue to the red in the optical 
to distinguish the blue from the red populations of our sample. In this way, we start by 
exploiting the $(g - r)$ and $(z - g)$ colors and the potential of the corresponding diagram 
to put in evidence different classes of systems. %\comment{Eu recomendo uma mudan\c ca na 
%apresenta\c c\~ao, come\c cado com um texto assim... depois migrando para falar que populamos 
%o diagrama com fotometria sint\'etica para certos tipos de objetos e calculadas via espectros 
%observados para outros tipos de objetos. E s\'o depois entrar na descri\c c\~ao que 
%est\'a sendo feita no in\'icio da pr\'oxima se\c c\~ao - dizendo onde cada classe cai em tal diagrama.
%O que acha? E desse modo o t\'itulo da subse\c c\~ao 3.4.1 pode ser eliminado.}

%Given that $(g - r)$ and $(z - g)$ colours are the input data to implement the machine learning %approaches, we first will speak about them.

%\subsubsection{\sout{$(g - r)$ versus $(z - g)$ color-color diagram}}

\begin{figure}
	\includegraphics[width=0.9\linewidth]{Figs/Fig-SPLUS-gr-zg.pdf}
        \caption{The $(g - r)$ versus $(z - g)$ synthetic color-color diagram of
          several classes of emission lines objects. Included in the diagrams, there
          are families of CLOUDY modelled halo PNe spanning a range of properties (density
          map region). Cyan circles represent S-PLUS photometry from observed spectra
          Grey diamonds represent H II regions in NGC 55. Red
          boxes display symbiotic stars, this group also includes Galactic and
          external SySt from NGC 205 IC 10 and NGC 185,. Yellow circles correspond to
          cataclysmic variables (CVs) from SDSS. Pink circles indicate  blue compact
          galaxies (BCGs) from SDSS. Orange triangles refer to SDSS
          star-forming galaxies (SDSS SFGs). SDSS QSOs at different redshift
          ranges are shown as light blue diamonds, and YSOs from Lupus and
          Sigma Orionis are represented by salmon stars. The diagonal dashed line represents
         a subjective criterion to separate the objects into two color types.  The arrow
         indicates the reddening vector with AV $\simeq$ 2 mag. {\sc Maybe 
         is a good idea to put the reddened vector of this color-color diagram.}}
    \label{fig:synthetic}
\end{figure}

In order to find the best color-color diagram to separating the final sample of
emission line objects into two colour-types, we first attempted to construct color-color
diagrams by using the S-PLUS synthetic photometry of several classes of emission line
objects.\footnote{It is important to note that there are other classes of objects
with emission lines that have not been included. Our main objective is to explore different
populations and/or types of these objects by using their photometric colors.}
The $(g - r)$ versus $(z - g)$ color-color diagram is displayed on
the Fig.~\ref{fig:synthetic}. The SySt span a wide range on the $(z - g)$ color,
  from approximately -0.5 to 6.0. This wide range on the color may refer to the different
  type spectral of the cold stellar component of the binary system. All the YSOs and many
  SySt are located on the top-left on the diagram indicating a reddening effect of the
  circumstellar disk for many SySt (for example, for those symbiotic with a Mira star)
  and YSOs. On the other hand, the PNe, HII regions, CVs, QSOs, and emission line galaxies
  are located on the lower-right region of the diagram. This indicates blue continuum
  present in each classes of theses objects, mainly by the presence of the high excitation
  component, for instance, white dwarf in planetary nebulae and cataclysm variable systems, and
  massive young stars in H II regions and starburst galaxies. Although some SySt are located
  in the region where appear the blue sources, this color-color diagram seems to separate
  very well two color types, blue from red sources. The dashed line in Fig.~\ref{fig:synthetic}
   highlights the blue and red zones.
  
  \begin{figure}
	\includegraphics[width=0.9\linewidth]{Figs/red-blue-colorObjects-gr-edit.jpg}
        \caption{The $(g - r)$ versus $(z - g)$ color-color diagram with all the emission line
        objects selected in S-PLUS. The inset figures represent the $(g - r)$ and  $(z - g)$
        colour distributions.}
    \label{fig:new-color}
\end{figure}

Figure~\ref{fig:new-color} shows the $(g - r)$ versus $(z - g)$ color-color diagram from 
our final list of H{$\alpha$} emitters.
%\sout{We constructed the $(g - r)$ versus $(z - g)$ color-color diagram using 
%our final list of H{$\alpha$} emitters, which is presented in Fig.~\ref{fig:new-color}.}
A bimodal, two-color population is suspected by contrast in the CMD and evidenced in the color distributions (see inset 
plots of the Fig.~\ref{fig:new-color}).
%\sout{As was we expected two-color population is showed in the diagram perceptible in the level of the
%color of the density region on which yellow areas represent a higher concentration of points.
%This is re-forced by the bi-modal shape of the $(g - r)$ and $(z - g)$ color distributions %(see inset 
%plots in Fig.~\ref{fig:new-color}).} 
The two peaks of the $(g - r)$ and $(z - g)$ distributions have a immediate correspondence with the blue and red zones pointed out from the synthetic color-color diagram (Fig.~\ref{fig:synthetic}).
%\sout{clearly correspond to blue and red sources, respectively.} 
This histograms also
show that the fraction of blue objects is considerable higher than the red ones.

\subsubsection{\texttt{HAC}}

The ideal way to choose the number of clusters can be done by displaying the \textbf{dendrogram diagram}.
Firstly, the hierarchical cluster output dendrogram %\sout{(tree; see for instance %Figure~\ref{fig:dendrogram})} 
can be implemented to obtain the desired clustering. Secondly, the dendrogram schema allows 
a convenient way to establish the entity-relationship at all levels of granularity.

Figure~\ref{fig:dendrogram} illustrates the use of a dendrogram based on the $(g - r)$ and $(z - g)$ colours 
of H$\alpha$ emitters.
It highlights 
%\sout{In conclusion, a dendrogram is a visualization in form of a tree showing} 
the order
and distances of groups in  the hierarchical clustering, stopping at 12 nodes:

\begin{itemize}
  
     \item The $x$-axis specifies the %\sout{labels. If it does not specify anything else, they
           %are the indices of the samples. This means, the number of points in nodes/clusters}.
           population in the nodes in a given level of grouping -- that summed up correspond to the total number of elements under investigation.
     \item The $y$-axis represents the %\sout{distances of the ``ward'' method. In other words,
           %this axis} 
           ``distance'', which is a measurement of the closeness of the clusters or data points in different levels of clustering.
       
\end{itemize}

Reading the diagram from the top to the bottom, we see that all systems are divided after the very first level from the top already into (only) two groups: coincidentally, they correspond to the red and blue populations of H$\alpha$ emitters presented in Fig.~\ref{fig:new-color} as it is showed in Fig.~\ref{fig:hierar}. From that point on, the groups were subdivided without evident distinction, and truncation was thus assumed when the 12-node level was reached. The truncation is an usual procedure when dealing with big data. 

%\sout{We first generated the dendrogram diagram to see the possibles clusters that can be %found with
%\texttt{HAC}. The dendrogram diagram sometimes can be pretty big for many samples, especially %in the
%context of the big data and the visualization of the cluster can be difficult.} 

%\sout{However, 
%the \textsc{dendrogram()} function, which is 
%include in the python package \texttt{scipy}\footnote{https://www.scipy.org/}, allows us 
%to generate this diagram, and has a feature to allow a better visualization of them. 
%This is the \textit{Dendrogram Truncation}.} 

%\sout{Figure~\ref{fig:dendrogram} exhibits the truncated dendrogram plot based on the colours
%parameters. This only shows the last 12 nodes out of all merges. Note that most labels are not showing. 
%This is because all other data were already merged into clusters before the last 12 merges. 
%From the diagram is perceptive that the longest distances correspond to 
%the two last merged that are the two clusters that we desire to produce with 
%the hierarchical clustering algorithm.}

\begin{figure}
	\includegraphics[width=0.9\linewidth]{Figs/Customer-Dendrograms.pdf}
    \caption{Truncated dendrogram of complete-linkage hierarchical clustered based on $(g - r)$ and $(z - g)$ colours.
    The cluster sizes are exposed in the brackets for the 12 truncated clusters.}
    \label{fig:dendrogram}
\end{figure}

\begin{figure}
	\includegraphics[width=0.9\linewidth]{Figs/blued-red-hierarchical.pdf}
    \caption{The $(g - r)$ versus $(z - g)$ color-color diagram with the two population
    found by implementing \texttt{HAC} algorithm. The blue and red symbols represent the 
    sources with intense blue continuum and those with intense red continuum, respectively. 
    The straight line is the same line of Figure~\ref{fig:synthetic}.}
    \label{fig:hierar}
\end{figure}

In this work, we implemented {\sc hac} by using the python machine learning library
\texttt{Scikit-learn}\footnote{\url{https://scikit-learn.org/stable/}} \citep{Pedregosa:2011}. Then we use the \textsc{dendrogram()} function, which is 
include in the Python package \texttt{scipy}\footnote{\url{https://www.scipy.org/}}, and the task \textit{Dendrogram Truncation}
to generate the (truncated) dendrogram.
There are  parameters to  be taken into account
when the algorithm is applied to the data: \texttt{n\_clusters}, \texttt{Affinity}, and \texttt{Linkage}.
\texttt{n\_clusters} 
defines the number of clusters expected by the user.
Given that our goal is to divide our sample \out{in} into two groups. we set this  parameter to "2". \texttt{Affinity} 
determine
the "metric" to compute the linkage. 
We have found that a simplistic metric, the ``Euclidean'', is effective for our purpose.
%\sout{We found that a simple ``Euclidean'' distance metric is
%effective.} 
\comment{Eu não entendi essa frase:} \texttt{Linkage} determines which distance to use between sets of observation. \luis{Linkage criterion determines the distance between sets of observations based on the pairwise distances between observations.}
The algorithm merge the pairs of cluster that minimize this criterion. \texttt{Ward's method} is used 
``ward'' \comment{(Quais são as outras opções?)} to minimize the variance of the clusters being merged \comment{(Eu não sei o que é variancia de clusters: variância do que?)}\luis{Mudei a frase}. Ward uses error sum of squares to measurements the variance between the clusters. Then, the two clusters with the smallest error sum of squares will form a new cluster.  As it was mentioned,
the input variables are the colors; ($g - r$) and ($z - g$).

%\sout{Figure.~\ref{fig:dendrogram} shows the dendrogram truncated diagram visualization that
%shows the branching of the sample from the main root. And Fig.~\ref{fig:hierar} shows
%the position of the two groups that resulted after to applying \texttt{HAC} in the
%color-color diagram. The two clusters represent the blue sources (blue simbols)
%and red sources (blue symbols) as already by analyzing the diagram of the
%Fig.~\ref{fig:synthetic}.} 
At this point, 
we have divided our list  into two groups in agreement
with the nature of their continuum. We found that the number of blue H{$\alpha$}
emitters is bigger than the red one.

\subsubsection{\texttt{HDBSCAN}}

\begin{figure*}
\centering
\begin{tabular}{l l}
  \includegraphics[width=0.5\linewidth, trim=10 10 5 8, clip]{Figs/blued-red-hdbscan.pdf}
   \includegraphics[width=0.5\linewidth, trim=10 10 5 8, clip]{Figs/blue-red-hdbscan-soft-alternative.pdf}
  \end{tabular}  
  \caption{As Figure~\ref{fig:synthetic} but with the results after to apply \texttt{HDBSCAN} (\textit{left panel})
  to the sample of H{$\alpha$}. The blue and red symbols correspond to blue and red sources, respectively.
  Gray symbols are the sources classified as noise by \texttt{HDBSCAN}. And the results after apply a soft clustering 
  to the \texttt{HDBSCAN} results (\textit{right panel}). The straight line is the same line as Figure~\ref{fig:synthetic}.}
\label{fig:hdbscan}
\end{figure*}

For the sake of comparison with results from \texttt{HAC}, we also used \texttt{HDBSCAN} to separate the blue sources from the red ones. 
%\sout{This is just for
%comparing the performance of two algorithms and show that the our results are consistent.}
The main difference between these two algorithms is that  \texttt{HDBSCAN}
is more conservative so that many data points are classified as noise.
For this purpose, we used the Python implementation of \texttt{HDBSCAN}\footnote{\url{https://hdbscan.readthedocs.io/en/latest/}}
\citep{McInnes:2017}. Like \texttt{HAC}, in addition to the colors input
parameters,  there are key parameters that should be considered  when the algorithm is
applied. 
Regarding the metric, it is assumed the ``Euclidian''.
%\sout{``Euclidean'' metric is implemented which results to be very efficient.} 
The two most critical parameters 
%\sout{to be implemented} 
are the ``minimum cluster size'' and ``minimum number
of samples''. The former refers to the smallest size grouping that we wish to consider a cluster.
We have adopted the value ``80''. 
\comment{(Eu não entendi o que você quer dizer com a frase seguinte:)} \luis{mudei um pouco a frase.}
The latter provides a measure of how conservative we want 
our clustering to be, expressed as the fraction of data classified as noise.
The value implemented was ``40''. With this configuration 
of our model, 
we found
two 
clusters. 
Several small clusters are found when assuming values for the minimum number of samples smaller than 40.
%\sout{If we use values of minimum number of samples smaller than ``40'' several small cluster %are found.}

Left panel of Fig.~\ref{fig:hdbscan} 
shows 
the two 
clusters found with \texttt{HDBSCAN}
\rlopes{, one containing XX red and the other with YY blue sources}. 
This result is fully consistent with those from \texttt{HAC}: the two primary clusters from \texttt{HDBSCAN} are located in the same region in
the ($g - r$)-($z - g$) diagram where lie the groups found with the \texttt{HAC}.
\comment{(Aqui eu acho que ficaria legal dizer qual é a porcentagem da amostra azul de um método ser identiicada no outromodelo; o mesmo para a vermelha.)}
In fact, by applying the
\texttt{condensed\_tree\_} to the data colors two clusters are selected 
(see Appendix~\ref{sec:trees} for more details about \texttt{condensed\_tree\_} attribute.)
%\sout{(for more detail about
%\texttt{condensed\_tree\_} attribute see appendix~\ref{sec:trees} and related
%Fig.~\ref{fig:treess}).} 
%\sout{The two primary clusters are located in the same region in
%the ($g - r$) versus ($z - g$) color-color diagram where lie the group found by the other %algorithm.}
%\sout{The main difference between the two algorithm is that  \texttt{HDBSCAN}
%is much conservative, then many data points are classified as noise.} 
%\sout{The two final cluster
%contain, which we are labeled blue and red sources contain xxx and xxx objects, respectively.}

\subsubsection{Soft clustering for \texttt{HDBSCAN}}

Perhaps, the main disadvantage of \texttt{HDBSCAN} is that many of the sources are labelled
as ``noise''
so that they are not assigned to any cluster. As mentioned,
this comes from the conservative nature of \texttt{HDBSCAN}
and the fact that these data sources (noise) are located far away of the cluster cores.
An alternative 
way
to avoid 
outliers (noises) classifications consists to use the concept 
of ``soft clustering'' (see section~\ref{sec:hdbscan}). We have carried out here soft
clustering 
from 
\texttt{HDBSCAN}\footnote{\url{https://hdbscan.readthedocs.io/en/latest/soft_clustering_explanation.html}} to assign every object to 
a cluster that they most likely belong to.
With this approach, points are not assigned in a deterministic way to clusters 
but to a vector of probabilities as a measure of belonging to different  clusters: the probability value at the $i$th entry of the vector
is the probability that a data point is a member of the $i$th cluster. 
We can, then, simply
assign cluster labels for every point by taking the most likely cluster it belongs to,
using probability thresholds. Soft clustering
\comment{(Já li essa denominação antes mas agora me pergunto: o que é soft clustering? Indicar aqui uma seção do texto na qual já foi explicado isso?)}
for \texttt{HDBSCAN} is achieved through an outlier score modification to 
consider how much an outlier is relative to each cluster, which is based on the
Global-Local Outlier Score from Hierarchies (GLOSH) algorithm \citep{Campello:2015}.
This is combined with a measure of distance from a given cluster to produce an estimate of the probability that any given data point belongs to any of the fixed groups drawn from the condensed tree. \comment{(Euachei que essa frase ficou muito grande; estou com receio de mudar e alterar o contexto. Veja plsse você consegue uma versão mais clara, talvez dividindo o conteúdo em duas frases)} \luis{Mudei um pouco esta frase, dividí em duas}.

The right panel of Fig.~\ref{fig:hdbscan} shows at what most likely cluster belongs the data
points classified as the noise by \texttt{HDBSCAN}. We have used blue and red colors for the
points that have the highest probability of being in the blue and red groups, respectively.
This fills out the clusters nicely. We see that there were many noise points that are most
likely to belong to the clusters we would expect, e.g. in agreement 
with 
results obtained
with \texttt{HAC}. Indeed, we now have improved our classification of our list into blue and 
red sources because we have estimated the probability of each source to belong
to each group \comment{(Eu não gostei muito dessa frase. Me pareceu deixar algo solto.)} \luis{Adicionei a seguinte frase, talvez feche melhor o paragrafo}. Instead of forcing the algorithm to make a decision as to which group a data point belongs to, just like \texttt{HAC} does, we have quantified the likelihood of a given observation belonging to any of the two clusters found in our data set, see, for instance, the two last columns of Table~\ref{tab:simbad}.  

\section{Results and discussion}
\label{sec:results}

\subsection{Matches with other databases}
\label{sec:matches}
We found a total of 389 emission line sources in all S-PLUS DR3.
To understand the nature of the objects  and the fractional
contribution of different classes of objects with emission lines
to the overall sample cross-matching with some catalogs available were carried out.

\subsubsection{Simbad}

\begin{table*}
\centering
\caption{A summary of the results obtained of the positional cross-match between
        between the S-PLUS list of emission line objects and the SIMBAD database.
        We used a search radios of 2 arcsec. SIMBAD categories of
        objects are listed in the first column. The numbers of objects of each SIMBAD 
        category are exposed in the third column.}
\label{tab:simbad-sources}
\begin{tabular}{lll} % four columns, alignment for each
  \hline
Main type    & Associated SIMBAD           & Number of S-PLUS objects    \\
             & types                        &  with SIMBAD match         \\
\hline
Nebulae                & HII, PN, SN, Candidate\_SN*, Nova  & 37               \\
Cataclysmic variable star  & CataclyV*, Candidate\_CV* & 30              \\
Variable star of RR Lyr type & RRLyr, Candidate\_RRLyr & 19              \\
X-ray source                & HMXB, X                  & 6               \\
Eclipsing binary            & EB*                    & 8                 \\            
Emission object             & EmObj                  & 4                 \\
Star                        & star, WD*, Candidate\_WD*, Blue, BlueSG*, PM*, low-mass* & 57 \\
UV-emission source          & UV                     & 2                 \\
Cluster of stars            & Cl*                    & 3                 \\
Far-infrared source         & FIR                    & 2                 \\
Mid-infrared source         & MIR                    & 1                 \\
Radio-source                & Radio                  & 7                \\
Molecular cloud             & MolCld                 & 2                 \\
Emission line galaxy        & EmG, HII\_G, StarburstG, BlueCompG & 102   \\
Part of a galaxy            & PartofG                & 9                 \\
Interacting galaxies        & IG                     & 10                \\
Galaxy in pair of galaxies      & GinPair            & 12                \\
Galaxy in group of galaxies     & GinGroup           & 18                \\
Galaxy in cluster of galaxies   & GinCl              & 23                \\
Low surface brightness galaxy   & LSB\_G             & 10                \\
Brightest galaxy in a cluster   & BClG               & 3                 \\
Globular cluster            & GlCl                   & 1                 \\
QSO                         & QSO, QSO\_Candidate    & 225               \\
AGN                         & AGN, AGN\_Candidate, Seyfert\_1, Seyfert\_2, BLLac, RadioG & 67\\
Galaxy                      &  Galaxy                & 421               \\
Possible gravitationally lensed image & Possible\_lensImage & 1          \\
\hline
Total                       &                               & 10         \\
\hline
\end{tabular}
\end{table*}

We made cross-match between our sample of objects with excesses of emission on
the $J$0660 filter and SIMBAD database. We searched for all objects using a radius of 2 arcsec
around the position of the optical source in question. We found 1000
matches that include a great variety of emission line objects. In table
\ref{tab:simbad-sources} is showed the different categories of objects found
with SIMBAD coincidence.

\paragraph{Nebulae.}

\begin{figure*}
%\setlength\tabcolsep{\figstampcolsep}
\centering
\begin{tabular}{ll}
  (a) & (b) \\
  \includegraphics[width=0.5\linewidth, trim=10 60 5 8, clip]{Figs/StenholmAcker_pn_g006_0-41_9_id176-SPLUS-s29s46-072842.pdf} \llap{\raisebox{3.7cm}{\includegraphics[width=0.2\linewidth, trim=200 10 300 30, clip]{Figs/PN_316.4731956938531--37.14456181858315_100_r.pdf}}} & \includegraphics[width=0.5\linewidth, trim=10 60 10 8, clip]{Figs/spec-0680-52200-0153-STRIPE82-0159-019049.pdf} \llap{\raisebox{3.7cm}{\includegraphics[width=0.2\linewidth, trim=200 10 300 30, clip]{Figs/HII_351.0847562454346--0.10695578184817373_150_r.pdf}}} \\
(c) & (d) \\
  \includegraphics[width=0.5\linewidth, trim=10 60 10 8, clip]{Figs/spec-0982-52466-0477-STRIPE82-0103-089600.pdf} \llap{\raisebox{3.7cm}{\includegraphics[width=0.2\linewidth, trim=200 10 300 30, clip]{Figs/CV_311.8365168802931-0.0021268467006492_100_r.pdf}}} & \includegraphics[width=0.5\linewidth, trim=10 60 10 8, clip]{Figs/spec-1033-52822-0623-STRIPE82-0134-016856.pdf} \llap{\raisebox{3.7cm}{\includegraphics[width=0.2\linewidth, trim=200 10 300 30, clip]{Figs/SN_333.17320112679323-0.5119642601996512_150_r.pdf}}} 
  \end{tabular}
  \caption{Summary of our selection results showing the spectrum (gray line en each panel) of 
          different classes of emission line sources identified in our target list. 
          A spectrum of a PN (\textit{a}) from  {\sc ref}. The SDSS spectra of an external H II region (\textit{b}), 
          a cataclysm variable star (\textit{c}), a super nova (\textit{d}).
          As in Figure~\ref{fig:Spectra}, coloured square and circles symbols represent the S-PLUS
          photometry. All these objects show a significance excesses on the $J0660$ filter in
          comparison with the broad-bands.}
  \label{fig:photo-image-knonw-objects}
\end{figure*}

\begin{figure*}
%\setlength\tabcolsep{\figstampcolsep}
\centering
\begin{tabular}{ll}
  (a) & (b) \\
  \includegraphics[width=0.5\linewidth, trim=10 60 5 8, clip]{Figs/spec-0399-51817-0140-STRIPE82-0031-029096.pdf} \llap{\raisebox{3.7cm}{\includegraphics[width=0.2\linewidth, trim=200 10 300 30, clip]{Figs/HIIGalaxy_21.69378980441009--0.6457451930656755_150_r.pdf}}} & \includegraphics[width=0.5\linewidth, trim=10 60 10 8, clip]{Figs/spec-0410-51877-0492-STRIPE82-0064-042346.pdf} \llap{\raisebox{3.7cm}{\includegraphics[width=0.2\linewidth, trim=200 10 300 30, clip]{Figs/RadioGalaxy_43.98850261320957-0.692640180601257_150_r.pdf}}} \\
(c) & (d) \\
\includegraphics[width=0.5\linewidth, trim=10 10 10 8, clip]{Figs/spec-0283-51959-0496-SPLUS-n01s20-026220.pdf}  \llap{\raisebox{3.7cm}{\includegraphics[width=0.2\linewidth, trim=200 10 300 30, clip]{Figs/Galaxy_176.35960159169613-0.0041113183584209_150_r.pdf}}} &
  \includegraphics[width=0.5\linewidth, trim=10 10 5 8, clip]{Figs/spec-0981-52435-0575-STRIPE82-0102-040352.pdf} \llap{\shortstack{%
        \includegraphics[width=0.2\linewidth, trim=200 10 300 30, clip]{Figs/Seyfer1_option2_310.91682422695794-0.481551561964735_100_r.pdf}\\
        \rule{0ex}{3.7cm}%
      }
  \rule{4.1cm}{0ex}} \\
  (e) & (f) \\
  \includegraphics[width=0.5\linewidth, trim=10 10 10 8, clip]{Figs/spec-0285-51930-0042-SPLUS-n02s23-042426.pdf} \llap{\raisebox{3.7cm}{\includegraphics[width=0.2\linewidth, trim=200 10 300 30, clip]{Figs/wr1_180.1095757612321--1.1019334954271696_100_r.pdf}}} & \includegraphics[width=0.5\linewidth, trim=10 10 10 8, clip]{Figs/spec-9146-58042-0510-STRIPE82-0131-050516.pdf}  \llap{\raisebox{3.7cm}{\includegraphics[width=0.2\linewidth, trim=200 10 300 30, clip]{Figs/QSO_331.37225331383246--0.5196269942518518_100_r.pdf}}}
  \end{tabular}
  \caption{SDSS spectra of a H II galaxy with $z = 0.006$ (\textit{a}), a radio galaxy with $z = 0.014$ (\textit{b}), 
  a star-forming galaxy with $z = 0.013$(\textit{c}). For this object, the H{$\alpha$} line 
  is responsible for the $J0660$ magnitude. And a Seyfert 1 with $z = 0.317$ (\textit{d}). 
  For this last object, the excess on the $J0660$ is due to the [O III] 4959, 5007 \AA~ emission lines. 
  a WR in a galaxy (\textit{e}) 
          and a QSO (\textit{f}) with red-shift of $\sim$ 2.45.  As in Figure~\ref{fig:photo-image-knonw-objects} coloured 
          symbols indicate the S-PLUS photometry.}
  \label{fig:photo-image-knonw-objects2}
\end{figure*}

As was mentioned, several classes of objects with nebulosity are listed in our sample. 
Those include H II region, planetary nebulae and supernova
remnants (SNRs). Such sources in our Galaxy and nearby galaxies were
identified here. The H II regions are objects with gas that being ionized by
amounts of UV light come from massive stars (OB type) on which
are formed the emission lines. In theses clouds of ionized gas,
new stars are formed. Unlike H II regions, planetary nebulae
represent the final stages of low- and intermediate-mass stars
where the gas previously ejected in the phase of AGB is ionized
by the high energetic radiation come from their central stars.

In our list of H{$\alpha$} emission line objects a PN appears cataloged in SIMBAD 
on which the S-PLUS photometry and SDSS spectrum is displayed on panel (\textit{a}) 
of Fig~\ref{fig:photo-image-knonw-objects}. Emission lines like H{$\alpha$} 
and [N II] are clearly perceptive in its spectrum.
This objects is a very interesting because is one of the twelves PNe that belong
to the Galactic halo. These PNe are low metallizite and present large velocities that
can give count of the origin and nuclesintheis of the early universe. 30 objects 
cataloged as HII regions on the SIMBAD database are in our list of H{$\alpha$} emitters. 
Panel (\textit{b}) oh the figure shows the S-PLUS photometry and SDSS spectrum 
of the extragalactic HII region GALEX 2417063145906373262.  

Supernova and nova are other type of emission lie objects on which shell of gas arise from
evolved stars. The physical nature on the gas of them is different to the typical gaseous 
nebulae (H II regions and planetary nebulae). The general energy-input mechanism is quiet 
different in each case. 
However, the emission-line spectra show general similarities, indicating that when the 
ionized gas is heating tends to produce the same emission-line photons, regardless 
of the mechanisms by which high temperature and ionization are produced. 
In agreement with SIMBAD coincidences, supernovas and novas are 
in our list as show table~\ref{tab:simbad-sources}. Panel \textit{d} of
Figure~\ref{fig:photo-image-knonw-objects} exhibits the SDSS spectrum, S-PLUS 
photometry and coloured image of an extragalatic SN. The emission lines are 
clearly perceptively in its spectrum. 

\paragraph{Binary systems.}
25 known CV and 5 candidate CV were selected with the algorithm.
CV are binary systems of very short orbital period, in which a low-mass and
early-type star fills its Roche region and transfers mass to a companion stars,
a dwarf white \citep{Patterson:1984}. Fig.~\ref{fig:photo-image-knonw-objects} 
shows the S-PLUS photometry overlapped to the SDSS spectrum of the CV FASTT 1560. 
As expected, this object was classified as blue source by the machine learning approaches.
Other binary system in our list are the X-binary sources and eclipsing binary. 
That type of object is expected to be in our sample because majority of them
are known to be H{$\alpha$} emitters.

\paragraph{Stars.}
Several stars were selected as shows the  SIMBAD matches.
The SIMBAD types include normal stars, white dwarf stars (WD*), white dwarf
candidates (Candidate\_WD*), blue stars, blue super-giant stars (BlueSG*), high
proper-motion stars (PM*) and low-mass star (low-mass*; M<1M$_\odot$). All these
objects appears in the S-PLUS catalog as H{$\alpha$} emitters. Some of them 
can be early/late-type emission-line stars and/or different types of stars with
the H{$\alpha$} emission lines. Although some of these stars could represent the 
fraction of contaminants of our sample. There are also matches to variable
stars of RR Lyr type as well as candidates of them.

\paragraph{Galaxies.}
Emission line galaxies also were selected. This category includes
several class of galaxies according to SIMBAD: emission-line galaxy (EmG),
blue compact galaxy (BlueCompG), H II galaxy (HII G), galaxy in cluster of galaxies
(GinCl), galaxy in group of galaxies (GinGroup), low surface brightness galaxy
(LSB G), interacting galaxies (IG), part of a Galaxy (PartofG), Seyfert 1 and
2 galaxies, AGN.

Many of the emission line galaxies of these category form part of the universe local.
This mean that their reds-hift cover the range on which the H{$\alpha$} emission line still
falls into the $J0660$ filter. Objects with $ z >  0.02$ the  H{$\alpha$} is outside 
of the $J0660$ filter. Table~\ref{tab:simbad} of the appendices shows the red-shift of the
emission line galaxies of our sample that have SIMBAD matches. Many of the galaxies have 
small red-shift values probing that mostly of these galaxies are real H{$\alpha$} emitters. 
In these kind of galaxies such as the blue 
compact and/or H II galaxies their starburst regions that consist of ensemble of massive 
ionizing stars and their respective giant H II regions cover the extension of the optical 
images and the emission lines of their spectra. Fig.~\ref{fig:photo-image-knonw-objects2}
shows the SDSS spectra and S-PLUS photometry of the H II galaxy 6dFGS gJ012646.5-003845 
(panel \textit{a}) and of the radio galaxy NVSS J025557+004132 (panel \textit{b}). 
The H II galaxy is classified  as blue source while the radio galaxy as red by the 
machine learning algorithm which is obvious from their spectra. 

On the other hand, several AGN, Seyfert 1 and Seyfert 2 galaxies have red-shift 
between 0.306 and 0.376 indicating that the excess on the $J0660$ filter
is not due to the H{$\alpha$} emission line but if due to the H{$\beta$} 
and [O III] 4959, 5007 \AA~ emission lines. %These emission lines at the red-shift range $0.306 < z < 0.376$ are detected 
%in the $J0660$ band. 
It is well known that the AGNs have very strong emission 
lines such as H{$\beta$}, [0 III] 4959, 5007 \AA~and H{$\alpha$}. 

One interesting discussion here, is that the SIMBAD type part of a Galaxy (PartofG)
 are actually galaxies with Wolf-Rayet (WR) signature in the low red shift
universe also named ``WR galaxy'' \citep{Osterbrock:1982}. WR stars in galaxies
is perceptible in their spectra which have strong emission lines such as H{$\alpha$}
and [NII]. These spectral features can be of them very similar to extragalactic
H II regions present typically of the outskirt of spiral galaxies. Panel (\textit{e}) 
of Fig.~\ref{fig:photo-image-knonw-objects} displays the S-PLUS and SDSS spectra 
of the galaxy with Wolf-Rayet signature [BKD2008] WR 14.

Some objects have simply as main type ``galaxy'' in SIMBAD.
This galaxies could be spiral galaxies on which the star formation is present. 
It is known that in the spiral arms of the galaxies are present H II regions 
as well as neutral gas (H I) and molecular gas. Actually some of these galaxies 
haves for secondary type H I. The reservoir of H I gas in galaxies must ultimately
feed their star formation, after cooling and forming molecular
clouds \citep{van-Driel:2016}. Panel \textit{c} of Figure~\ref{fig:photo-image-knonw-objects2} 
shows the S-PLUS photometry and the SDSS spectra of a star-forming galaxy. The SDSS spectrum 
clearly exhibits strong emission lines. Note that almost all these galaxies are 
classified as bluer objects by \texttt{HAC} and \texttt{HDBSCAN}.
However, mostly of the Seyfert 2 and radio-galaxies and an handful of 
galaxies are classified as red sources. 
These galaxies, belonging to the red sub-sample, are probably nearby, 
red early-type galaxies with emission lines, mainly, they could be LINERS, 
AGB or radio galaxies \citep{Capetti:2011}.

\paragraph{QSOs.}

Other extragalactic objects with emission lines that were selected are the
QSOs. In the case of the QSOs is not the H{$\alpha$} emission line that affects
 the $J$0660 filter. This filter is affected by other strong emission lines present in
these objects that to specific redshift drop into the $J$0660 filter.
Some of these emission lines are H{$\beta$}, C {\sc IV} 1550 \AA, C {\sc III]} 1909 \AA,
  Mg {\sc II} 2798 \AA~(see, also \citealp{Gutierrez:2020, Nakazono:2021}).
  All the objects were classified as blue sources which was expected. Panel \textit{f} of
  Figure~\ref{fig:photo-image-knonw-objects2} show the S-PLUS photomerty and SDSS 
  spectrum of the QSO 2SLAQ J220529.34-003110.6
  which has red-shift of $\sim$2.45 indicating that the emission line corresponds to
  the line C {\sc III]}.
  
\paragraph{Other type of objects.}
  
  Miscellanies the objects were selected as H{$\alpha$} emitters that appear classified on
  SIMBAD as it can see in the table~\ref{tab:simbad-sources}. They are included cluster of stars, 
  FIR sources, MIR objects, molecular clouds, UV-emission source, among other.

\subsubsection{SDSS and LAMOST}

We also made cross-match between the sample of H{$\alpha$} emission line objects and SDSS DR16
\citep{Ahumada:2020}. We have used a 2$\sigma$ error circle as the cross-matching radius.
We got 200 spectra from which 195 objects exhibit strong emission lines. 5 objects do not 
show emission lines, they probably are stars or galaxies.

We also cross-correlate our list with Large Sky Area Multi-Object Fiber Spectroscopic Telescope
(LAMOST; \citealp{Wu:2011}) using a radius of 2 arcsec for the match.  These spectra also
show strong emission lines. These results shows that this technique actually are very effective
for selecting sources with strong emission lines.

Comparing with These spectra mainly correspond to HII regions, cataclysm variables, SN, emission-line galaxies 
(blue compact galaxies, H II galaxies, star-forming galaxies, among others), AGN (Seyfert 1 and 2), and QSOs. 
However, more detailed analyzes are necessary to see that other possibles types 
of objects are included in these samples of spectra.

\subsection{Magnitudes and color distributions}

\begin{figure}
  \begin{tabular}{l l l}
  \includegraphics[width=\columnwidth]{Figs/distribution_r-group.pdf} \\
    \includegraphics[width=\columnwidth]{Figs/distribution-ri-group.pdf}\\
    \includegraphics[width=\columnwidth]{Figs/distribution-Halpha-group.pdf}
  \end{tabular}
  \caption{Distribution of $r$ magnitude  (\textit{top panel}), $(r - i)$ color (\textit{middle panel}), and $(r - J0660)$
    color (\textit{bottom panel}) for the blue and red sources of the sample of H$\alpha$ emitters.
    Both sample are normalized to their maximum counts, xx and xx objects for blue and
    red sources, respectively. The smooth curves represent a Kernel density estimation for both samples.}
    \label{fig:diagram-distri}
\end{figure}

Figure~\ref{fig:diagram-distri} shows the magnitude and color distribution of the blue and red 
sources --classification performed with machine learning on Section~\ref{sec:clustering}--.
Upper panel of the figure shows the distribution of the blue and red sources of the $r$-magnitude.
The peaks of the $r$ distribution is the same for the blue and red sources, which is around
$r \sim 19.9$. However, the density at the peak is higher in red sources than the blue ones
indicating that the proportion of red objects with $r$-value of 19.9 is higher than the blue sources. 
The fraction of object is very small at $r < 16$ for each group of objects.
The distribution of the objects for both group increase in the range $16 \gtrsim r \gtrsim
19.9$. However, the density as indicating the Kernel density estimation curves is higher in 
the blue sources. This implies that the fraction of sources with the magnitude range 
is considerable higher in comparison with the red group. In conclusion much blue sources 
tends to be brighter than the red ones.

Middle panel of figure displays the ($r - i$) distribution of the blue and red emission line
objects. The peaks of the $(r - i)$ distributions are distinct for the blue and red sources.
The peaks of the red sources at high red continuum,  $(r - i) = 0.5$, compared to the value
of the blue sample, $(r - i) = -0.9$. All these results are consistent because the $(r - i)$
color is an indicating of reddened sources. It is also consistent with previous works. 
For instance, \citet{Wevers:2017} used different algorithms based on the  $(r - i)$ color
to successfully select blue outliers from the the Galactic Bulge 
Survey (GBS; \citealp{Jonker:2011}).

Bottom panel of Figure~\ref{fig:diagram-distri} shows the
$(r - J0660)$ color distribution of the blue and red objects. The fraction of objects selected
as emitter rises drastically with $J$0660 excess, until at sufficiently large excesses. 
The peaks of the $(r - J0660)$ color distribution are relatively different for both groups 
of sources. Having the blue sources the peaks at  $(r - J0660) = 0.5$ while for the red 
ones the value peak is $(r - J0660) = 0.7$.

\section{Conclusions}
\label{sec:conclu}

We have created and analyzed a sample of emission-line objects in the 
local universe selected from the S-PLUS catalog.
By identifying the locus of main-sequence and giant star on the ($r - J0660$) versus ($r - i$) 
colour-colour diagram and considering objects 
with H$\alpha$-excess, those located above of this locus. 
The sample contains 9,000 sources that were identified as H{$\alpha$} emitters.
The big advantage here compared to previous work is that we are providing a significant 
sample of H $\alpha$ emitters in 12 bands. Seven and five narrow- and broad-band filters,
respectively, covering the wavelength range from $\sim$3000 to $\sim$9000\AA. 
In other words, much more information can be extracted from them in future work only using S-PLUS photometry. 

Match with spectroscopic databases (SDSS and Lamost) evince that mostly of the objects 
selected are emission line objects, showing  that at least the XX\% of the objects are 
genuine H$\alpha$ emitters. 

The ($g - r$) and ($(z - g$) colour distributions of the H{$\alpha$} emitters are bimodal, 
evidencing two populations. One peak corresponding to blue sources and the another peak to the 
red sources. Agreement to this results we explore the ($g - r$) versus ($z - g$) color-color diagram 
to separate the blue sources from the red, finding that it can be used for this task. But there is a zone 
on the diagram on which the blue and red objects are overlapped. 
For this reason, we use two types of unsupervised machine learning to group our sample of H{$\alpha$} 
emitters into two groups by color-type. \texttt{HAC} and \texttt{HDBSCAN} clustering algorithms were 
used to distinguish the blue sources from the red. The two approaches found the same two clusters, 
but given \texttt{HDBSCAN} is much conservative algorithm than  \texttt{HAC}
many of the objects were labeled as noise. To solve that we also used a soft clustering approach 
for \texttt{HDBSCAN} to estimate the probabilities of each data point belongs to (blue or red group). 
Finding more consistent results between \texttt{HAC} and \texttt{HDBSCAN}. We argue that
the ($g - r$) and ($(z - g$) colors are useful parameters to classify objects into color-types,
with hierarchical soft clustering of the colours features of a given sample providing a 
convenient means of sorting and classifying sources in the colour parameter space. 

The bluer objects corresponds mainly to  CV, QSOs, PNe, dwarf compact galaxies, H II regions, 
among others. The redder sources are early type galaxies with emission lines (for instances, 
radio-galaxies and Seyfert 2 galaxies), probably young/active late-type stars or 
even symbiotic stars. 

\section*{Acknowledgements}

\rlopes{RLO acknowledges financial support from the Brazilian institutions CNPq (PQ-312705/2020-4) and FAPESP (\#2020/00457-4).}

The S-PLUS project, including the T80-South robotic telescope and
the S-PLUS scientific survey, was founded as a partnership between the
Fundação de Amparo à Pesquisa do Estado de S\~{a}o Paulo
(FAPESP), the Observatório Nacional (ON), the Federal University of
Sergipe (UFS), and the Federal University of Santa Catarina
(UFSC), with important financial and practical contributions from
other collaborating institutes in Brazil, Chile (Universidad de La
Serena), and Spain (Centro de Estudios de Física del Cosmos de
Aragón, CEFCA). We further acknowledge financial support from
the São Paulo Research Foundation (FAPESP), the Brazilian National
Research Council (CNPq), the Coordination for the Improvement of
Higher Education Personnel (CAPES), the Carlos Chagas Filho Rio
de Janeiro State Research Foundation (FAPERJ), and the Brazilian
Innovation Agency (FINEP).

Funding for the SDSS and SDSS-II has been provided by the Alfred P.
Sloan Foundation, the Participating Institutions, the National Science
Foundation, the U.S. Department of Energy, the National Aeronautics
and Space Administration, the Japanese Monbukagakusho, the Max
Planck Society, and the Higher Education Funding Council for England.
The SDSS Web Site is \url{http://www.sdss.org/}.

The SDSS is managed by the Astrophysical Research Consortium for
the Participating Institutions. The Participating Institutions
are the American Museum of Natural History, Astrophysical Institute Potsdam,
University of Basel, University of Cambridge, Case Western Reserve University,
University of Chicago, Drexel University, Fermilab, the Institute for Advanced
Study, the Japan Participation Group, Johns Hopkins University, the Joint
Institute for Nuclear Astrophysics, the Kavli Institute for Particle Astrophysics
and Cosmology, the Korean Scientist Group, the Chinese Academy of Sciences (LAMOST),
Los Alamos National Laboratory, the Max-Planck-Institute for Astronomy (MPIA),
the Max-Planck-Institute for Astrophysics (MPA), New Mexico State University,
Ohio State University, University of Pittsburgh, University of Portsmouth,
Princeton University, the United States Naval Observatory, and the University
of Washington.

Guoshoujing Telescope (the Large Sky Area Multi-Object Fiber Spectroscopic
Telescope LAMOST) is a National Major Scientific Project built by the Chinese
Academy of Sciences. Funding for the project has been provided by the National
Development and Reform Commission. LAMOST is operated and managed by the
National Astronomical Observatories, Chinese Academy of Sciences. 

Scientific software
and databases used in this work include 
TOPCAT\footnote{\url{http://www.star.bristol.ac.uk/~mbt/topcat/}} \citep{Taylor:2005}, 
simbad and vizier from Strasbourg Astronomical Data Center (CDS)\footnote{\url{https://cds.u-strasbg.fr/}} 
and the following  python packages: numpy, astropy, matplotlib, seaborn, scikit-learn.

%%%%%%%%%%%%%%%%%%%%%%%%%%%%%%%%%%%%%%%%%%%%%%%%%%
\section*{Data Availability}


%%%%%%%%%%%%%%%%%%%% REFERENCES %%%%%%%%%%%%%%%%%%

% The best way to enter references is to use BibTeX:

\bibliographystyle{mnras}
\bibliography{ref} 

%%%%%%%%%%%%%%%%%%%%%%%%%%%%%%%%%%%%%%%%%%%%%%%%%%

%%%%%%%%%%%%%%%%% APPENDICES %%%%%%%%%%%%%%%%%%%%%

\clearpage
\appendix
\section{Condensed Trees}
\label{sec:trees}

The condensed Trees is a diagram for \texttt{HDBSCAN} that allows to see the cluster 
hierarchy as a dendrogram.  It can be displayed via the \texttt{condensed\_tree\_}
attribute of the \texttt{HDBSCAN} package. Figure~\ref{fig:condensed-trees} shows
the condensed trees which was obtained by using the ($r - g$) and ($g - z$) colours as the
the input parameters. It is possible to see that \texttt{HDBSCAN} has found
two clusters in agreement with previous results. This means that they represent the blue 
and red sources.  

\begin{figure*}
	\includegraphics[width=0.9\linewidth]{Figs/cluster-hierarchy-hdbscan.pdf}
        \caption{The condensed Trees for our sample of H{$\alpha$} emitters. 
        The width and color of each branch represent the number of points in the cluster at that level.
        The orange and blue ellipses represent the branches selected by the \texttt{HDBSCAN} algorithm.}
    \label{fig:condensed-trees}
\end{figure*}

\newcommand\TableHeader{
  \hline\hline
  Id Object & \(\mathrm{RA}\) & \(\mathrm{Dec}\) & Type & Redshift & Group & P(Blue) &  P(Red)\\
            &                 &                  &      &          &{\sc hac}& {\sc hdbscan}& {\sc hdbscan} \\
  \hline 
}

%\clearpage
\section{Simbad objects}

\begin{center}
\onecolumn
\begin{longtable}{l r r c c c c c}
 \caption{Objects from the SIMBAD data base. The first column presents the ID SIMBAD of the source in question. 
Right ascension and declination are shown in the second and third columns, respectively. 
The type is given in the fourth column. The red-shift, if exist in SIMBAD, is displayed in 
the fifth column. The colour-type classification performed with \textsc{HAC} algorithm is 
presented in the sixth column. The seventh and eighth columns show the probability estimated 
from \textsc{hdbscan} approach to being a blue and red source, respectively.. \label{tab:simbad}}\\
 \TableHeader\endfirsthead 
 \caption[]{--continued}\\
 \TableHeader\endhead
 \hline \endfoot
 %% \begin{table}
%% \begin{tabular}{cccc}
%% main_type & RA & DEC & Label_hier \\
Star & 0.4968921619628843 & -29.3112214428821 & Blue \\
QSO & 0.6279554824066291 & 0.8331230804411939 & Blue \\
ClG & 0.6982473212090481 & -0.373204977953541 & - \\
Star & 1.286523747867553 & -30.851162659705377 & Red \\
QSO & 1.658281373282977 & -0.6156060790712984 & - \\
QSO & 1.791687271704504 & 0.8914261244525057 & Blue \\
QSO & 2.038930048182944 & 0.8264638152844974 & - \\
Galaxy & 2.328179105177891 & -0.6519420575037337 & Blue \\
QSO & 2.666987531261032 & -29.740921877458867 & Blue \\
Star & 2.7030486187917577 & -29.791327943410987 & Red \\
Galaxy & 2.7306881758399904 & -30.739852073510534 & Blue \\
QSO & 3.1199364195988752 & -31.04443153319788 & Blue \\
Seyfert 1 & 3.3638162881857334 & 0.875614770300196 & Blue \\
EmG & 3.6199743565125737 & -0.7455029176005269 & Blue \\
Star & 3.7333117402685474 & 0.3176892728434901 & Blue \\
QSO & 3.860510825587982 & 0.3037236200425542 & Blue \\
QSO & 3.898110576074378 & 0.8989176411595012 & Blue \\
Galaxy & 4.117681731557023 & 1.1338942388667368 & Blue \\
QSO & 4.174451424491121 & -31.44905004903269 & Blue \\
QSO & 4.380258459418438 & -0.8164386494043362 & Blue \\
StarburstG & 4.416543859611166 & 0.5062627804218517 & Blue \\
QSO & 4.474256625814649 & 0.8493703288672522 & - \\
QSO & 4.801640792188885 & 0.0554916881824542 & Blue \\
QSO & 4.917646791908225 & -0.9099503373848068 & Blue \\
QSO & 4.958578954314056 & -0.6779700699260857 & - \\
QSO & 5.657902318221463 & 0.0886788337120439 & Blue \\
EmObj & 6.280815147745145 & 0.3125654118801338 & Blue \\
Seyfert 1 & 6.333011956389121 & 0.525479973176037 & Blue \\
Galaxy & 6.974318052697582 & -0.9667199252402 & Blue \\
Galaxy & 7.320040644293336 & -1.0064266381068807 & Blue \\
QSO & 7.416749471161749 & 1.0913027872498264 & Blue \\
AGN & 7.464354807727703 & 0.7000037288417243 & Red \\
QSO & 7.823701945596341 & 0.2847385223835051 & - \\
Unknown & 7.906256073243137 & -29.47091029710251 & Blue \\
Galaxy & 7.961265608895669 & -28.92685732854085 & Blue \\
Galaxy & 7.969005194202078 & -29.592587811984853 & Blue \\
QSO & 8.035561688713287 & -0.8843461346218345 & Blue \\
QSO & 8.144241522878199 & -0.2658504803749659 & Blue \\
Galaxy & 8.144555784075553 & -42.66955742286215 & Blue \\
QSO & 8.178072183921541 & 0.5197450140043085 & Blue \\
Galaxy & 8.477954545229275 & -29.936853996206867 & Blue \\
AGN & 8.821276175970617 & -42.088597877208215 & Red \\
QSO & 8.941080203860544 & 0.3849934246315855 & Blue \\
AGN & 9.10579672267925 & -0.4853030603908992 & Red \\
Galaxy & 9.160164736675396 & -32.5790767643802 & Blue \\
QSO & 9.308785545285224 & -0.9344374247364892 & - \\
QSO & 9.342361420176386 & -0.1946009608261733 & Blue \\
EmObj & 9.4213937498701 & 0.5555554739851708 & Blue \\
QSO & 9.747289111980264 & -0.714454350224955 & Blue \\
BlueCompG & 9.876175084523542 & 1.3391251468348433 & Blue \\
FIR & 9.89493608957293 & 0.8602458369384121 & Blue \\
FIR & 9.89493608957293 & 0.8602458369384121 & Blue \\
Galaxy & 10.565064891250165 & -32.21589781391012 & Blue \\
QSO & 10.682789961127495 & 1.283933341653539 & Blue \\
Galaxy & 10.84118200857788 & -33.31747498120273 & Blue \\
CataclyV* & 10.896509451927638 & -0.6248809118328724 & Blue \\
QSO & 11.065875879315966 & -0.717518646194803 & - \\
QSO & 11.434797546075147 & -31.95811304109312 & Blue \\
Galaxy & 11.609577132522691 & -1.237997024045413 & Blue \\
QSO & 12.112387692202818 & -34.22741726899439 & Blue \\
QSO & 12.32716144605164 & 1.2191920530524625 & Blue \\
Seyfert 1 & 12.955915679701484 & 0.5649407893753414 & Blue \\
QSO & 12.98183382570476 & -26.962028288894874 & Blue \\
PartofG & 12.99884215031428 & -0.489110520414667 & Blue \\
Galaxy & 13.215039941128667 & -27.32576442209961 & Blue \\
QSO & 13.43231563480846 & 1.363190953245345 & Blue \\
Galaxy & 13.564745864622545 & -1.0822213962934328 & Blue \\
EmG & 13.667158618404784 & -32.011715009609304 & Blue \\
QSO & 13.683104869130515 & -30.51502373758133 & Blue \\
QSO & 13.883649917774902 & -31.260498014964096 & Blue \\
Galaxy & 13.922147889436662 & -0.9418439978440708 & Blue \\
UV & 13.96395390310537 & -30.945219315827195 & Blue \\
Star & 13.97147997409192 & -28.91591361083288 & Blue \\
Galaxy & 13.97214405926334 & -33.65042770377668 & Blue \\
QSO & 14.04137683954713 & -31.36906110248861 & Blue \\
QSO & 14.162698397718229 & -31.96627152132016 & Blue \\
Galaxy & 14.30251703197933 & -0.3660285011264472 & Blue \\
QSO & 14.668422466041417 & -30.03336355079838 & Blue \\
GinCl & 14.767074131559603 & 1.0011685238164214 & Blue \\
Galaxy & 14.806556179582149 & -34.32102363637064 & Blue \\
Galaxy & 14.90035476839902 & -30.344152784983983 & Red \\
QSO & 14.953354509073169 & -1.3181152771969302 & Blue \\
QSO & 14.971706058633467 & -39.53259262490825 & Blue \\
GinCl & 14.99825027722706 & -0.8658482234469952 & Red \\
Star & 15.018499579744676 & -33.65902649700721 & Blue \\
Unknown & 15.041426928322918 & -32.02530671766837 & Blue \\
Galaxy & 15.067378592944028 & -34.96127978456718 & Blue \\
Galaxy & 15.340682319480235 & -0.050506307384101 & Blue \\
QSO & 15.561054131595537 & -30.13160407774772 & Blue \\
Seyfert 1 & 15.625098200551925 & -0.5352093094283729 & Blue \\
GinCl & 15.632402188184084 & 1.3433615405781432 & Blue \\
QSO & 15.901621012754958 & -0.9191165595206996 & Blue \\
QSO & 16.05775922175709 & -1.264431931617225 & Blue \\
QSO & 16.580131544341818 & 0.8064950808416276 & Red \\
AGN & 16.722666519750696 & -32.72831437441763 & Blue \\
AGN & 16.74555738353547 & 1.0772660282004771 & Red \\
QSO & 16.77312563028888 & 0.1024885367041135 & Blue \\
LSB G & 16.943641045715655 & 1.0639605673889347 & Blue \\
BClG & 16.952571807415076 & 0.7482365853106138 & Red \\
GinCl & 17.256603326003162 & 1.3781935749384482 & Blue \\
GinCl & 17.256603326003162 & 1.3781935749384482 & Blue \\
QSO & 17.28164156677598 & 0.1138851962867193 & Blue \\
HII G & 17.28310789223492 & 1.12097992550518 & Blue \\
QSO & 17.32733457723796 & 0.90539845012828 & Blue \\
QSO & 17.35815775854618 & -0.6275003706886488 & Blue \\
EmG & 17.558224802220092 & -30.4123526694582 & Blue \\
EmG & 17.829265187901036 & -30.00504645851688 & Blue \\
QSO & 17.86813658313074 & 0.0286974483465913 & Blue \\
Galaxy & 18.052665880956955 & -33.94196296129968 & Red \\
QSO & 18.12729657912228 & 0.2449304176402998 & Blue \\
Star & 18.24172293238632 & 0.9769381424794108 & Blue \\
Galaxy & 18.304286168638374 & -32.436079929066175 & Blue \\
LSB G & 18.418416425105388 & 0.8774947989896621 & - \\
Seyfert 1 & 18.509796280365048 & -0.7974488232759221 & Blue \\
QSO & 18.521860772867413 & -31.150780875277498 & Blue \\
Galaxy & 18.65047255772875 & -32.644711309144114 & Blue \\
HII G & 18.882863357422373 & -0.8622960010814391 & Blue \\
Galaxy & 18.887732613726776 & -0.8596562230501371 & Blue \\
HII G & 18.89077699675322 & -0.8587455347938298 & Blue \\
HII G & 18.89077699675322 & -0.8587455347938298 & Blue \\
QSO & 18.92575064401607 & 0.3834537397700357 & Blue \\
Galaxy & 19.15999417596711 & -32.92752270653384 & Blue \\
Galaxy & 19.417523978322624 & -33.07795844094275 & Blue \\
Galaxy & 19.485155478290817 & -30.44051337374056 & Blue \\
Galaxy & 19.523785998210776 & -33.05253885507861 & Blue \\
QSO & 19.57552004108609 & 0.2487480482840988 & Blue \\
Seyfert 1 & 19.623433609428748 & 0.7637166618270906 & Blue \\
Galaxy & 19.70478983753148 & -33.33696862040094 & Blue \\
EmG & 19.97594947041034 & -34.24999710931423 & Blue \\
Galaxy & 20.04161862712166 & -33.23630428587631 & Blue \\
QSO & 20.29473653398262 & -0.8436434200227299 & Blue \\
Star & 20.46746328553068 & -33.937725015878605 & Red \\
HII G & 20.55777706221541 & 0.9587780047376696 & Blue \\
Galaxy & 20.571187870663927 & -34.04488260918208 & Blue \\
QSO & 20.611445353426262 & 0.0577366753714225 & Blue \\
QSO & 20.75741214062869 & 0.0565525553589094 & Blue \\
Galaxy & 20.9619712625909 & -29.19623005164965 & Blue \\
GinCl & 20.97812192371461 & 0.2823414770093598 & Blue \\
Galaxy & 20.984790111603804 & 0.2086134560133449 & Blue \\
Galaxy & 20.989463788795724 & -33.80208215339882 & Blue \\
Galaxy & 21.02387513879961 & 0.984703935513288 & - \\
QSO & 21.067433955055776 & -32.20603385094096 & Blue \\
Galaxy & 21.12566535772545 & -33.645961451411665 & Red \\
QSO & 21.26913199118856 & -32.28740231179497 & Blue \\
EmG & 21.3593347521788 & -30.742435844021443 & Blue \\
Galaxy & 21.45538057489708 & -28.162097768619763 & - \\
Galaxy & 21.612608797820563 & 0.9810769034670448 & Blue \\
Galaxy & 21.657201901148973 & -34.58716304788603 & Blue \\
HII G & 21.69378980441009 & -0.6457451930656755 & Blue \\
GinCl & 21.99710751467788 & -29.086653030427502 & Blue \\
GinCl & 22.36059472065752 & -1.1997118023036937 & Red \\
QSO & 22.642393025135533 & -0.3518162354106257 & Blue \\
Galaxy & 22.84101560993188 & -33.101715461339545 & Blue \\
Galaxy & 22.940224488235035 & -32.94910608953699 & Blue \\
Galaxy & 22.94684892850287 & -33.18197514224045 & Blue \\
Galaxy & 23.22263487815018 & -33.4451915146773 & Blue \\
low-mass* & 23.268819219724307 & 0.0655786809704273 & Red \\
Galaxy & 23.50191659447233 & -1.0664487372750466 & Blue \\
Galaxy & 23.716830840704777 & -0.6486782673628352 & Blue \\
QSO & 23.753457358462345 & -0.6817256241073724 & Blue \\
Seyfert 1 & 23.823057046706182 & -0.3275050653110525 & Blue \\
Galaxy & 23.8798401596094 & -31.614165292143262 & Blue \\
QSO & 24.257111717644538 & -1.3497465276851617 & Blue \\
QSO & 24.3725070057774 & -32.12103455786598 & Blue \\
QSO & 24.655329021886622 & 0.4717925959967649 & Blue \\
QSO & 24.962782843879992 & 0.4272217892643302 & Blue \\
Seyfert 1 & 25.07109927100977 & -0.8341590148622415 & Blue \\
QSO & 25.35682525075279 & 0.1321647769641862 & Red \\
QSO & 25.60303358518956 & -32.07047314351019 & Blue \\
QSO & 25.764557663482428 & -29.88188949191848 & Blue \\
EmG & 25.82619164196742 & -34.206232083806285 & Red \\
QSO & 26.83799104587179 & -0.7514735524191466 & Red \\
QSO & 26.913386215352126 & -28.883097657544933 & Blue \\
QSO & 26.9133952038632 & -28.88312791748778 & Blue \\
Galaxy & 27.02603800602231 & -0.4782147764500624 & Blue \\
QSO & 27.05101817682357 & 0.0315348188803727 & Blue \\
Star & 27.18405084268284 & -27.936387555535745 & Blue \\
Galaxy & 27.320953851728728 & -32.74253858762196 & Blue \\
QSO & 27.33969542295298 & -0.5391525524002503 & Blue \\
Galaxy & 28.165578926823763 & 1.0994164155371124 & Red \\
Galaxy & 28.224387503312013 & 1.2043017183289575 & Red \\
Seyfert 1 & 28.24067423260991 & -28.810501310105817 & Blue \\
QSO & 28.38271202195532 & 0.3813754396509161 & Blue \\
HII G & 28.502069390098665 & -0.752764335854246 & Blue \\
QSO & 28.53861595248377 & 0.4459129033998668 & Blue \\
QSO & 28.545594305017943 & -28.870735753209058 & Blue \\
QSO & 28.564494450701485 & -28.8819264781154 & Blue \\
EmG & 28.66852283441473 & -0.1121248724133715 & Blue \\
Galaxy & 28.86195614598901 & 0.1043810217993179 & Blue \\
Galaxy & 28.871127003162204 & -0.6575329793880781 & Blue \\
RRLyr & 29.55730413894375 & 1.028736096882655 & Blue \\
QSO & 29.63394027051608 & -30.284120909894927 & Blue \\
QSO & 29.63398878244968 & -30.28409277206394 & Blue \\
QSO & 29.70926847240945 & -30.077239118059776 & - \\
QSO & 29.89784479360552 & 0.0670730705671857 & Blue \\
QSO & 30.105854148548683 & 0.4879895823228443 & Red \\
QSO & 30.22926062206253 & -29.59069575111854 & Blue \\
GinGroup & 30.31036584086444 & -31.728583712955565 & Blue \\
QSO & 30.31471271654344 & 0.5264157435180954 & Blue \\
QSO & 30.500247645814035 & -0.155885605135301 & Blue \\
QSO & 31.14774462904211 & -45.99000661095606 & Blue \\
Galaxy & 31.25347770425113 & 1.401015050803151 & Blue \\
Galaxy & 31.83471909535146 & -33.03174281330757 & - \\
QSO & 32.01870359151427 & -0.0063899756832095 & Blue \\
QSO & 32.11278143734105 & -0.8688706450161886 & Blue \\
QSO & 32.34166379711758 & -0.9153964683450352 & Blue \\
Galaxy & 33.10437995801359 & -33.083042512325804 & - \\
Galaxy & 33.60091505746918 & -33.24777452929094 & Blue \\
Galaxy & 33.69844878804215 & -32.709786589316614 & Blue \\
QSO & 33.87092614743144 & -0.8874609989930496 & Blue \\
Galaxy & 34.006750737249284 & -31.61438842877632 & Blue \\
Galaxy & 34.05728735762157 & -30.84910309956719 & Blue \\
QSO & 34.07162177605453 & -1.17963313592842 & Blue \\
QSO & 34.543835265892945 & -1.0297901208560476 & Blue \\
CataclyV* & 34.866661573073486 & -30.76277662129932 & Blue \\
EmG & 35.3022376237971 & -30.43306611004649 & Blue \\
Galaxy & 35.82052533020276 & -1.0137779997035794 & Blue \\
Seyfert 1 & 36.07140219729569 & 0.10724100290268 & - \\
Galaxy & 36.21975140764737 & -34.10951561423316 & Blue \\
Galaxy & 36.37884288953809 & -0.8353138394540365 & Blue \\
PartofG & 36.61785108970139 & 1.1604400554801353 & Blue \\
PartofG & 36.61785108970139 & 1.1604400554801353 & Blue \\
Galaxy & 36.69277213609015 & -43.59156135457005 & Blue \\
EmG & 36.81029852296449 & 1.0934074799976583 & Blue \\
Galaxy & 36.83039522616293 & 1.025612852538597 & Blue \\
QSO & 36.90951623742346 & -31.607345845404307 & Blue \\
QSO & 36.99250641431996 & 0.0404421118141437 & - \\
Galaxy & 37.06050668375983 & -39.267782818411305 & Blue \\
PartofG & 37.1197026918213 & -1.1496158836885042 & Blue \\
Star & 37.43891173256027 & 0.1489862508458985 & Blue \\
Seyfert 1 & 37.47787637713116 & -30.599559376424764 & Blue \\
Star & 37.48751137869475 & -1.0089856734353149 & Red \\
Galaxy & 37.54069701310436 & -31.60505123369656 & Blue \\
Seyfert 1 & 37.58720416243717 & 0.2321681675793745 & Blue \\
Galaxy & 38.03891271958179 & -1.3861949894293877 & Red \\
Galaxy & 38.07534927116573 & -33.84528829588545 & Blue \\
QSO & 38.12763748929023 & -1.2817952492366265 & Blue \\
Galaxy & 38.174504835656805 & -39.295229245873685 & Blue \\
Galaxy & 38.20293944460951 & 0.8607798584028256 & Blue \\
Nova & 38.34423535712647 & 0.8498264300321089 & Blue \\
EmG & 38.39285833594685 & -39.04470809996566 & Blue \\
QSO & 38.39735766972906 & -1.1290537121536242 & Blue \\
Galaxy & 39.11980508074 & -0.9749932035748634 & Blue \\
QSO & 39.14871940890983 & -0.5342706372644691 & - \\
QSO & 40.24643683967374 & 0.7627476177593495 & Blue \\
QSO & 40.6454475160122 & -1.0644227483678184 & Blue \\
HII & 40.69556506354039 & 0.0239358351857076 & Blue \\
HII & 40.89837979736232 & 1.3771938278420643 & Blue \\
HII & 40.90704549503712 & 1.3729265625996037 & Blue \\
HII & 40.92809118446975 & 1.3595621460159046 & Blue \\
HII & 40.933458965510255 & 1.3779211213547329 & Blue \\
Galaxy & 41.52198188668681 & -33.08317001603085 & Blue \\
Star & 41.60058125535593 & -0.4980253582007756 & Blue \\
QSO & 41.60311859466317 & -0.5045181639339942 & Blue \\
PartofG & 41.60593375574607 & -0.5027112363938046 & Blue \\
Star & 41.61070793418088 & -0.5000978872359795 & Blue \\
QSO & 42.75267977777988 & 0.2853735115843281 & Blue \\
Galaxy & 43.06969129708544 & 0.2947725157476631 & Blue \\
QSO & 43.21668023139433 & -0.3698979850578264 & Blue \\
Seyfert 1 & 43.60887244004112 & -0.6896473847720612 & Blue \\
RadioG & 43.98850261320957 & 0.692640180601257 & Red \\
QSO & 44.030210339218165 & 1.1774555299992029 & Blue \\
HII & 44.10549886585857 & 0.5748223254603584 & Blue \\
Galaxy & 44.118458575759135 & 0.6078355659352473 & Blue \\
Galaxy & 44.439740352809245 & -33.482079149106816 & Red \\
Seyfert 1 & 44.79324003091252 & -0.3777292200872463 & Blue \\
Galaxy & 45.79093859633274 & 0.2271602138922271 & - \\
Galaxy & 46.05196517556931 & -1.1927199888695097 & Red \\
Galaxy & 46.07402391239703 & -0.8254163698523541 & Blue \\
Seyfert 1 & 46.144829680398814 & -0.4751996068898478 & Blue \\
QSO & 46.20769411378219 & -0.1370507568661265 & Blue \\
Blue & 46.2415646756668 & 0.9538832761700016 & Blue \\
Galaxy & 46.32601371119922 & -0.1594812724372881 & Blue \\
QSO & 46.55298185202796 & 1.3659244829765935 & Blue \\
AGN & 46.6217332236032 & -33.89231237580181 & Blue \\
Galaxy & 46.62637094367816 & -0.1063532167324183 & Red \\
Galaxy & 46.81501686424852 & 0.7312750476021798 & Blue \\
QSO & 46.98979992986573 & 0.120017369478221 & Blue \\
QSO & 47.87373270347458 & -0.2837361426333072 & Blue \\
WD* & 47.87885012806618 & -31.880860044582445 & Blue \\
GinGroup & 48.202558396796846 & -31.48630904526674 & Blue \\
Galaxy & 48.24315359852245 & -0.0815656430783389 & Blue \\
Galaxy & 48.48367671208701 & -31.470164460703632 & Blue \\
Galaxy & 48.61771428658176 & 0.7519402356960914 & Blue \\
Galaxy & 49.06378553118447 & -31.209241913161897 & Blue \\
Galaxy & 49.0750504981236 & -0.5069279648561904 & Blue \\
Galaxy & 49.21104988541157 & -33.30106752208543 & Blue \\
Galaxy & 49.21108348383686 & -33.30109956831036 & Blue \\
Galaxy & 49.62107027321726 & -0.0112548928671764 & Blue \\
QSO & 49.68820265523018 & -0.3125867996899615 & Blue \\
QSO & 49.905357996326025 & -0.4447129467729893 & Blue \\
QSO Candidate & 50.68707745560916 & 0.7450965554947673 & Blue \\
GinPair & 51.26729795433997 & -36.92773995131392 & Blue \\
Galaxy & 51.3044429528618 & -36.369356180322086 & Blue \\
GlCl & 51.48028596578508 & -32.88302604889567 & Blue \\
QSO & 53.10952432881614 & -1.1905549158813071 & Blue \\
QSO & 53.10958810574174 & -1.190642242028551 & Blue \\
Seyfert 2 & 53.292001605589874 & 0.1469919509442148 & Red \\
QSO & 53.74365145217437 & -0.1288517115468432 & Blue \\
QSO & 54.58960430404596 & 0.5184971700047292 & Blue \\
BLLac & 54.67234367033864 & -35.52621813852562 & Blue \\
BLLac & 54.67235585673245 & -35.52616012986378 & Blue \\
QSO & 54.86439460038415 & -34.61862342769125 & Blue \\
HMXB & 55.05160899500496 & -35.62780439037955 & Blue \\
EmG & 55.08289490156473 & 1.0585242193939064 & Blue \\
QSO & 55.09580981936138 & -35.26861884407053 & Blue \\
AGN & 55.21032155384233 & -35.439371817285185 & Blue \\
GinGroup & 55.38562960618687 & -34.88861116743838 & Blue \\
Seyfert 1 & 55.69883615683876 & 1.1591704436722594 & Blue \\
Star & 55.76939354280558 & 0.4200849545421101 & Blue \\
Galaxy & 56.01948427011065 & -38.13665123007838 & Red \\
QSO & 56.03437204036832 & -0.5182824235374773 & Blue \\
Galaxy & 56.11555619710486 & -0.4611633605478731 & Red \\
QSO & 56.32088432677072 & -0.2638128432166242 & Blue \\
GinCl & 56.43910392669589 & -36.34612610981142 & Blue \\
Seyfert 2 & 56.51053562993339 & -0.0162825970895807 & Red \\
GinCl & 56.84143331660593 & -32.85143564003734 & Red \\
LSB G & 57.28305368748847 & 1.162003553131273 & Blue \\
LSB G & 57.28697520415392 & 1.1628501269812663 & Blue \\
EB* & 57.830647615852506 & 0.5379309203257563 & Red \\
Galaxy & 58.760643529078 & -38.594508054321246 & Red \\
Candidate WD* & 58.81671653823677 & -37.49575812066213 & - \\
Star & 58.91021078569528 & 0.4763724886809515 & Blue \\
Galaxy & 59.023230308727214 & -49.47798028782121 & Blue \\
Galaxy & 59.21160650977977 & -0.2430266735645456 & Red \\
Galaxy & 59.342101075938615 & -37.03168069989505 & Blue \\
Galaxy & 59.384437936839966 & -0.0132210920265977 & Blue \\
Galaxy & 59.76629933224044 & 1.359321839400579 & Blue \\
Galaxy & 59.76629933224044 & 1.359321839400579 & Blue \\
EmG & 60.12243387703378 & -49.03011512394491 & Blue \\
AGN Candidate & 60.19197563815276 & -34.40769068554132 & Blue \\
Galaxy & 60.22136110841383 & -35.23783495731069 & Blue \\
QSO & 60.794009739694005 & -34.94910807950281 & Blue \\
Galaxy & 60.98645768730256 & -38.23292063597632 & Blue \\
Galaxy & 61.17143615831319 & -34.96550088599031 & Blue \\
Galaxy & 61.33498273596544 & -36.81638793661925 & Blue \\
Galaxy & 61.33500967007201 & -36.81633617582762 & Blue \\
Star & 61.366561383475336 & -38.18946988054663 & Blue \\
QSO & 62.877144254160285 & -33.89197020875841 & Blue \\
Galaxy & 62.9931906508685 & -37.97889641219351 & Blue \\
GinGroup & 63.22209217387362 & -31.30837649675762 & Blue \\
Galaxy & 63.24997813266718 & -38.32846190144912 & Blue \\
Candidate WD* & 65.02824515881092 & -32.85555378019325 & - \\
Galaxy & 65.23682931827716 & -32.84522864914376 & - \\
Galaxy & 66.49750313852422 & -43.27294370064416 & Blue \\
Galaxy & 66.63573911050027 & -41.03239942176444 & Blue \\
Galaxy & 66.68476952384738 & -42.261434192503216 & Blue \\
Galaxy & 66.68695800025141 & -42.09458791504701 & Blue \\
Galaxy & 66.68714006680887 & -42.97731957959581 & Blue \\
Galaxy & 66.92599861968759 & -42.63898741357326 & Blue \\
Galaxy & 67.11963551092221 & -43.24142208891916 & Blue \\
CataclyV* & 67.91485862835738 & -30.25392017250568 & Blue \\
Galaxy & 68.18465267691735 & -32.02224565575888 & Red \\
Galaxy & 68.29080626237631 & -43.77044714232464 & Blue \\
Galaxy & 68.83252077391391 & -42.20328938270354 & Blue \\
Galaxy & 68.92411946177829 & -47.878850433212456 & Blue \\
Galaxy & 69.29655862698584 & -46.68521526277068 & Blue \\
Galaxy & 69.3258708273691 & -32.36953198691898 & Blue \\
Candidate RRLyr & 69.3959956471664 & -44.34132589285379 & Blue \\
Galaxy & 69.59805855964451 & -33.08572978600444 & Blue \\
Galaxy & 69.86435708444397 & -42.98661407860815 & Blue \\
Galaxy & 71.43743287875098 & -38.64685520996582 & Blue \\
Galaxy & 72.09605095751637 & -44.88268692292141 & Blue \\
Galaxy & 72.5097904389203 & -47.47749473675231 & Blue \\
Galaxy & 72.95356751115321 & -46.6268828264031 & Blue \\
Star & 73.12966757047025 & -44.18451458170195 & Blue \\
Galaxy & 73.33107814604597 & -32.775789122825564 & Blue \\
Seyfert 1 & 73.67932821336544 & -48.22227714319606 & Blue \\
SN & 73.71975877176362 & -37.32095899536727 & Blue \\
GinPair & 73.75077658898451 & -37.25984603891258 & Blue \\
Galaxy & 73.8604762285104 & -30.5912705964922 & Blue \\
Galaxy & 149.25704125132305 & -26.49125060546518 & Blue \\
Galaxy & 149.27756719714128 & -19.11835403915 & Blue \\
Galaxy & 149.34776368562885 & -7.214281786532718 & Blue \\
Galaxy & 149.65458057970807 & -47.0834549455375 & Blue \\
Galaxy & 149.94509282349162 & -19.46665636988552 & Blue \\
RRLyr & 149.9619948222865 & -38.50635457434919 & Blue \\
Galaxy & 150.02266834541734 & -38.791294342178766 & Blue \\
GinGroup & 150.02430652185143 & -31.553009436144805 & Red \\
Galaxy & 150.20467895584443 & -30.544964090090623 & Blue \\
Galaxy & 150.24614431502653 & 3.46430477903397 & Blue \\
LSB G & 150.28779184781126 & -19.441570601534337 & Blue \\
Galaxy & 150.39201389787095 & -7.882130572619383 & Blue \\
Star & 150.54887788731048 & -19.426983577758755 & Blue \\
QSO & 150.56596957389203 & -0.1821573352244082 & Blue \\
Galaxy & 150.6057355105053 & 1.3268834909408125 & Blue \\
GinGroup & 150.6613029459264 & -45.49829273146848 & Red \\
HII & 150.7159322821982 & -26.15665080119176 & Blue \\
BlueSG* & 150.7278728767585 & -26.149887286064303 & Blue \\
HII & 150.7346337031606 & -26.14957887246477 & Blue \\
HII & 150.7477842615568 & -26.14623490714214 & Blue \\
HII & 150.74781937989135 & -26.146224890426577 & Blue \\
AGN Candidate & 150.75873809239062 & -26.149651149010563 & Blue \\
Galaxy & 150.7687797537192 & -19.82725437530812 & Blue \\
Galaxy & 150.81269716967137 & -5.9091034216348906 & Blue \\
BlueSG* & 150.8235011834385 & -26.167151664932163 & Blue \\
QSO & 150.9246975604601 & -15.135802684821307 & Blue \\
Galaxy & 150.96798139179825 & -31.41348352964001 & Blue \\
Galaxy & 151.0827329372396 & -44.425732478471055 & Blue \\
Galaxy & 151.28199229547042 & -19.85840135525412 & Blue \\
Galaxy & 151.32213147773072 & 1.639363549720132 & Blue \\
Galaxy & 151.36878333879326 & -38.12502118371568 & Blue \\
QSO & 151.41616300935794 & 4.154090690739466 & Blue \\
RRLyr & 151.45414160514633 & -25.69641389129041 & Blue \\
Galaxy & 151.57165594109023 & -6.5743506042744455 & Blue \\
Galaxy & 151.59250580663203 & -26.83255786101104 & Blue \\
GinGroup & 151.63078465018722 & -32.04330204360609 & Red \\
HII G & 151.63902577766424 & -29.93550371363763 & Blue \\
X & 151.64125084856218 & -29.93654830008841 & Blue \\
Galaxy & 151.79613569006344 & -19.06794716771129 & Blue \\
Galaxy & 151.863516303881 & -31.9241460721396 & Blue \\
EB* & 151.89099681523237 & -30.32206607484225 & Blue \\
CataclyV* & 151.894380024813 & -20.292330786813192 & Blue \\
CataclyV* & 151.894397417937 & -20.29234976425421 & Blue \\
Galaxy & 152.04483687127907 & -33.517284694292314 & Red \\
Galaxy & 152.09172791204222 & -14.810029537324146 & Blue \\
Galaxy & 152.12536783724758 & -26.359156873827526 & Blue \\
GinGroup & 152.30756133387325 & -43.00250347539357 & Red \\
Galaxy & 152.4594322607411 & -35.462014678338335 & Blue \\
Galaxy & 152.49472827763157 & -20.516514889740176 & Blue \\
Galaxy & 152.71002820639652 & -30.423435447476333 & Blue \\
Galaxy & 152.71586515599603 & -35.00780396726373 & Blue \\
EmG & 152.79126867563463 & -20.870559523696024 & Blue \\
Galaxy & 152.80618764806343 & -29.45774182077688 & Blue \\
EB* & 153.0033682767619 & -36.95700736727947 & Red \\
Galaxy & 153.0140154103735 & -32.46855221932521 & Blue \\
Star & 153.1982442378898 & -47.56420453243028 & Blue \\
Galaxy & 153.24854544281163 & -34.935163258478006 & Blue \\
EmG & 153.42462857062154 & -34.85508456238106 & Blue \\
Galaxy & 153.475351986868 & -34.73973403487098 & Blue \\
Galaxy & 153.60648341878957 & -30.708348670780456 & Blue \\
Galaxy & 153.61172675715056 & -23.484691315285414 & Blue \\
EmG & 153.6739002429698 & -44.85392470634676 & Blue \\
EmG & 153.68898998838594 & -34.059173447990155 & Red \\
Galaxy & 153.7005254616359 & -43.53042741275906 & Blue \\
Galaxy & 153.738831902119 & -43.61921964149586 & Red \\
EmG & 153.93560354915405 & -20.295547464050404 & Red \\
Star & 153.99297118154576 & -47.969198865203865 & Blue \\
Seyfert 2 & 154.07782142695768 & -33.56382825508867 & Red \\
EmG & 154.3048015423118 & -21.066758849089943 & Blue \\
Galaxy & 154.52385478212994 & -31.646946374240247 & Blue \\
QSO & 154.59067560656675 & -21.66881849744636 & Blue \\
Candidate CV* & 154.72293935897585 & -40.11213368993416 & Blue \\
Galaxy & 154.755114203685 & -37.67199013139984 & Blue \\
HII G & 154.8382251423767 & -22.14259842833651 & Blue \\
HII G & 154.83865197802905 & -22.1433030664676 & Blue \\
EmG & 154.92214062464984 & -25.814701960679475 & Blue \\
Galaxy & 155.11884791676323 & -23.47924382987377 & Blue \\
Galaxy & 155.13632707464745 & -23.448329902566297 & Blue \\
Candidate CV* & 155.1756706372574 & -33.83399007627319 & Blue \\
Star & 155.18046602174618 & -20.798507707460985 & Blue \\
EmG & 155.20376267846092 & -23.46585528316608 & Red \\
Galaxy & 155.28862213986295 & -32.86107252975438 & Blue \\
Galaxy & 155.3376370346406 & -21.607687357095795 & Blue \\
Galaxy & 155.50926414134003 & -39.87941443605516 & Blue \\
AGN Candidate & 155.66642433355923 & -30.49182167237173 & Blue \\
Galaxy & 155.74810048793466 & -42.82746318744498 & Blue \\
GinGroup & 155.75973043034236 & -39.16661255574749 & Blue \\
EmG & 155.91777428751132 & -35.825975288268005 & Red \\
EmG & 155.91778896569792 & -35.825982705450976 & Red \\
Galaxy & 156.08946269809243 & -43.917108333597646 & Blue \\
Seyfert 2 & 156.1309531167453 & -23.55266399516915 & Red \\
EB* & 156.30609371925286 & -35.67130335611048 & Blue \\
Galaxy & 156.53085826350642 & -24.555606178694944 & Blue \\
GinCl & 156.5904324509266 & -29.19937594593883 & Blue \\
CataclyV* & 156.77427199442948 & -43.72812711312928 & Blue \\
Galaxy & 156.83444311537312 & -23.8054382317716 & Blue \\
SN & 156.9597791681585 & -43.90576634134269 & Red \\
SN & 156.95987056162335 & -43.90578498083468 & Red \\
X & 156.96365603572264 & -43.89957006382954 & Blue \\
IG & 156.96371293405733 & -43.903895415957074 & Red \\
IG & 156.96375808362262 & -43.90381928622784 & Red \\
X & 156.9655292631724 & -43.90377424167152 & Blue \\
X & 156.96556060694496 & -43.90370677872753 & Blue \\
HII & 156.97034612111383 & -43.90320661166486 & Blue \\
HII & 156.9703863167105 & -43.90311040224713 & Blue \\
HII & 156.9703863167105 & -43.90311040224713 & Blue \\
AGN & 157.17900380093772 & -31.0382414041846 & Red \\
CataclyV* & 157.1827613944662 & -16.21759565677345 & Blue \\
RRLyr & 157.24187820994632 & -30.14102445233805 & Blue \\
GinGroup & 157.25296711210694 & -40.08275785582687 & Blue \\
EmG & 157.29613556479967 & -30.342418658284583 & Blue \\
Galaxy & 157.6274499803631 & -36.47975718945754 & Blue \\
Galaxy & 157.6274682004481 & -36.479748605745336 & Blue \\
GinGroup & 157.7403937388915 & -34.70791610390661 & Blue \\
Galaxy & 157.75073627506603 & -40.17847373737768 & Red \\
EmG & 157.8745742217274 & -32.71308133498473 & Blue \\
Seyfert 2 & 157.96713209517503 & -34.853619773852564 & Red \\
Galaxy & 157.98905881760817 & -41.81141143980291 & Blue \\
GinCl & 158.24673600744714 & -27.54358314016397 & Blue \\
Galaxy & 158.32248528920698 & -43.07862690646628 & Blue \\
GinGroup & 158.5031268315751 & -35.282679860025794 & Blue \\
GinGroup & 158.50313124725002 & -35.28257740972755 & Blue \\
GinCl & 158.61142286949843 & -27.50109937658653 & Blue \\
EmG & 158.66144673587488 & -28.583352366148148 & Blue \\
EmG & 158.72747432181612 & -20.548791249883188 & Blue \\
Galaxy & 158.74382525909093 & -40.91202846007671 & Blue \\
GinCl & 158.76197984814078 & -29.506601635883555 & Blue \\
EmG & 158.78217874356557 & -27.991310579496265 & Blue \\
EmG & 158.82799732327132 & -36.87848044358572 & Blue \\
GinCl & 158.84031456075968 & -27.695706429361035 & Blue \\
Galaxy & 158.88207558555908 & -30.83334548440517 & Blue \\
Galaxy & 158.89231455964594 & -44.57808865211405 & Blue \\
Galaxy & 159.01109904047328 & -24.906702733164057 & Blue \\
Galaxy & 159.02890003630893 & -28.295841627615104 & Blue \\
EmG & 159.09213938275613 & -25.37651306213532 & Blue \\
GinCl & 159.12642175053503 & -27.90110338436836 & Blue \\
GinCl & 159.1895107330749 & -28.167422472389195 & Blue \\
HII G & 159.22859031523527 & -26.240560203605337 & Blue \\
Galaxy & 159.23368459218906 & -26.903779869033656 & Blue \\
GinCl & 159.25764732775707 & -28.3671309867131 & Blue \\
Galaxy & 159.2685358307238 & -31.365924306516472 & Blue \\
EmG & 159.30363493989583 & -27.683956942836808 & Blue \\
GinCl & 159.33289005555525 & -28.238872359967427 & Blue \\
Galaxy & 159.3425230747454 & -27.544963483924352 & Blue \\
MIR & 159.47885157911082 & -24.42906249628565 & Red \\
IG & 159.52433508127856 & -25.09446001133432 & Blue \\
HII & 159.55988117082038 & -38.09040420988944 & Blue \\
GinCl & 159.6195197846552 & -28.51528044835931 & Blue \\
Galaxy & 159.62643334547482 & -23.54853332522397 & Blue \\
EmG & 159.63924688805704 & -27.737156832802217 & Blue \\
EmG & 159.67291882499617 & -25.592286212707226 & Blue \\
Galaxy & 159.73852013561282 & -20.04493226744504 & Blue \\
Galaxy & 159.80424882724313 & -20.636809607995904 & Blue \\
EmG & 159.8583643577597 & -23.75467363684896 & Blue \\
CataclyV* & 159.9998687263755 & -47.023971746909325 & Blue \\
CataclyV* & 159.9998687263755 & -47.023971746909325 & Blue \\
IG & 160.12922543109804 & -29.26958257985193 & Blue \\
EmG & 160.24459285878757 & -21.78454117893259 & Red \\
Galaxy & 160.2588369311913 & -30.794483697929085 & Blue \\
RRLyr & 160.26609841870317 & -34.18983231402896 & Blue \\
Seyfert 1 & 160.31322227492322 & -21.02303660410036 & Blue \\
Galaxy & 160.36546852844964 & -31.780316177810057 & Blue \\
Galaxy & 160.36546852844964 & -31.780316177810057 & Blue \\
Galaxy & 160.3964705972055 & -37.46929287562097 & Blue \\
Galaxy & 160.41429810549982 & -27.777270815267524 & Blue \\
IG & 160.52757582075228 & -22.105582546244552 & Blue \\
Galaxy & 160.581242806875 & -36.320460681862535 & Blue \\
Galaxy & 160.65827140741635 & -23.935679330514013 & Blue \\
EmG & 160.8791677599853 & -30.77221131800008 & Blue \\
EmG & 160.8791677599853 & -30.77221131800008 & Blue \\
Galaxy & 161.0404678812419 & -20.81930794850729 & Blue \\
IG & 161.25085749109647 & -22.15227748471596 & Blue \\
Galaxy & 161.394795164108 & -24.28370035395108 & Blue \\
EmG & 161.57129581271013 & -28.42321162775831 & Blue \\
Galaxy & 161.62610118824867 & -30.321619430632296 & Blue \\
EmG & 161.6602141109401 & -36.353314928386006 & Blue \\
Galaxy & 161.69807525519326 & -23.32773309331226 & Blue \\
Star & 161.84960610101908 & -41.99703765702019 & Blue \\
EmG & 161.9348517270216 & -20.96356099589758 & Blue \\
HII G & 161.9671272252362 & -20.08148354643084 & Blue \\
HII G & 161.96719252130313 & -20.081516627814896 & Blue \\
Radio & 162.09779631183395 & -25.16210701557664 & Red \\
Galaxy & 162.10543718248564 & -21.850134134896635 & Blue \\
SN & 162.10604102445896 & -25.16002800864644 & Red \\
Galaxy & 162.17634873432067 & -32.6437113818742 & Blue \\
Galaxy & 162.44532875160957 & -28.67697878087972 & Blue \\
Galaxy & 162.6651817194897 & -18.542879179509185 & Red \\
Galaxy & 162.75154034827315 & -20.23925755451401 & Blue \\
Galaxy & 162.75753741468907 & -28.33790302356877 & Blue \\
Galaxy & 162.86415841998232 & -19.6936241482243 & Blue \\
Galaxy & 162.95445605894844 & -21.88819830344904 & Blue \\
Galaxy & 163.13766385805908 & -23.149877815633783 & Blue \\
Candidate SN* & 163.6701579306735 & -39.22193302241251 & Blue \\
Galaxy & 163.84009494937516 & -23.424249462174128 & Blue \\
Galaxy & 163.8401072399262 & -23.42423690390457 & Blue \\
Galaxy & 164.15997638632015 & -20.78672855473099 & Blue \\
Galaxy & 164.20211131014338 & -19.833433033461127 & Blue \\
Galaxy & 164.26801410841605 & -33.15564025093645 & Red \\
Galaxy & 164.3079665625576 & -47.66979296726016 & Red \\
Galaxy & 164.40217291905444 & -35.5043965318927 & Blue \\
Galaxy & 164.68435964333 & -19.158632283113736 & Blue \\
CataclyV* & 164.7459588212315 & -31.60947413507693 & Blue \\
Galaxy & 164.79107984418525 & -27.9998147852096 & Blue \\
Candidate RRLyr & 165.46357641452303 & -46.88458955272901 & Blue \\
PM* & 165.49154204855074 & -23.790923666401383 & Blue \\
CataclyV* & 165.90237819943616 & -21.629405520763783 & Blue \\
GinPair & 165.9423233893419 & -23.245019645533223 & Blue \\
UV & 165.99608941173514 & -18.77670167044572 & Blue \\
IG & 166.8045693341223 & -19.81867010242588 & Blue \\
EmG & 166.8297173445012 & -19.55555857294575 & Blue \\
EmG & 166.8297504542546 & -19.55541255601515 & Blue \\
IG & 166.9493403358462 & -20.02226907187982 & Blue \\
Galaxy & 167.71070433927775 & -21.97451931840779 & Blue \\
Galaxy & 167.73846916883397 & -21.9481524929022 & Blue \\
Galaxy & 168.4623790067395 & -21.44856383511027 & Blue \\
EmG & 168.67501934507874 & -23.727739165310048 & Blue \\
QSO & 169.1815896684582 & -17.194855188675877 & Blue \\
Galaxy & 169.31254592914237 & -18.973439165695304 & - \\
Galaxy & 169.39609173910995 & -22.751735680185 & Red \\
EB* & 170.73371282684457 & -24.47777875977149 & Blue \\
Galaxy & 172.39240723207388 & -24.777518733865605 & Blue \\
AGN & 172.868588307185 & -19.98410951303733 & Blue \\
Candidate CV* & 173.0792138701351 & -21.661910741435744 & - \\
Galaxy & 173.60216658108814 & 1.1543503176286625 & Blue \\
Star & 174.05290786239414 & 0.0819136002772735 & Blue \\
EmG & 174.15304036864492 & 0.8172611304944561 & Blue \\
Galaxy & 174.15329276947446 & 0.8154915247975553 & Blue \\
CataclyV* & 174.34240673564196 & 1.816372111607291 & Blue \\
Candidate WD* & 174.45820819375535 & -20.1269725960127 & Blue \\
Galaxy & 174.726375893798 & -1.6428088681911317 & Blue \\
RRLyr & 174.73164786697532 & -21.19658104320207 & Blue \\
Galaxy & 174.75578476142175 & 1.3382478314723678 & Blue \\
QSO & 174.7681233226218 & -1.4402891840929772 & Blue \\
Galaxy & 175.3959908754587 & -18.19457670541773 & Blue \\
HII G & 175.44030773451672 & -1.901335843741868 & Blue \\
Galaxy & 175.53808352236646 & -18.16909360257072 & Red \\
PartofG & 175.55137074064442 & 0.3342699608891871 & Blue \\
Galaxy & 175.5609804603061 & -2.53146576861475 & Blue \\
RRLyr & 175.65814831106258 & -20.45605247119997 & Blue \\
Seyfert 1 & 175.712294430323 & 1.516178983683592 & Blue \\
QSO & 175.8722683259682 & -2.0554055350002094 & - \\
QSO & 175.8722971552406 & -2.0555300405054746 & - \\
Galaxy & 175.94213914323512 & -1.2761143494938 & - \\
Galaxy & 175.9475214240503 & 1.5149705815849348 & Blue \\
RRLyr & 176.03675655570706 & 1.4057473653603636 & Blue \\
EmG & 176.21230592350696 & 1.72354409641951 & Blue \\
Candidate WD* & 176.23234742536548 & -17.944273338719373 & Blue \\
Galaxy & 176.28348295379152 & -0.9883941397936244 & Blue \\
Galaxy & 176.29884651458195 & -0.90069812354485 & Red \\
Galaxy & 176.3135693424706 & -20.746519437466933 & Blue \\
Galaxy & 176.35960159169613 & 0.0041113183584209 & Blue \\
Galaxy & 176.50187966637736 & 0.1769481366071585 & Red \\
Galaxy & 176.53216927105808 & -0.4579777695914926 & Blue \\
QSO & 176.6796613966468 & 1.1885571745316508 & Blue \\
EmG & 176.79778025270713 & -0.4516064637217532 & Blue \\
Seyfert 1 & 177.07589066324908 & -1.6399572049281346 & Blue \\
Galaxy & 177.07645388246996 & -1.6418060467883486 & Blue \\
QSO & 177.41500197480892 & 1.773741489322032 & Blue \\
HII G & 177.59908479508587 & -0.5282934246500881 & Blue \\
GinCl & 177.65124196182416 & -0.5685118230038877 & Blue \\
Galaxy & 177.65163991073865 & -0.5673845606788012 & Blue \\
QSO & 177.70537423275803 & -0.8636364611502556 & Blue \\
Galaxy & 177.8042441388274 & -22.890342062408017 & Red \\
Galaxy & 177.87268901124386 & -0.0593301076444981 & Red \\
Galaxy & 177.88734562043996 & -2.372761176366136 & Blue \\
Seyfert 1 & 177.88896201644386 & -2.3727022231183543 & Blue \\
Galaxy & 178.07031354806253 & 1.3909661333218957 & Red \\
EmG & 178.07224961864256 & -2.8840749153908187 & Blue \\
GinGroup & 178.11507669366992 & -20.10390480837867 & Blue \\
Seyfert 1 & 178.15543681764203 & -2.4692788013971625 & Blue \\
HII G & 178.15697510920216 & -2.468564555740332 & Blue \\
Seyfert 1 & 178.19800706306424 & -0.668799198555378 & Blue \\
Seyfert 1 & 178.19803381207012 & -0.6688280067157996 & Blue \\
Galaxy & 178.30862939134568 & -3.409047236958316 & Blue \\
Galaxy & 178.36942473779808 & -3.230237358905925 & Blue \\
QSO & 178.4393258721732 & -2.722344158881375 & Blue \\
GinPair & 178.55129628495746 & 0.1367777756088964 & Blue \\
Galaxy & 178.7357899597813 & 0.1848611250308225 & - \\
Galaxy & 178.79661341326374 & 0.4848658484267525 & Blue \\
Galaxy & 178.79873915999408 & 0.4902688388336615 & Blue \\
Galaxy & 179.29940481773815 & -2.686921090064285 & Blue \\
Galaxy & 179.3012016296504 & -2.686482916082229 & Blue \\
Galaxy & 179.36682377284328 & -19.624040211696983 & Blue \\
Galaxy & 179.40451295413072 & -2.027027964558633 & Blue \\
Galaxy & 179.4045305314517 & -2.026996683135112 & Blue \\
QSO & 179.4500979791226 & 1.722464633451602 & Blue \\
QSO & 179.47609913312544 & -1.6377672684560611 & - \\
Galaxy & 179.48620454609028 & -20.565663973239875 & Blue \\
Galaxy & 179.53333868601857 & -17.893384995930617 & Blue \\
Galaxy & 179.59916114110445 & -19.517546281041724 & Blue \\
EmG & 179.74240913634958 & -19.02992833661004 & Blue \\
Galaxy & 179.8478669187984 & -1.7228572218896734 & Blue \\
SN & 179.87055152576207 & -19.25634139982329 & Blue \\
Galaxy & 180.08253579679277 & -20.802092442909533 & Blue \\
Galaxy & 180.08415872887417 & -1.1066153905837022 & Blue \\
QSO & 180.0906960738449 & -2.72525835885637 & Blue \\
PartofG & 180.1095757612321 & -1.1019334954271696 & Blue \\
QSO & 180.15952618608105 & 1.2129112212203663 & Blue \\
QSO & 180.1873094362587 & -18.995686347147792 & Blue \\
Galaxy & 180.1978076966742 & -3.420035887658516 & Blue \\
HI & 180.1988428326189 & -0.02340688109002 & Blue \\
GinPair & 180.29522824339665 & -1.2972657148586302 & Blue \\
QSO & 180.34692325828544 & 0.4745788784794196 & Blue \\
Galaxy & 180.3769799367477 & -23.31854795335849 & Blue \\
Cl* & 180.46003622778284 & -18.870100269365853 & Blue \\
HMXB & 180.46006145306487 & -18.87215768996868 & Blue \\
HII & 180.4603613485204 & -18.86737026502464 & Blue \\
Radio & 180.46306050492643 & -18.87467270267031 & Blue \\
Radio & 180.4635139848257 & -18.8625819554819 & Blue \\
Cl* & 180.46623334670275 & -18.874490272565943 & Blue \\
HII & 180.4678191428082 & -18.87206620305753 & Blue \\
HII & 180.47066639368427 & -18.86764989585912 & Red \\
HII & 180.4707539538797 & -18.869071106666347 & Blue \\
HII & 180.4729992541083 & -18.86226786400905 & Blue \\
Radio & 180.47322728801936 & -18.88582768843834 & Red \\
Cl* & 180.47726007541223 & -18.868760846298173 & Blue \\
PartofG & 180.4772638227824 & -18.88438817240201 & Blue \\
MolCld & 180.4806328363593 & -18.880180958610644 & Blue \\
HII & 180.48142237348915 & -18.873019641423856 & Blue \\
MolCld & 180.481983657978 & -18.870542013361973 & Blue \\
Radio & 180.482084992802 & -18.87855472820429 & Blue \\
HII & 180.4845311148264 & -18.87743596736424 & Blue \\
EB* & 180.52810919054875 & -23.051671253398336 & Blue \\
Galaxy & 180.70997875542824 & 0.3254128494396162 & Red \\
WD* & 181.31583811226784 & -2.706276551980267 & Blue \\
Galaxy & 181.6572228298745 & -15.28811460859046 & Blue \\
Galaxy & 181.71110394382373 & -14.21553394093172 & Blue \\
QSO & 181.7517098625927 & 1.1990112046487067 & Blue \\
QSO & 182.33562177580043 & -0.4820361729402547 & Blue \\
QSO & 182.545072752764 & -0.6527062334835679 & Blue \\
Galaxy & 182.6100536766531 & -0.0870043020990194 & Blue \\
Seyfert 1 & 182.68163468989783 & -0.6523719250391294 & Red \\
EmG & 182.7543746779469 & 1.3402916186010196 & Blue \\
Seyfert 1 & 183.0613723384793 & 0.0723961267201827 & Blue \\
Galaxy & 183.06624544494187 & -0.5647749352872001 & Blue \\
EB* & 183.24909669974548 & 1.823100765666048 & Red \\
Galaxy & 183.27052753312853 & -0.6503293355092394 & Red \\
Galaxy & 183.41164031698213 & -1.2934275172309344 & Blue \\
Galaxy & 183.4506660746881 & -14.52772420569266 & Blue \\
QSO & 183.6469096827458 & -1.990114618560348 & Blue \\
QSO & 183.81343790615867 & -1.5946745490378629 & Blue \\
Galaxy & 183.91446236872125 & -2.363112482714665 & Blue \\
Galaxy & 184.0314339047394 & -2.4326620218980715 & Blue \\
Galaxy & 184.5002257605925 & 0.4326907734274652 & - \\
Galaxy & 184.52944788519463 & -3.1080109031254355 & Red \\
Galaxy & 184.5794111776165 & -14.20553320828172 & Blue \\
Galaxy & 184.579419031614 & -14.205547557474516 & Blue \\
QSO & 184.73251321873693 & 2.0005938638657152 & Blue \\
QSO & 184.9269721119905 & -0.3059539785935566 & Blue \\
HII G & 184.97137095309444 & 1.77334461639504 & Blue \\
Galaxy & 185.0155358422371 & 1.1089971952641655 & Red \\
LSB G & 185.04802604861416 & 1.9586348136308644 & Blue \\
Galaxy & 185.11998162334305 & -1.8391750369277204 & Blue \\
Galaxy & 185.12660572308948 & -0.4508268628498345 & - \\
QSO & 185.37906214223312 & 1.124474721489336 & Blue \\
Star & 185.39336391937871 & -14.964030135530916 & Blue \\
Galaxy & 185.48261972911092 & -1.593344147053986 & Blue \\
Star & 185.66389783419783 & -15.486697903533098 & Blue \\
Galaxy & 185.84330648828225 & -0.1338075560321897 & Blue \\
Galaxy & 186.051943489446 & 0.5669364629276971 & Blue \\
QSO & 186.08805453702857 & 0.3984481006440221 & Blue \\
AGN & 186.4280748118269 & 0.5727372676461775 & Blue \\
Galaxy & 186.4474227096049 & -1.3349274091067191 & Blue \\
HII G & 186.5943422902263 & -1.2547941267807978 & Blue \\
HII G & 186.5946956629417 & -1.2534088007482242 & Blue \\
QSO & 186.6069592196776 & 1.267941658397558 & Blue \\
Seyfert 1 & 186.67269731032096 & -0.3347511966887692 & Blue \\
GinPair & 186.7689240600145 & -0.9059699735306485 & Blue \\
GinPair & 186.7690053713781 & -0.9061085611109012 & Blue \\
QSO & 186.77970923418985 & 1.1364780564297168 & Blue \\
Cl* & 186.9419456708534 & 1.6004126119631985 & Blue \\
Galaxy & 187.0663127889704 & 1.828815876976476 & Blue \\
EmG & 187.21391718345924 & -2.441436093650128 & Blue \\
Galaxy & 187.311049563872 & -1.365324370079482 & Blue \\
LSB G & 187.44305324875543 & -1.2950038406448354 & Blue \\
RadioG & 187.4953452683044 & 0.0272052015597826 & Red \\
Galaxy & 187.72629290525728 & 0.9640254113499284 & Blue \\
PartofG & 187.95005920551247 & -2.970279786813015 & Blue \\
EmG & 188.01126587842555 & 0.5235179491769296 & Blue \\
HII G & 188.0984798952765 & -1.7400910842643869 & Blue \\
Blue & 188.1507145595841 & -3.310957494367809 & Blue \\
Star & 188.17323029456747 & 0.0573372798919609 & Blue \\
RRLyr & 189.26006228209985 & -15.278740375443848 & Blue \\
Star & 189.83103561160047 & -14.791965932344512 & - \\
Galaxy & 191.1121599071608 & -12.876636746547932 & Blue \\
Galaxy & 192.69598912587915 & -14.483746732287928 & Blue \\
Galaxy & 194.7464380329297 & -13.86174217154326 & Blue \\
Candidate CV* & 194.75339729526405 & -13.57832034488012 & Blue \\
Galaxy & 194.76947807533804 & -14.773193673700838 & Blue \\
Galaxy & 194.88643900048527 & -15.23870530424191 & Blue \\
Galaxy & 195.01265981525043 & -15.204780538418568 & Blue \\
GinPair & 195.1637302491048 & -14.666465631319904 & Blue \\
Galaxy & 195.27952833886835 & -13.517323249547193 & Blue \\
QSO & 195.68164279128675 & -13.931326571318332 & Blue \\
EmObj & 195.88935067936143 & -14.323090018355355 & Blue \\
Galaxy & 196.2068792963303 & -13.191177561311532 & Blue \\
Galaxy & 196.2182957489161 & -12.37186182183362 & Blue \\
Galaxy & 196.49382904825976 & -12.669019185775264 & Blue \\
Galaxy & 197.5365518439323 & -12.20565697288585 & Blue \\
EmG & 197.9928613567656 & -12.064271847970153 & Blue \\
Galaxy & 198.1183640318735 & -10.5901002725042 & Blue \\
EmObj & 198.40060156082 & -12.470900254658584 & Blue \\
Galaxy & 198.7832210949353 & -12.518085580918816 & Blue \\
Galaxy & 199.4203839129064 & -10.183193478269848 & Blue \\
Galaxy & 199.42655914037883 & -0.3376844496286302 & - \\
AGN & 199.4331632480484 & -1.000304791159635 & Blue \\
Galaxy & 199.8428608076528 & -15.156551152235588 & Blue \\
GinGroup & 199.9279871875005 & -11.474580164900791 & Blue \\
Star & 199.9899853471545 & -0.5796217824328943 & Blue \\
QSO & 199.99667654428632 & -12.487991060254044 & Blue \\
Galaxy & 200.0847263890424 & -12.571681196376176 & Blue \\
QSO & 200.0977807916915 & -0.7918937289501875 & - \\
Galaxy & 200.39451837704632 & -15.18209388051804 & Blue \\
Galaxy & 200.40763851841496 & -14.855478567286989 & Blue \\
EmG & 200.5712834913193 & -0.5484394981488907 & - \\
RRLyr & 200.93371811191045 & -12.053272612599368 & Blue \\
BlueCompG & 201.45279121661528 & -11.610508041620404 & Blue \\
BlueCompG & 201.4528190508685 & -11.610542402646049 & Blue \\
Galaxy & 201.5212567425004 & -9.370169275455012 & Blue \\
RRLyr & 202.99778408537745 & -9.884062720187078 & Blue \\
Radio & 204.0776714159391 & -7.3810632714249085 & Blue \\
Galaxy & 204.1377042494816 & -6.47921900688106 & Red \\
Blue & 204.7882876783339 & -8.327997709939716 & Blue \\
Galaxy & 205.2707900815086 & -7.018268794365892 & Blue \\
RRLyr & 205.8792268775973 & -15.316371740752832 & Blue \\
CataclyV* & 205.91016519674625 & -8.234371065828142 & Blue \\
SN & 206.66333089551333 & -9.643338332630568 & Blue \\
Blue & 206.95757353263463 & -4.169611778952146 & Blue \\
Galaxy & 207.4258304698629 & -2.199743232705236 & Blue \\
Blue & 207.63885638270955 & -12.278578148358983 & Blue \\
Galaxy & 207.8487547626973 & -6.069918761179799 & Blue \\
Galaxy & 207.89854491435565 & -2.554155873440916 & Blue \\
Galaxy & 208.0162602836453 & -2.122849268399833 & Blue \\
Galaxy & 208.0176716348379 & -2.1302616063806163 & Blue \\
Galaxy & 208.5470343354216 & -3.4408967610650127 & Blue \\
Galaxy & 208.5475170682714 & -3.4408463277549943 & Blue \\
QSO & 208.69386729420503 & -10.684057890909807 & Blue \\
GinPair & 208.89159973272297 & -5.971399339776448 & Blue \\
Galaxy & 208.907000978763 & -4.195508295444852 & Blue \\
Galaxy & 208.93941167819 & -6.0044269690496535 & Blue \\
Galaxy & 208.94449637980523 & -6.011323983078441 & Blue \\
QSO & 209.0116324685152 & -2.4398085262225853 & - \\
Galaxy & 209.2228497784912 & -2.647844687664313 & Blue \\
Galaxy & 209.5359168247792 & -4.145441774787235 & Blue \\
Galaxy & 209.67286668662445 & -1.5210215591711298 & Red \\
Galaxy & 210.6878474329628 & -7.373825503885681 & Blue \\
QSO & 216.7245347178442 & 5.421495966728137 & Blue \\
Galaxy & 216.795092182775 & 5.133141307382639 & Blue \\
Possible lensImage & 217.2307884592734 & 5.0060874486010265 & Blue \\
Blue & 217.2310767118411 & 5.0055279363461525 & Blue \\
GinCl & 217.430980101215 & 5.356359632037527 & Red \\
BClG & 217.4943714069119 & 4.769808218972206 & Red \\
Galaxy & 220.3668212426573 & 5.864515543127945 & Blue \\
QSO & 223.43549771425327 & 4.946107899138475 & Blue \\
AGN & 223.89042173389376 & 4.778676422249849 & Red \\
QSO & 300.4321955778412 & 0.8217833390436116 & Blue \\
low-mass* & 301.1349485385638 & 0.1781570740733244 & - \\
RRLyr & 302.7008333786705 & -0.2177414034676869 & Blue \\
RRLyr & 302.7008333786705 & -0.2177414034676869 & Blue \\
RRLyr & 305.64797588529285 & -0.6694194677482896 & Blue \\
RRLyr & 305.6574820578185 & -0.0473696356750876 & Blue \\
Galaxy & 307.05008700522615 & 0.288400031713299 & Blue \\
QSO & 307.2783691077352 & 0.9148870344062292 & Blue \\
Seyfert 1 & 310.91682422695794 & 0.481551561964735 & Blue \\
QSO & 311.6087878150387 & 0.3938280343112397 & Red \\
CataclyV* & 311.8364843167464 & 0.0021704620321628 & Blue \\
CataclyV* & 311.8365168802931 & 0.0021268467006492 & Blue \\
Seyfert 1 & 312.2956269426095 & 0.265970422493022 & Blue \\
QSO & 312.4859068430339 & -0.2004724242450093 & Blue \\
QSO & 312.48592414481044 & -0.200477643373699 & Blue \\
QSO & 313.3198559349733 & 0.9892076810353208 & - \\
QSO & 313.466816984702 & -0.2670765601766264 & Blue \\
Star & 314.0606317101171 & -0.680733084738148 & - \\
QSO & 314.302863294349 & 0.2031735829473498 & Blue \\
QSO & 314.4198074848838 & 0.9052857685742688 & Red \\
Candidate WD* & 314.5268677152485 & -30.1383582004064 & Blue \\
Galaxy & 314.6022450825876 & -32.722916673933284 & Blue \\
Galaxy & 314.70702257259086 & -44.34018284222228 & Blue \\
Galaxy & 314.9896989117314 & -21.65968492556584 & Blue \\
CataclyV* & 315.05879832773167 & 0.7460919799647099 & Blue \\
Galaxy & 315.48312134293826 & -0.5235874873138652 & Red \\
Galaxy & 315.48512664422447 & -39.394516374889456 & Blue \\
QSO & 315.6737967478414 & -32.87890999401036 & Blue \\
QSO & 315.67381555407934 & -32.87898718209263 & Blue \\
Galaxy & 315.7592781700893 & -45.24473926016992 & Blue \\
Star & 315.98609213331895 & -21.790860533436685 & Blue \\
IG & 316.03549398691706 & -43.53423041944633 & Blue \\
GinGroup & 316.04653712211586 & -43.592725469851146 & Blue \\
Galaxy & 316.08943735381246 & -30.197214902706765 & Red \\
EmG & 316.2304449432887 & -0.5893867983932235 & Blue \\
Galaxy & 316.3361978919587 & -45.98869609290722 & Blue \\
Galaxy & 316.41115448294687 & -42.7812242143475 & Blue \\
PN & 316.47319569385303 & -37.14456181858315 & Blue \\
Star & 316.7000786165858 & -40.334348615950496 & Blue \\
GinPair & 316.79991779423176 & -47.55699311008755 & Blue \\
GinPair & 316.80774783068466 & -47.55702616044946 & Blue \\
EmG & 318.00385158352736 & -0.280345841969967 & Blue \\
AGN & 318.10247638771125 & -41.48147711626288 & Blue \\
EB* & 318.3673788604048 & 0.0590671476084678 & Blue \\
Galaxy & 318.8297135149132 & -33.22628044991965 & Blue \\
CataclyV* & 319.517818265895 & -34.22874036730848 & Blue \\
Radio(cm) & 320.7617323963711 & -29.251115062981185 & Blue \\
Candidate CV* & 321.72725697504 & -1.348366630745761 & Blue \\
Galaxy & 321.7844092060996 & -30.9523274970352 & Blue \\
EmG & 322.4759106821289 & -1.0564795011399617 & Blue \\
QSO & 322.8730471709395 & -45.697352338099286 & Blue \\
Galaxy & 323.1762194624441 & -1.0525538970193808 & Blue \\
Seyfert 1 & 323.18856454382785 & 0.0296602067769526 & Blue \\
QSO & 323.408936906668 & 1.4413770493499043 & Blue \\
QSO & 323.72952597635833 & 0.1824349898807966 & Blue \\
QSO & 324.20730055187585 & -1.481159088195661 & Blue \\
QSO & 324.53121345495185 & -45.13834354299933 & - \\
Seyfert 1 & 324.57901744173665 & 1.2062510372667887 & Blue \\
Candidate CV* & 324.9065914699761 & -2.6536034951853305 & Blue \\
QSO & 325.2768379376971 & 0.7926481060166569 & Blue \\
Galaxy & 325.4280346483323 & 0.7597010317264318 & Blue \\
QSO & 325.47932878054104 & -1.292825833976646 & Blue \\
QSO & 325.479329853818 & -1.292849791540993 & Blue \\
SN & 326.0956022427804 & -29.91639450734865 & Red \\
EmG & 326.2329872209474 & 0.3850203444962297 & Blue \\
Galaxy & 326.4167001057865 & -29.32690805408077 & Red \\
EmG & 327.1275218035224 & -0.7979329669827752 & Blue \\
EmG & 327.127535195112 & -0.797904422039106 & - \\
QSO & 327.5112425223878 & 1.2288440704579693 & Blue \\
QSO & 327.54385873207247 & -0.1668265647596742 & Blue \\
Galaxy & 328.2731418249827 & -31.471640527742185 & Blue \\
Galaxy & 328.2731418249827 & -31.471640527742185 & Blue \\
Galaxy & 328.2731418249827 & -31.471640527742185 & Blue \\
Galaxy & 328.2731418249827 & -31.471640527742185 & Blue \\
Galaxy & 328.57500430647343 & 0.9422073085758816 & Blue \\
Galaxy & 329.0576398034554 & -1.1618941444509447 & Blue \\
Galaxy & 329.0576982984844 & -1.1619911379394898 & Blue \\
Galaxy & 329.08245077543467 & -1.1676749952103018 & Blue \\
Galaxy & 329.0825336876782 & -1.167693306004015 & Blue \\
Galaxy & 329.3370362394045 & -25.133993290826115 & Blue \\
HII G & 329.6011751965016 & -0.737145415174484 & Blue \\
Galaxy & 329.76204963872004 & -0.5550013061974669 & Blue \\
Galaxy & 329.7629603765794 & -1.9550737068589068 & Blue \\
Galaxy & 330.4586482938667 & -0.7074127748175637 & Blue \\
Galaxy & 330.5294901692158 & -26.44390115708122 & Blue \\
Star & 330.8130711615091 & 1.2891533854468642 & Blue \\
Galaxy & 331.22366426801165 & -25.051456106463764 & Blue \\
QSO & 331.37225331383246 & -0.5196269942518518 & Blue \\
IG & 331.7304425939274 & -31.052944378400603 & Red \\
Galaxy & 331.89396381531327 & -28.65815178172343 & Blue \\
GinGroup & 332.1018041812308 & -29.05100572704748 & Blue \\
Galaxy & 332.21648908406854 & -30.64966912723208 & - \\
QSO & 332.2165318654766 & -1.1010232176217527 & Blue \\
QSO & 332.21654479357403 & -1.1010212789856892 & Blue \\
GinPair & 332.2329099311068 & -27.22278199790022 & Blue \\
QSO & 332.3293752575985 & -24.12012102380704 & Red \\
Galaxy & 332.34547450375544 & -25.417946056108004 & Blue \\
WD* & 332.45263484436884 & -30.23217545253064 & Blue \\
Galaxy & 332.4639565862639 & 1.1500007990678438 & Red \\
QSO & 332.4773666618215 & -1.454892271729591 & Blue \\
EmG & 332.50311376214466 & -31.23332648999496 & Blue \\
Galaxy & 332.51558209616104 & -25.335365131244853 & Blue \\
Galaxy & 332.5240068270393 & -27.91075708397664 & Blue \\
Galaxy & 332.7416878016213 & -25.07531023085714 & Blue \\
Galaxy & 332.7430386884259 & -27.658157768445044 & Blue \\
SN & 333.17320112679323 & 0.5119642601996512 & Blue \\
QSO & 333.3985380422876 & -28.42824210005132 & Blue \\
Galaxy & 333.51191857192214 & -27.53928600571248 & Blue \\
Galaxy & 333.5119857829216 & -27.53930260557384 & Blue \\
EmG & 333.5134692326156 & -29.38272127795172 & Blue \\
Galaxy & 333.5212380480937 & -29.381350935720462 & Blue \\
EmG & 333.600966286032 & -29.98098137914205 & Blue \\
Galaxy & 333.6746944454729 & -28.44433992901234 & Blue \\
Galaxy & 333.69650411507115 & -29.68677871511806 & Blue \\
EmG & 333.8212699349557 & -28.89929996827705 & Blue \\
QSO & 333.88574209843875 & -28.301075647502863 & Blue \\
EmG & 334.1253072836468 & -29.01480498338059 & Red \\
Galaxy & 334.2771669200802 & -30.579476065999444 & Red \\
QSO & 334.3435194955581 & 1.0767543256819685 & Blue \\
Galaxy & 334.421449319186 & -27.365163382540192 & Blue \\
Galaxy & 334.5579398251045 & 0.2736951228444171 & - \\
Galaxy & 334.56262843755 & 1.254694273842988 & Blue \\
AGN & 334.57190645739905 & 0.6065838369084469 & Red \\
Seyfert 1 & 334.58077291303573 & -27.26228422509564 & Blue \\
Galaxy & 334.6947482712885 & -1.188558994516001 & Blue \\
Galaxy & 334.69475415064494 & -1.1885577406117744 & Blue \\
Galaxy & 334.71939499895126 & -1.0529142510236456 & Blue \\
EmG & 334.8579955716732 & -30.852129377014343 & Blue \\
Galaxy & 334.9364411418472 & -0.2444533826638444 & Red \\
EmG & 334.93782729725075 & -29.57039409080591 & Blue \\
Galaxy & 334.9744213770292 & 0.4846689985595231 & Blue \\
Star & 335.08900553863754 & 0.6779085075975249 & Blue \\
Seyfert 1 & 335.3067522132758 & -28.07246239885318 & Red \\
Galaxy & 335.72883636182854 & -30.70788858469623 & Blue \\
Galaxy & 335.8070780085201 & -28.979058837654552 & Blue \\
QSO & 335.88681821013813 & -1.10412204251795 & Blue \\
EmG & 335.90012802575814 & -28.527653500762096 & Blue \\
QSO & 336.01397097341777 & -0.9567096099850134 & Blue \\
QSO & 336.01401051604205 & -0.9566969699045356 & Blue \\
Candidate CV* & 336.0677440421202 & -29.40602881395364 & Blue \\
Galaxy & 336.912099323913 & -31.136207086254505 & Blue \\
Galaxy & 337.1046549602956 & -0.3714489675823685 & Red \\
Galaxy & 337.19918624431745 & -30.91199843220645 & Blue \\
Galaxy & 337.1992648299098 & -30.911989220539105 & Blue \\
Galaxy & 337.22360618425245 & -30.980939741660382 & Blue \\
Galaxy & 337.22360894565986 & -30.98095563687505 & Blue \\
QSO & 337.3458366571309 & -2.011785900315515 & - \\
QSO & 337.48557309949206 & 0.5240235390794395 & Blue \\
Galaxy & 337.5076760184757 & -29.597960152831284 & Blue \\
Galaxy & 337.50768925239095 & -29.59794652431351 & Blue \\
BlueCompG & 337.653463904362 & -0.1099571744851075 & Blue \\
Galaxy & 337.7750079939556 & -0.1955294013694825 & Blue \\
Star & 337.80812689962085 & -31.33455219401752 & Blue \\
QSO & 338.2155980841278 & -30.54711277145381 & Blue \\
Galaxy & 338.4273343986453 & -30.32651312377564 & Blue \\
Star & 338.5277783244117 & 0.0224447956398556 & Red \\
CataclyV* & 338.66637521384126 & 0.6909664418071105 & Blue \\
Galaxy & 338.73567173096313 & -31.145574300500385 & Blue \\
Galaxy & 338.73569433672816 & -31.1455850268799 & Blue \\
Seyfert 2 & 338.78506511786827 & -0.8998369935534453 & Red \\
Seyfert 2 & 338.785109135079 & -0.8998316294838345 & Red \\
EmG & 338.88432907328246 & -29.77583940043047 & Blue \\
Galaxy & 338.92933238127915 & -0.9101766325827358 & Blue \\
Galaxy & 338.92938653521645 & -0.9101558150095174 & Blue \\
Galaxy & 338.9294202112855 & -0.9100982563435506 & Blue \\
low-mass* & 338.9330431386031 & -0.6589082844776932 & Red \\
HII & 338.9369542908973 & -26.040945424398046 & Blue \\
HII & 338.9391682825249 & -26.03634410855905 & Blue \\
HII & 338.942864761339 & -26.074740893365583 & Blue \\
HII & 338.9509617589531 & -26.02333197997694 & Blue \\
QSO & 339.139729783869 & 0.4480037094611175 & Blue \\
QSO & 339.20666171031564 & 0.9038310163367896 & - \\
Galaxy & 339.3495140772359 & -1.0219920514291434 & Blue \\
HB* & 339.3744168628378 & -1.09702098520342 & - \\
QSO & 339.5968891921001 & -0.952268013442764 & Blue \\
QSO & 339.5969264365535 & -0.9522370681869864 & Blue \\
QSO & 339.6845594683822 & -0.948694319373839 & Blue \\
Galaxy & 340.3312015654902 & -39.97314857397814 & Blue \\
Galaxy & 340.4562435334129 & -30.329015431975083 & Blue \\
Galaxy & 340.81825118013086 & -39.86108916393093 & Blue \\
Galaxy & 340.8308173956169 & -39.87900062583571 & Blue \\
Galaxy & 340.854608426132 & -39.92223577354626 & Red \\
Galaxy & 340.96755531580084 & -0.3832981583866523 & Blue \\
QSO & 341.3799824113044 & -0.7526108924743485 & Blue \\
QSO & 341.3799836615205 & -0.7525947660724432 & Blue \\
QSO & 341.4164203070588 & -0.4054482118161328 & Blue \\
QSO & 342.4836663184702 & 0.0384316128868782 & Blue \\
Galaxy & 342.55381866523845 & -0.666369376718931 & Blue \\
QSO & 342.95730339479854 & -0.4698259059306293 & Blue \\
EmG & 342.9879147578966 & -29.414118119813367 & Blue \\
Star & 343.2393387782164 & 0.4587823345606925 & Blue \\
Seyfert 1 & 343.4706720102698 & -30.162151096626257 & Blue \\
QSO & 343.54646400880233 & -31.45314972730561 & Blue \\
QSO & 343.5498267824822 & -0.8303920878390473 & Blue \\
QSO & 343.5498379046172 & -0.8303498619565828 & Blue \\
EmG & 343.76622684097015 & -30.32036536315227 & Blue \\
Star & 344.7838160610223 & -31.454652481082007 & Blue \\
Galaxy & 345.12539458156243 & -0.5016202410575213 & Blue \\
QSO & 345.5049924642404 & 0.5131468261640252 & Blue \\
QSO & 345.6476494287596 & -28.941580257566937 & Blue \\
QSO & 345.8183544369748 & -0.2031687049072709 & Blue \\
CataclyV* & 345.9651406629765 & 1.11426658072561 & Blue \\
QSO & 346.1180884490532 & 0.950341888510796 & Blue \\
QSO & 346.1839995707146 & -1.047637756170335 & Blue \\
Galaxy & 347.04492000641267 & -1.299588506149179 & Blue \\
QSO & 347.231188850614 & 0.6182549476788003 & Blue \\
GinGroup & 347.2316580483845 & -30.85783430936128 & Blue \\
QSO & 347.3096188733344 & -30.986794418426523 & Blue \\
Seyfert 1 & 347.44230168607083 & 0.0136411947639146 & Blue \\
Seyfert 1 & 347.4423475040321 & 0.0136153759642055 & Blue \\
EmG & 347.67494008100954 & -1.1633342696409468 & Blue \\
QSO & 347.8963319528839 & -31.445592377850765 & Blue \\
Galaxy & 348.0373247446722 & -31.070356404741503 & Red \\
QSO & 348.13067741527794 & -1.1937043288787734 & Blue \\
QSO & 348.2461021601764 & 1.1349593899913708 & Blue \\
QSO & 348.299611065616 & -0.7605497687369258 & Blue \\
HII G & 348.46608265183744 & -1.1752128153912331 & Blue \\
LSB G & 348.7105150733019 & 1.390757174203209 & Blue \\
QSO & 348.8308033561476 & -30.64921915522369 & Blue \\
CataclyV* & 348.88240046675367 & -30.813531857016685 & Blue \\
QSO & 349.2168428172675 & 0.8571987065141106 & Blue \\
Seyfert 1 & 349.42752127421284 & 0.093129280239253 & Blue \\
QSO & 349.92814830462225 & -30.44152729995124 & Blue \\
Galaxy & 350.1467244968801 & -0.8808002402296434 & Blue \\
Galaxy & 350.1467502457788 & -0.8807767191983554 & Blue \\
Galaxy & 350.3604588557264 & -31.12485243647311 & Blue \\
RRLyr & 350.88051361237183 & 1.134997376685103 & Blue \\
HII & 351.0847562454346 & -0.1069557818481737 & Blue \\
HII G & 351.08902800249376 & -0.1081688053941925 & Blue \\
QSO & 351.2406451424701 & 0.3648363114586009 & Blue \\
Galaxy & 351.3517724774783 & 0.7700617331859075 & Blue \\
CataclyV* & 351.46448670155314 & -1.6732732825345382 & Blue \\
Seyfert 1 & 351.4812848799453 & -0.619649969309404 & - \\
BlueCompG & 351.9320769678968 & -2.0154836384737007 & Blue \\
Galaxy & 351.9349111408217 & -2.0130122785984863 & Blue \\
HII G & 352.05124626986577 & -1.0624481917322088 & Blue \\
CataclyV* & 352.25183229093045 & -29.77943600230389 & Blue \\
QSO & 352.768315319454 & -0.7102948577467144 & Red \\
Galaxy & 352.99905235631786 & -0.8051318589747921 & Blue \\
EmG & 353.22826977772627 & -30.978836100206514 & Blue \\
BClG & 353.2361807443991 & 1.1897411683153614 & - \\
QSO & 353.25093145251367 & -0.3418033383059611 & Blue \\
QSO & 353.660612982189 & 0.3949855898346673 & Blue \\
Galaxy & 353.8374272193407 & 1.1742822684735112 & Red \\
QSO & 353.8445277828368 & -0.1097797881436761 & Blue \\
LSB G & 354.19570227897805 & 0.6232818326354911 & Blue \\
QSO & 354.3417350883094 & 0.3775467635964666 & Blue \\
AGN & 354.38244578497734 & 0.4333010734691366 & Blue \\
GinCl & 354.4482155147033 & 0.295228446902126 & - \\
Galaxy & 355.16011198023494 & -0.8918305906086714 & Blue \\
QSO & 355.8714951658231 & -30.03336765255501 & Blue \\
CataclyV* & 356.1688766402904 & -0.2016823731104521 & Blue \\
Galaxy & 357.0999479884554 & -1.7919613357037034 & Blue \\
Galaxy & 357.50647665269133 & -30.18530531379982 & Blue \\
Galaxy & 357.8152522370048 & -1.0744996159314814e-05 & Blue \\
Galaxy & 357.8153294830434 & -5.8998646198314144e-05 & Blue \\
QSO & 358.9422665997112 & -0.3952102105534735 & Blue \\
QSO & 359.32653511559846 & 0.7307049350125405 & Red \\
QSO & 359.5218549462989 & -1.3649719892078715 & Blue \\
%% \end{tabular}
%% \end{table}

 \end{longtable} 
 \end{center}

%\clearpage
\section{SDSS spectra}
\begin{center}
  \begin{longtable}{l l l l l }
  \caption{Espectra from SDSS DR16 \label{tab:spec-sdss}}\
  \endfirsthead
  \caption[]{--continued}\\
  \endhead
  \hline \endfoot
    \includegraphics[width=0.19\linewidth, clip]{Figs/Figs-sdss/spec-0269-51910-0319-SPLUS-n01s01-019000.pdf} & \includegraphics[width=0.19\linewidth, clip]{Figs/Figs-sdss/spec-0282-51658-0451-SPLUS-n02n18-016802.pdf} & \includegraphics[width=0.19\linewidth, clip]{Figs/Figs-sdss/spec-0283-51959-0147-SPLUS-n01s20-010125.pdf} & \includegraphics[width=0.19\linewidth, clip]{Figs/Figs-sdss/spec-0283-51959-0496-SPLUS-n01s20-026220.pdf} & \includegraphics[width=0.19\linewidth, clip]{Figs/Figs-sdss/spec-0284-51943-0170-SPLUS-n01s21-002032.pdf} \\
    \includegraphics[width=0.19\linewidth, clip]{Figs/Figs-sdss/spec-0285-51930-0042-SPLUS-n02s23-042426.pdf} & \includegraphics[width=0.19\linewidth, clip]{Figs/Figs-sdss/spec-0285-51930-0049-SPLUS-n02s23-042530.pdf} & \includegraphics[width=0.19\linewidth, clip]{Figs/Figs-sdss/spec-0285-51930-0521-SPLUS-n02n23-022190.pdf} & \includegraphics[width=0.19\linewidth, clip]{Figs/Figs-sdss/spec-0285-51930-0549-SPLUS-n01s23-038457.pdf} & \includegraphics[width=0.19\linewidth, clip]{Figs/Figs-sdss/spec-0287-52023-0264-SPLUS-n02s24-039734.pdf} \\
    \includegraphics[width=0.19\linewidth, clip]{Figs/Figs-sdss/spec-0288-52000-0171-SPLUS-n01s26-017374.pdf} & \includegraphics[width=0.19\linewidth, clip]{Figs/Figs-sdss/spec-0288-52000-0561-SPLUS-n02n26-016552.pdf} & \includegraphics[width=0.19\linewidth, clip]{Figs/Figs-sdss/spec-0289-51990-0202-SPLUS-n02s28-039352.pdf} & \includegraphics[width=0.19\linewidth, clip]{Figs/Figs-sdss/spec-0289-51990-0210-SPLUS-n02s27-030322.pdf} & \includegraphics[width=0.19\linewidth, clip]{Figs/Figs-sdss/spec-0289-51990-0234-SPLUS-n01s27-014192.pdf} \\
    \includegraphics[width=0.19\linewidth, clip]{Figs/Figs-sdss/spec-0289-51990-0542-SPLUS-n01s28-027608.pdf} & \includegraphics[width=0.19\linewidth, clip]{Figs/Figs-sdss/spec-0327-52294-0620-SPLUS-n02s19-022704.pdf} & \includegraphics[width=0.19\linewidth, clip]{Figs/Figs-sdss/spec-0329-52056-0529-SPLUS-n02s20-017833.pdf} & \includegraphics[width=0.19\linewidth, clip]{Figs/Figs-sdss/spec-0330-52370-0072-SPLUS-n03s22-026437.pdf} & \includegraphics[width=0.19\linewidth, clip]{Figs/Figs-sdss/spec-0330-52370-0131-SPLUS-n03s21-001338.pdf} \\
    \includegraphics[width=0.19\linewidth, clip]{Figs/Figs-sdss/spec-0330-52370-0144-SPLUS-n03s21-009182.pdf} & \includegraphics[width=0.19\linewidth, clip]{Figs/Figs-sdss/spec-0385-51783-0342-STRIPE82-0164-033376.pdf} & \includegraphics[width=0.19\linewidth, clip]{Figs/Figs-sdss/spec-0385-51783-0395-STRIPE82-0164-019383.pdf} & \includegraphics[width=0.19\linewidth, clip]{Figs/Figs-sdss/spec-0389-51795-0544-STRIPE82-0008-021736.pdf} & \includegraphics[width=0.19\linewidth, clip]{Figs/Figs-sdss/spec-0390-51900-0596-STRIPE82-0010-026778.pdf} \\
    \includegraphics[width=0.19\linewidth, clip]{Figs/Figs-sdss/spec-0391-51782-0125-STRIPE82-0011-016278.pdf} & \includegraphics[width=0.19\linewidth, clip]{Figs/Figs-sdss/spec-0391-51782-0394-STRIPE82-0010-016724.pdf} & \includegraphics[width=0.19\linewidth, clip]{Figs/Figs-sdss/spec-0391-51782-0625-STRIPE82-0014-009267.pdf} & \includegraphics[width=0.19\linewidth, clip]{Figs/Figs-sdss/spec-0392-51793-0583-STRIPE82-0016-049657.pdf} & \includegraphics[width=0.19\linewidth, clip]{Figs/Figs-sdss/spec-0395-51783-0107-STRIPE82-0023-019829.pdf} \\
    \includegraphics[width=0.19\linewidth, clip]{Figs/Figs-sdss/spec-0395-51783-0525-STRIPE82-0022-026424.pdf} & \includegraphics[width=0.19\linewidth, clip]{Figs/Figs-sdss/spec-0397-51794-0336-STRIPE82-0026-058736.pdf} & \includegraphics[width=0.19\linewidth, clip]{Figs/Figs-sdss/spec-0287-52023-0466-SPLUS-n01s25-028549.pdf} & \includegraphics[width=0.19\linewidth, clip]{Figs/Figs-sdss/spec-0330-52370-0439-SPLUS-n03s21-047308.pdf} & \includegraphics[width=0.19\linewidth, clip]{Figs/Figs-sdss/spec-0376-52143-0086-STRIPE82-0139-018589.pdf} \\
    \includegraphics[width=0.19\linewidth, clip]{Figs/Figs-sdss/spec-0384-51821-0565-STRIPE82-0164-066404.pdf} & \includegraphics[width=0.19\linewidth, clip]{Figs/Figs-sdss/spec-0399-51817-0140-STRIPE82-0031-029096.pdf} & \includegraphics[width=0.19\linewidth, clip]{Figs/Figs-sdss/spec-0411-51817-0069-STRIPE82-0067-049701.pdf} & \includegraphics[width=0.19\linewidth, clip]{Figs/Figs-sdss/spec-0515-52051-0011-SPLUS-n02n22-059209.pdf} & \includegraphics[width=0.19\linewidth, clip]{Figs/Figs-sdss/spec-0677-52606-0168-STRIPE82-0151-035421.pdf} \\
    \includegraphics[width=0.19\linewidth, clip]{Figs/Figs-sdss/spec-0686-52519-0104-STRIPE82-0005-024320.pdf} & \includegraphics[width=0.19\linewidth, clip]{Figs/Figs-sdss/spec-0695-52202-0324-STRIPE82-0028-045006.pdf} & \includegraphics[width=0.19\linewidth, clip]{Figs/Figs-sdss/spec-0518-52282-0022-SPLUS-n02n26-033149.pdf} & \includegraphics[width=0.19\linewidth, clip]{Figs/Figs-sdss/spec-0519-52283-0278-SPLUS-n02n26-032045.pdf} & \includegraphics[width=0.19\linewidth, clip]{Figs/Figs-sdss/spec-0519-52283-0291-SPLUS-n02n26-039812.pdf} \\
    \includegraphics[width=0.19\linewidth, clip]{Figs/Figs-sdss/spec-0520-52288-0312-SPLUS-n02n28-039381.pdf} & \includegraphics[width=0.19\linewidth, clip]{Figs/Figs-sdss/spec-0530-52026-0013-SPLUS-n02n44-027436.pdf} & \includegraphics[width=0.19\linewidth, clip]{Figs/Figs-sdss/spec-0572-52289-0114-SPLUS-n04n01-004018.pdf} & \includegraphics[width=0.19\linewidth, clip]{Figs/Figs-sdss/spec-0573-52325-0471-SPLUS-n04n01-030147.pdf} & \includegraphics[width=0.19\linewidth, clip]{Figs/Figs-sdss/spec-0584-52049-0618-SPLUS-n05n50-017121.pdf} \\
    \includegraphics[width=0.19\linewidth, clip]{Figs/Figs-sdss/spec-0669-52559-0359-STRIPE82-0004-022337.pdf} & \includegraphics[width=0.19\linewidth, clip]{Figs/Figs-sdss/spec-0673-52162-0547-STRIPE82-0140-024629.pdf} & \includegraphics[width=0.19\linewidth, clip]{Figs/Figs-sdss/spec-0674-52201-0135-SPLUS-s02s12-025538.pdf} & \includegraphics[width=0.19\linewidth, clip]{Figs/Figs-sdss/spec-0677-52606-0545-STRIPE82-0152-022237.pdf} & \includegraphics[width=0.19\linewidth, clip]{Figs/Figs-sdss/spec-0678-52884-0272-STRIPE82-0151-054004.pdf} \\
    \includegraphics[width=0.19\linewidth, clip]{Figs/Figs-sdss/spec-0679-52177-0248-STRIPE82-0155-008068.pdf} & \includegraphics[width=0.19\linewidth, clip]{Figs/Figs-sdss/spec-0680-52200-0153-STRIPE82-0159-019049.pdf} & \includegraphics[width=0.19\linewidth, clip]{Figs/Figs-sdss/spec-0680-52200-0576-STRIPE82-0160-024148.pdf} & \includegraphics[width=0.19\linewidth, clip]{Figs/Figs-sdss/spec-0681-52199-0099-SPLUS-s02s22-033516.pdf} & \includegraphics[width=0.19\linewidth, clip]{Figs/Figs-sdss/spec-0681-52199-0169-STRIPE82-0161-044581.pdf} \\
    \includegraphics[width=0.19\linewidth, clip]{Figs/Figs-sdss/spec-0681-52199-0350-STRIPE82-0160-012044.pdf} & \includegraphics[width=0.19\linewidth, clip]{Figs/Figs-sdss/spec-0682-52525-0044-SPLUS-s02s23-032106.pdf} & \includegraphics[width=0.19\linewidth, clip]{Figs/Figs-sdss/spec-0682-52525-0493-STRIPE82-0164-022218.pdf} & \includegraphics[width=0.19\linewidth, clip]{Figs/Figs-sdss/spec-0685-52203-0487-STRIPE82-0002-029499.pdf} & \includegraphics[width=0.19\linewidth, clip]{Figs/Figs-sdss/spec-0695-52202-0519-STRIPE82-0028-012953.pdf} \\
    \includegraphics[width=0.19\linewidth, clip]{Figs/Figs-sdss/spec-0696-52209-0324-STRIPE82-0030-034826.pdf} & \includegraphics[width=0.19\linewidth, clip]{Figs/Figs-sdss/spec-0696-52209-0493-STRIPE82-0032-008674.pdf} & \includegraphics[width=0.19\linewidth, clip]{Figs/Figs-sdss/spec-0696-52209-0580-STRIPE82-0032-035715.pdf} & \includegraphics[width=0.19\linewidth, clip]{Figs/Figs-sdss/spec-0697-52226-0259-STRIPE82-0033-008289.pdf} & \includegraphics[width=0.19\linewidth, clip]{Figs/Figs-sdss/spec-0701-52179-0153-STRIPE82-0045-050333.pdf} \\
    \includegraphics[width=0.19\linewidth, clip]{Figs/Figs-sdss/spec-0702-52178-0098-STRIPE82-0047-013896.pdf} & \includegraphics[width=0.19\linewidth, clip]{Figs/Figs-sdss/spec-0702-52178-0126-STRIPE82-0047-015137.pdf} & \includegraphics[width=0.19\linewidth, clip]{Figs/Figs-sdss/spec-0703-52209-0097-STRIPE82-0049-015869.pdf} & \includegraphics[width=0.19\linewidth, clip]{Figs/Figs-sdss/spec-0705-52200-0577-STRIPE82-0056-038958.pdf} & \includegraphics[width=0.19\linewidth, clip]{Figs/Figs-sdss/spec-0707-52177-0374-STRIPE82-0060-019709.pdf} \\
    \includegraphics[width=0.19\linewidth, clip]{Figs/Figs-sdss/spec-0981-52435-0575-STRIPE82-0102-040352.pdf} & \includegraphics[width=0.19\linewidth, clip]{Figs/Figs-sdss/spec-0982-52466-0091-STRIPE82-0103-087316.pdf} & \includegraphics[width=0.19\linewidth, clip]{Figs/Figs-sdss/spec-0982-52466-0477-STRIPE82-0103-089600.pdf} & \includegraphics[width=0.19\linewidth, clip]{Figs/Figs-sdss/spec-0983-52443-0348-STRIPE82-0104-019279.pdf} & \includegraphics[width=0.19\linewidth, clip]{Figs/Figs-sdss/spec-0983-52443-0459-STRIPE82-0106-052252.pdf} \\
    \includegraphics[width=0.19\linewidth, clip]{Figs/Figs-sdss/spec-0984-52442-0311-STRIPE82-0105-069446.pdf} & \includegraphics[width=0.19\linewidth, clip]{Figs/Figs-sdss/spec-0984-52442-0533-STRIPE82-0108-039253.pdf} & \includegraphics[width=0.19\linewidth, clip]{Figs/Figs-sdss/spec-0990-52465-0605-STRIPE82-0124-071122.pdf} & \includegraphics[width=0.19\linewidth, clip]{Figs/Figs-sdss/spec-1023-52818-0521-STRIPE82-0106-057954.pdf} & \includegraphics[width=0.19\linewidth, clip]{Figs/Figs-sdss/spec-1024-52826-0028-STRIPE82-0109-049060.pdf} \\
    \includegraphics[width=0.19\linewidth, clip]{Figs/Figs-sdss/spec-1033-52822-0623-STRIPE82-0134-016856.pdf} & \includegraphics[width=0.19\linewidth, clip]{Figs/Figs-sdss/spec-1062-52619-0328-STRIPE82-0080-043239.pdf} & \includegraphics[width=0.19\linewidth, clip]{Figs/Figs-sdss/spec-1065-52586-0230-STRIPE82-0071-033558.pdf} & \includegraphics[width=0.19\linewidth, clip]{Figs/Figs-sdss/spec-1067-52616-0258-STRIPE82-0065-022001.pdf} & \includegraphics[width=0.19\linewidth, clip]{Figs/Figs-sdss/spec-1068-52614-0292-STRIPE82-0061-011616.pdf} \\
    \includegraphics[width=0.19\linewidth, clip]{Figs/Figs-sdss/spec-1068-52614-0500-STRIPE82-0064-029944.pdf} & \includegraphics[width=0.19\linewidth, clip]{Figs/Figs-sdss/spec-1069-52590-0193-STRIPE82-0059-041056.pdf} & \includegraphics[width=0.19\linewidth, clip]{Figs/Figs-sdss/spec-1071-52641-0013-STRIPE82-0057-050992.pdf} & \includegraphics[width=0.19\linewidth, clip]{Figs/Figs-sdss/spec-1071-52641-0266-STRIPE82-0053-009717.pdf} & \includegraphics[width=0.19\linewidth, clip]{Figs/Figs-sdss/spec-1071-52641-0358-STRIPE82-0054-042573.pdf} \\
    \includegraphics[width=0.19\linewidth, clip]{Figs/Figs-sdss/spec-1073-52649-0091-STRIPE82-0049-021805.pdf} & \includegraphics[width=0.19\linewidth, clip]{Figs/Figs-sdss/spec-1078-52643-0443-STRIPE82-0034-002402.pdf} & \includegraphics[width=0.19\linewidth, clip]{Figs/Figs-sdss/spec-1096-52974-0283-STRIPE82-0155-009108.pdf} & \includegraphics[width=0.19\linewidth, clip]{Figs/Figs-sdss/spec-1101-52621-0223-SPLUS-s02s12-030410.pdf} & \includegraphics[width=0.19\linewidth, clip]{Figs/Figs-sdss/spec-1102-52883-0155-STRIPE82-0139-036676.pdf} \\
    \includegraphics[width=0.19\linewidth, clip]{Figs/Figs-sdss/spec-1103-52873-0382-STRIPE82-0136-022276.pdf} & \includegraphics[width=0.19\linewidth, clip]{Figs/Figs-sdss/spec-1103-52873-0393-STRIPE82-0136-015923.pdf} & \includegraphics[width=0.19\linewidth, clip]{Figs/Figs-sdss/spec-1106-52912-0102-SPLUS-s02s05-046746.pdf} & \includegraphics[width=0.19\linewidth, clip]{Figs/Figs-sdss/spec-1106-52912-0194-STRIPE82-0129-031160.pdf} & \includegraphics[width=0.19\linewidth, clip]{Figs/Figs-sdss/spec-1106-52912-0456-STRIPE82-0130-042161.pdf} \\
    \includegraphics[width=0.19\linewidth, clip]{Figs/Figs-sdss/spec-1114-53179-0599-STRIPE82-0110-008449.pdf} & \includegraphics[width=0.19\linewidth, clip]{Figs/Figs-sdss/spec-1115-52914-0588-STRIPE82-0108-009933.pdf} & \includegraphics[width=0.19\linewidth, clip]{Figs/Figs-sdss/spec-1116-52932-0478-STRIPE82-0104-027921.pdf} & \includegraphics[width=0.19\linewidth, clip]{Figs/Figs-sdss/spec-1143-52592-0242-SPLUS-s02s08-040033.pdf} & \includegraphics[width=0.19\linewidth, clip]{Figs/Figs-sdss/spec-1144-53238-0450-STRIPE82-0138-042700.pdf} \\
    \includegraphics[width=0.19\linewidth, clip]{Figs/Figs-sdss/spec-1152-52941-0144-STRIPE82-0123-036291.pdf} & \includegraphics[width=0.19\linewidth, clip]{Figs/Figs-sdss/spec-1152-52941-0599-STRIPE82-0123-045029.pdf} & \includegraphics[width=0.19\linewidth, clip]{Figs/Figs-sdss/spec-1157-52643-0059-STRIPE82-0081-029627.pdf} & \includegraphics[width=0.19\linewidth, clip]{Figs/Figs-sdss/spec-1242-52901-0401-STRIPE82-0082-046470.pdf} & \includegraphics[width=0.19\linewidth, clip]{Figs/Figs-sdss/spec-1474-52933-0180-STRIPE82-0129-037205.pdf} \\
    \includegraphics[width=0.19\linewidth, clip]{Figs/Figs-sdss/spec-1474-52933-0189-STRIPE82-0129-024346.pdf} & \includegraphics[width=0.19\linewidth, clip]{Figs/Figs-sdss/spec-1474-52933-0296-STRIPE82-0127-008930.pdf} & \includegraphics[width=0.19\linewidth, clip]{Figs/Figs-sdss/spec-1475-52903-0561-STRIPE82-0134-039543.pdf} & \includegraphics[width=0.19\linewidth, clip]{Figs/Figs-sdss/spec-1476-52964-0017-SPLUS-s02s08-040860.pdf} & \includegraphics[width=0.19\linewidth, clip]{Figs/Figs-sdss/spec-1502-53741-0230-STRIPE82-0035-010718.pdf} \\
    \includegraphics[width=0.19\linewidth, clip]{Figs/Figs-sdss/spec-1509-52942-0633-STRIPE82-0058-020465.pdf} & \includegraphics[width=0.19\linewidth, clip]{Figs/Figs-sdss/spec-1511-52946-0192-STRIPE82-0059-038994.pdf} & \includegraphics[width=0.19\linewidth, clip]{Figs/Figs-sdss/spec-1512-53742-0281-STRIPE82-0061-012261.pdf} & \includegraphics[width=0.19\linewidth, clip]{Figs/Figs-sdss/spec-1512-53742-0296-STRIPE82-0061-011773.pdf} & \includegraphics[width=0.19\linewidth, clip]{Figs/Figs-sdss/spec-1512-53742-0351-STRIPE82-0061-059714.pdf} \\
    \includegraphics[width=0.19\linewidth, clip]{Figs/Figs-sdss/spec-1512-53742-0430-STRIPE82-0064-022463.pdf} & \includegraphics[width=0.19\linewidth, clip]{Figs/Figs-sdss/spec-1512-53742-0471-STRIPE82-0064-008459.pdf} & \includegraphics[width=0.19\linewidth, clip]{Figs/Figs-sdss/spec-1513-53741-0119-STRIPE82-0067-031321.pdf} & \includegraphics[width=0.19\linewidth, clip]{Figs/Figs-sdss/spec-1529-52930-0422-STRIPE82-0082-045627.pdf} & \includegraphics[width=0.19\linewidth, clip]{Figs/Figs-sdss/spec-1558-53271-0059-STRIPE82-0053-022298.pdf} \\
    \includegraphics[width=0.19\linewidth, clip]{Figs/Figs-sdss/spec-3587-55182-0140-STRIPE82-0015-024713.pdf} & \includegraphics[width=0.19\linewidth, clip]{Figs/Figs-sdss/spec-3588-55184-0604-STRIPE82-0016-029684.pdf} & \includegraphics[width=0.19\linewidth, clip]{Figs/Figs-sdss/spec-3589-55186-0260-STRIPE82-0017-024243.pdf} & \includegraphics[width=0.19\linewidth, clip]{Figs/Figs-sdss/spec-3589-55186-0374-STRIPE82-0015-013616.pdf} & \includegraphics[width=0.19\linewidth, clip]{Figs/Figs-sdss/spec-3590-55201-0554-STRIPE82-0018-038274.pdf} \\
    \includegraphics[width=0.19\linewidth, clip]{Figs/Figs-sdss/spec-3609-55201-0610-STRIPE82-0044-021486.pdf} & \includegraphics[width=0.19\linewidth, clip]{Figs/Figs-sdss/spec-3735-55209-0834-STRIPE82-0026-037373.pdf} & \includegraphics[width=0.19\linewidth, clip]{Figs/Figs-sdss/spec-3775-55207-0067-SPLUS-n03s19-045911.pdf} & \includegraphics[width=0.19\linewidth, clip]{Figs/Figs-sdss/spec-3776-55209-0054-SPLUS-n03s23-033092.pdf} & \includegraphics[width=0.19\linewidth, clip]{Figs/Figs-sdss/spec-3776-55209-0768-SPLUS-n02s22-016624.pdf} \\
    \includegraphics[width=0.19\linewidth, clip]{Figs/Figs-sdss/spec-3777-55210-0616-SPLUS-n02s25-003956.pdf} & \includegraphics[width=0.19\linewidth, clip]{Figs/Figs-sdss/spec-3845-55323-0406-SPLUS-n01s24-009687.pdf} & \includegraphics[width=0.19\linewidth, clip]{Figs/Figs-sdss/spec-3847-55588-0744-SPLUS-n02n27-026395.pdf} & \includegraphics[width=0.19\linewidth, clip]{Figs/Figs-sdss/spec-3847-55588-0794-SPLUS-n02n27-022071.pdf} & \includegraphics[width=0.19\linewidth, clip]{Figs/Figs-sdss/spec-4004-55321-0054-SPLUS-n02s37-051418.pdf} \\
    \includegraphics[width=0.19\linewidth, clip]{Figs/Figs-sdss/spec-4037-55631-0460-SPLUS-n02s44-029163.pdf} & \includegraphics[width=0.19\linewidth, clip]{Figs/Figs-sdss/spec-4194-55450-0185-STRIPE82-0119-046591.pdf} & \includegraphics[width=0.19\linewidth, clip]{Figs/Figs-sdss/spec-4194-55450-0940-STRIPE82-0120-007072.pdf} & \includegraphics[width=0.19\linewidth, clip]{Figs/Figs-sdss/spec-4197-55479-0626-STRIPE82-0126-031903.pdf} & \includegraphics[width=0.19\linewidth, clip]{Figs/Figs-sdss/spec-4200-55499-0994-STRIPE82-0136-009636.pdf} \\
    \includegraphics[width=0.19\linewidth, clip]{Figs/Figs-sdss/spec-4202-55445-0409-SPLUS-s02s09-042917.pdf} & \includegraphics[width=0.19\linewidth, clip]{Figs/Figs-sdss/spec-4204-55470-0588-STRIPE82-0142-034401.pdf} & \includegraphics[width=0.19\linewidth, clip]{Figs/Figs-sdss/spec-4206-55471-0352-STRIPE82-0147-043690.pdf} & \includegraphics[width=0.19\linewidth, clip]{Figs/Figs-sdss/spec-4208-55476-0362-STRIPE82-0151-013776.pdf} & \includegraphics[width=0.19\linewidth, clip]{Figs/Figs-sdss/spec-4208-55476-0632-STRIPE82-0152-040956.pdf} \\
    \includegraphics[width=0.19\linewidth, clip]{Figs/Figs-sdss/spec-4209-55478-0069-STRIPE82-0155-044609.pdf} & \includegraphics[width=0.19\linewidth, clip]{Figs/Figs-sdss/spec-4209-55478-0532-STRIPE82-0154-027251.pdf} & \includegraphics[width=0.19\linewidth, clip]{Figs/Figs-sdss/spec-4210-55444-0080-STRIPE82-0159-019011.pdf} & \includegraphics[width=0.19\linewidth, clip]{Figs/Figs-sdss/spec-4213-55449-0038-STRIPE82-0165-038101.pdf} & \includegraphics[width=0.19\linewidth, clip]{Figs/Figs-sdss/spec-4215-55471-0919-STRIPE82-0002-030464.pdf} \\
    \includegraphics[width=0.19\linewidth, clip]{Figs/Figs-sdss/spec-4217-55478-0338-STRIPE82-0003-028018.pdf} & \includegraphics[width=0.19\linewidth, clip]{Figs/Figs-sdss/spec-4217-55478-0796-STRIPE82-0004-029068.pdf} & \includegraphics[width=0.19\linewidth, clip]{Figs/Figs-sdss/spec-4219-55480-0333-STRIPE82-0009-033328.pdf} & \includegraphics[width=0.19\linewidth, clip]{Figs/Figs-sdss/spec-4374-55883-0621-SPLUS-s02s02-030289.pdf} & \includegraphics[width=0.19\linewidth, clip]{Figs/Figs-sdss/spec-4377-55828-0912-SPLUS-s02s07-022389.pdf} \\
    \includegraphics[width=0.19\linewidth, clip]{Figs/Figs-sdss/spec-4380-55857-0774-SPLUS-s02s10-002532.pdf} & \includegraphics[width=0.19\linewidth, clip]{Figs/Figs-sdss/spec-4740-55651-0188-SPLUS-n02n18-039610.pdf} & \includegraphics[width=0.19\linewidth, clip]{Figs/Figs-sdss/spec-4748-55631-0047-SPLUS-n02n25-025655.pdf} & \includegraphics[width=0.19\linewidth, clip]{Figs/Figs-sdss/spec-4778-55706-0848-SPLUS-n05n55-008444.pdf} & \includegraphics[width=0.19\linewidth, clip]{Figs/Figs-sdss/spec-4781-55653-0846-SPLUS-n05n50-021799.pdf} \\
    \includegraphics[width=0.19\linewidth, clip]{Figs/Figs-sdss/spec-6780-56267-0244-STRIPE82-0053-009665.pdf} & \includegraphics[width=0.19\linewidth, clip]{Figs/Figs-sdss/spec-6780-56577-0280-STRIPE82-0053-044359.pdf} & \includegraphics[width=0.19\linewidth, clip]{Figs/Figs-sdss/spec-6781-56274-0268-STRIPE82-0057-039669.pdf} & \includegraphics[width=0.19\linewidth, clip]{Figs/Figs-sdss/spec-6781-56274-0387-STRIPE82-0055-012470.pdf} & \includegraphics[width=0.19\linewidth, clip]{Figs/Figs-sdss/spec-6781-56599-0575-STRIPE82-0056-038512.pdf} \\
    \includegraphics[width=0.19\linewidth, clip]{Figs/Figs-sdss/spec-6782-56572-0334-STRIPE82-0059-014841.pdf} & \includegraphics[width=0.19\linewidth, clip]{Figs/Figs-sdss/spec-7820-56984-0072-STRIPE82-0065-043212.pdf} & \includegraphics[width=0.19\linewidth, clip]{Figs/Figs-sdss/spec-7822-57041-0034-STRIPE82-0061-046539.pdf} & \includegraphics[width=0.19\linewidth, clip]{Figs/Figs-sdss/spec-7822-57041-0278-STRIPE82-0061-047759.pdf} & \includegraphics[width=0.19\linewidth, clip]{Figs/Figs-sdss/spec-7850-56956-0614-STRIPE82-0001-001877.pdf} \\
    \includegraphics[width=0.19\linewidth, clip]{Figs/Figs-sdss/spec-7856-57260-0204-STRIPE82-0018-064634.pdf} & \includegraphics[width=0.19\linewidth, clip]{Figs/Figs-sdss/spec-7864-56979-0753-STRIPE82-0008-034263.pdf} & \includegraphics[width=0.19\linewidth, clip]{Figs/Figs-sdss/spec-7866-57002-0979-STRIPE82-0014-010956.pdf} & \includegraphics[width=0.19\linewidth, clip]{Figs/Figs-sdss/spec-7868-57006-0207-STRIPE82-0015-016996.pdf} & \includegraphics[width=0.19\linewidth, clip]{Figs/Figs-sdss/spec-7868-57006-0691-STRIPE82-0014-014477.pdf} \\
    \includegraphics[width=0.19\linewidth, clip]{Figs/Figs-sdss/spec-9144-57666-0109-STRIPE82-0127-008619.pdf} & \includegraphics[width=0.19\linewidth, clip]{Figs/Figs-sdss/spec-9145-58041-0210-SPLUS-s02s05-026507.pdf} & \includegraphics[width=0.19\linewidth, clip]{Figs/Figs-sdss/spec-9146-58042-0345-SPLUS-s02s07-034127.pdf} & \includegraphics[width=0.19\linewidth, clip]{Figs/Figs-sdss/spec-9146-58042-0510-STRIPE82-0131-050516.pdf} & \includegraphics[width=0.19\linewidth, clip]{Figs/Figs-sdss/spec-9147-58038-0401-SPLUS-s02s08-004413.pdf} \\
    \includegraphics[width=0.19\linewidth, clip]{Figs/Figs-sdss/spec-9150-58043-0660-STRIPE82-0143-028004.pdf} & \includegraphics[width=0.19\linewidth, clip]{Figs/Figs-sdss/spec-9151-58067-0073-STRIPE82-0147-026690.pdf} & \includegraphics[width=0.19\linewidth, clip]{Figs/Figs-sdss/spec-9152-58041-0463-STRIPE82-0147-005730.pdf} & \includegraphics[width=0.19\linewidth, clip]{Figs/Figs-sdss/spec-9154-58013-0749-STRIPE82-0156-010722.pdf} & \includegraphics[width=0.19\linewidth, clip]{Figs/Figs-sdss/spec-9154-58013-0971-STRIPE82-0158-003478.pdf} \\
    \includegraphics[width=0.19\linewidth, clip]{Figs/Figs-sdss/spec-9163-58043-0013-STRIPE82-0122-036068.pdf} & \includegraphics[width=0.19\linewidth, clip]{Figs/Figs-sdss/spec-9163-58043-0440-STRIPE82-0120-045064.pdf} & \includegraphics[width=0.19\linewidth, clip]{Figs/Figs-sdss/spec-9166-58051-0118-STRIPE82-0132-037548.pdf} & \includegraphics[width=0.19\linewidth, clip]{Figs/Figs-sdss/spec-9168-58067-0396-STRIPE82-0136-026983.pdf} & \includegraphics[width=0.19\linewidth, clip]{Figs/Figs-sdss/spec-9172-58015-0449-STRIPE82-0148-018355.pdf} \\
    \includegraphics[width=0.19\linewidth, clip]{Figs/Figs-sdss/spec-9180-57693-0793-STRIPE82-0120-000615.pdf} & \includegraphics[width=0.19\linewidth, clip]{Figs/Figs-sdss/spec-9180-57693-0928-STRIPE82-0121-040706.pdf} & \includegraphics[width=0.19\linewidth, clip]{Figs/Figs-sdss/spec-9198-57713-0286-STRIPE82-0169-028275.pdf} & \includegraphics[width=0.19\linewidth, clip]{Figs/Figs-sdss/spec-9201-57724-0285-STRIPE82-0164-020336.pdf} & \includegraphics[width=0.19\linewidth, clip]{Figs/Figs-sdss/spec-9204-57712-0040-STRIPE82-0156-034804.pdf} \\
    \includegraphics[width=0.19\linewidth, clip]{Figs/Figs-sdss/spec-9207-57667-0050-STRIPE82-0148-001139.pdf} & \includegraphics[width=0.19\linewidth, clip]{Figs/Figs-sdss/spec-9210-57656-0552-STRIPE82-0161-039767.pdf} & \includegraphics[width=0.19\linewidth, clip]{Figs/Figs-sdss/spec-9392-58104-0419-STRIPE82-0026-005547.pdf} & \includegraphics[width=0.19\linewidth, clip]{Figs/Figs-sdss/spec-9395-58113-0530-STRIPE82-0032-003020.pdf} & \includegraphics[width=0.19\linewidth, clip]{Figs/Figs-sdss/spec-9403-58018-0690-STRIPE82-0002-034805.pdf} \\
    \includegraphics[width=0.19\linewidth, clip]{Figs/Figs-sdss/spec-9405-58048-0414-STRIPE82-0009-022679.pdf} & \includegraphics[width=0.19\linewidth, clip]{Figs/Figs-sdss/spec-9406-58067-0121-STRIPE82-0007-021121.pdf} & \includegraphics[width=0.19\linewidth, clip]{Figs/Figs-sdss/spec-9406-58067-0648-STRIPE82-0006-027457.pdf} & \includegraphics[width=0.19\linewidth, clip]{Figs/Figs-sdss/spec-9406-58067-0764-STRIPE82-0008-036347.pdf} & \includegraphics[width=0.19\linewidth, clip]{Figs/Figs-sdss/spec-9407-58041-0415-STRIPE82-0004-031191.pdf} \\
    \includegraphics[width=0.19\linewidth, clip]{Figs/Figs-sdss/spec-9409-58051-0225-STRIPE82-0015-053521.pdf} & \includegraphics[width=0.19\linewidth, clip]{Figs/Figs-sdss/spec-9409-58051-0529-STRIPE82-0014-019228.pdf} & \includegraphics[width=0.19\linewidth, clip]{Figs/Figs-sdss/spec-0330-52370-0471-SPLUS-n03s21-043085.pdf} & \includegraphics[width=0.19\linewidth, clip]{Figs/Figs-sdss/spec-0331-52368-0215-SPLUS-n03s23-001039.pdf} & \includegraphics[width=0.19\linewidth, clip]{Figs/Figs-sdss/spec-0331-52368-0449-SPLUS-n02s23-034336.pdf} \\
    \includegraphics[width=0.19\linewidth, clip]{Figs/Figs-sdss/spec-0332-52367-0306-SPLUS-n03s23-034002.pdf} & \includegraphics[width=0.19\linewidth, clip]{Figs/Figs-sdss/spec-0334-51993-0065-SPLUS-n03s28-019988.pdf} & \includegraphics[width=0.19\linewidth, clip]{Figs/Figs-sdss/spec-0334-51993-0365-SPLUS-n02s27-030519.pdf} & \includegraphics[width=0.19\linewidth, clip]{Figs/Figs-sdss/spec-0334-51993-0443-SPLUS-n02s28-028453.pdf} & \includegraphics[width=0.19\linewidth, clip]{Figs/Figs-sdss/spec-0371-52078-0576-STRIPE82-0128-050321.pdf} \\
    \includegraphics[width=0.19\linewidth, clip]{Figs/Figs-sdss/spec-0372-52173-0286-STRIPE82-0127-009047.pdf} & \includegraphics[width=0.19\linewidth, clip]{Figs/Figs-sdss/spec-0372-52173-0296-SPLUS-s02s04-032140.pdf} & \includegraphics[width=0.19\linewidth, clip]{Figs/Figs-sdss/spec-0373-51788-0507-STRIPE82-0132-025641.pdf} & \includegraphics[width=0.19\linewidth, clip]{Figs/Figs-sdss/spec-0376-52143-0160-STRIPE82-0139-046699.pdf} & \includegraphics[width=0.19\linewidth, clip]{Figs/Figs-sdss/spec-0376-52143-0631-STRIPE82-0142-027354.pdf} \\
    \includegraphics[width=0.19\linewidth, clip]{Figs/Figs-sdss/spec-0377-52145-0294-STRIPE82-0141-019376.pdf} & \includegraphics[width=0.19\linewidth, clip]{Figs/Figs-sdss/spec-0377-52145-0484-STRIPE82-0142-017985.pdf} & \includegraphics[width=0.19\linewidth, clip]{Figs/Figs-sdss/spec-0378-52146-0223-STRIPE82-0145-047131.pdf} & \includegraphics[width=0.19\linewidth, clip]{Figs/Figs-sdss/spec-0380-51792-0575-STRIPE82-0152-056707.pdf} & \includegraphics[width=0.19\linewidth, clip]{Figs/Figs-sdss/spec-0381-51811-0086-STRIPE82-0153-010058.pdf} \\
    \includegraphics[width=0.19\linewidth, clip]{Figs/Figs-sdss/spec-0383-51818-0011-STRIPE82-0159-008213.pdf} & \includegraphics[width=0.19\linewidth, clip]{Figs/Figs-sdss/spec-0383-51818-0150-STRIPE82-0159-029883.pdf} & \includegraphics[width=0.19\linewidth, clip]{Figs/Figs-sdss/spec-0383-51818-0290-SPLUS-s02s20-036837.pdf} & \includegraphics[width=0.19\linewidth, clip]{Figs/Figs-sdss/spec-0384-51821-0564-STRIPE82-0164-061328.pdf} & \includegraphics[width=0.19\linewidth, clip]{Figs/Figs-sdss/spec-0399-51817-0324-STRIPE82-0032-035005.pdf} \\
    \includegraphics[width=0.19\linewidth, clip]{Figs/Figs-sdss/spec-0399-51817-0477-STRIPE82-0032-012194.pdf} & \includegraphics[width=0.19\linewidth, clip]{Figs/Figs-sdss/spec-0406-51817-0160-STRIPE82-0052-004127.pdf} & \includegraphics[width=0.19\linewidth, clip]{Figs/Figs-sdss/spec-0406-51900-0604-STRIPE82-0054-040145.pdf} & \includegraphics[width=0.19\linewidth, clip]{Figs/Figs-sdss/spec-0408-51821-0296-STRIPE82-0057-019420.pdf} & \includegraphics[width=0.19\linewidth, clip]{Figs/Figs-sdss/spec-0409-51871-0225-STRIPE82-0059-048815.pdf} \\
    \includegraphics[width=0.19\linewidth, clip]{Figs/Figs-sdss/spec-0409-51871-0596-STRIPE82-0062-010774.pdf} & \includegraphics[width=0.19\linewidth, clip]{Figs/Figs-sdss/spec-0410-51877-0217-STRIPE82-0063-024542.pdf} & \includegraphics[width=0.19\linewidth, clip]{Figs/Figs-sdss/spec-0410-51877-0492-STRIPE82-0064-042346.pdf} & \includegraphics[width=0.19\linewidth, clip]{Figs/Figs-sdss/spec-0410-51877-0519-STRIPE82-0064-031755.pdf} & \includegraphics[width=0.19\linewidth, clip]{Figs/Figs-sdss/spec-0410-51877-0527-STRIPE82-0064-069216.pdf} \\
    \includegraphics[width=0.19\linewidth, clip]{Figs/Figs-sdss/spec-0411-51817-0153-STRIPE82-0067-050741.pdf} & \includegraphics[width=0.19\linewidth, clip]{Figs/Figs-sdss/spec-0411-51817-0159-STRIPE82-0067-036513.pdf} & \includegraphics[width=0.19\linewidth, clip]{Figs/Figs-sdss/spec-0411-51817-0578-STRIPE82-0068-044762.pdf} & \includegraphics[width=0.19\linewidth, clip]{Figs/Figs-sdss/spec-0411-51817-0590-STRIPE82-0068-034611.pdf} & \includegraphics[width=0.19\linewidth, clip]{Figs/Figs-sdss/spec-0413-51821-0590-STRIPE82-0074-040778.pdf} \\
    \includegraphics[width=0.19\linewidth, clip]{Figs/Figs-sdss/spec-0415-51810-0420-STRIPE82-0078-040926.pdf} & \includegraphics[width=0.19\linewidth, clip]{Figs/Figs-sdss/spec-0416-51811-0065-STRIPE82-0081-036966.pdf} & \includegraphics[width=0.19\linewidth, clip]{Figs/Figs-sdss/spec-0416-51811-0427-STRIPE82-0080-021849.pdf} & \includegraphics[width=0.19\linewidth, clip]{Figs/Figs-sdss/spec-0416-51811-0537-STRIPE82-0080-039699.pdf} & \includegraphics[width=0.19\linewidth, clip]{Figs/Figs-sdss/spec-0501-52235-0474-SPLUS-n03n01-003922.pdf} \\
    \includegraphics[width=0.19\linewidth, clip]{Figs/Figs-sdss/spec-0514-51994-0016-SPLUS-n02n21-042463.pdf} & \includegraphics[width=0.19\linewidth, clip]{Figs/Figs-sdss/spec-0687-52518-0113-STRIPE82-0008-003185.pdf} & \includegraphics[width=0.19\linewidth, clip]{Figs/Figs-sdss/spec-0687-52518-0444-STRIPE82-0008-056238.pdf} & \includegraphics[width=0.19\linewidth, clip]{Figs/Figs-sdss/spec-0687-52518-0474-STRIPE82-0008-013742.pdf} & \includegraphics[width=0.19\linewidth, clip]{Figs/Figs-sdss/spec-0688-52203-0315-STRIPE82-0007-044146.pdf} \\
    \includegraphics[width=0.19\linewidth, clip]{Figs/Figs-sdss/spec-0689-52262-0087-STRIPE82-0013-019963.pdf} & \includegraphics[width=0.19\linewidth, clip]{Figs/Figs-sdss/spec-0689-52262-0247-STRIPE82-0011-015494.pdf} & \includegraphics[width=0.19\linewidth, clip]{Figs/Figs-sdss/spec-0689-52262-0468-STRIPE82-0012-027933.pdf} & \includegraphics[width=0.19\linewidth, clip]{Figs/Figs-sdss/spec-0691-52199-0602-STRIPE82-0020-053566.pdf} & \includegraphics[width=0.19\linewidth, clip]{Figs/Figs-sdss/spec-0692-52201-0083-STRIPE82-0021-022763.pdf} \\
    \includegraphics[width=0.19\linewidth, clip]{Figs/Figs-sdss/spec-0692-52201-0129-STRIPE82-0021-016992.pdf} & \includegraphics[width=0.19\linewidth, clip]{Figs/Figs-sdss/spec-0692-52201-0240-STRIPE82-0019-035399.pdf} & \includegraphics[width=0.19\linewidth, clip]{Figs/Figs-sdss/spec-0710-52203-0111-STRIPE82-0069-058305.pdf} & \includegraphics[width=0.19\linewidth, clip]{Figs/Figs-sdss/spec-0710-52203-0439-STRIPE82-0068-006406.pdf} & \includegraphics[width=0.19\linewidth, clip]{Figs/Figs-sdss/spec-0710-52203-0620-STRIPE82-0070-026450.pdf} \\
    \includegraphics[width=0.19\linewidth, clip]{Figs/Figs-sdss/spec-0711-52202-0064-STRIPE82-0071-035792.pdf} & \includegraphics[width=0.19\linewidth, clip]{Figs/Figs-sdss/spec-0713-52178-0207-STRIPE82-0077-010093.pdf} & \includegraphics[width=0.19\linewidth, clip]{Figs/Figs-sdss/spec-0714-52201-0033-STRIPE82-0081-053377.pdf} & \includegraphics[width=0.19\linewidth, clip]{Figs/Figs-sdss/spec-0714-52201-0077-STRIPE82-0081-034915.pdf} & \includegraphics[width=0.19\linewidth, clip]{Figs/Figs-sdss/spec-0802-52289-0276-STRIPE82-0065-022257.pdf} \\
    \includegraphics[width=0.19\linewidth, clip]{Figs/Figs-sdss/spec-0804-52286-0305-STRIPE82-0069-065782.pdf} & \includegraphics[width=0.19\linewidth, clip]{Figs/Figs-sdss/spec-0810-52672-0443-STRIPE82-0078-005601.pdf} & \includegraphics[width=0.19\linewidth, clip]{Figs/Figs-sdss/spec-0914-52721-0103-SPLUS-n03s44-026823.pdf} & \includegraphics[width=0.19\linewidth, clip]{Figs/Figs-sdss/spec-1079-52621-0370-STRIPE82-0032-035288.pdf} & \includegraphics[width=0.19\linewidth, clip]{Figs/Figs-sdss/spec-1079-52621-0506-STRIPE82-0032-023385.pdf} \\
    \includegraphics[width=0.19\linewidth, clip]{Figs/Figs-sdss/spec-1083-52520-0185-STRIPE82-0021-044621.pdf} & \includegraphics[width=0.19\linewidth, clip]{Figs/Figs-sdss/spec-1085-52531-0175-STRIPE82-0017-027488.pdf} & \includegraphics[width=0.19\linewidth, clip]{Figs/Figs-sdss/spec-1086-52525-0146-STRIPE82-0013-036763.pdf} & \includegraphics[width=0.19\linewidth, clip]{Figs/Figs-sdss/spec-1089-52913-0196-STRIPE82-0007-024265.pdf} & \includegraphics[width=0.19\linewidth, clip]{Figs/Figs-sdss/spec-1089-52913-0199-STRIPE82-0007-023673.pdf} \\
    \includegraphics[width=0.19\linewidth, clip]{Figs/Figs-sdss/spec-1089-52913-0386-STRIPE82-0006-034786.pdf} & \includegraphics[width=0.19\linewidth, clip]{Figs/Figs-sdss/spec-1090-52903-0539-STRIPE82-0006-015081.pdf} & \includegraphics[width=0.19\linewidth, clip]{Figs/Figs-sdss/spec-1094-52524-0594-STRIPE82-0164-020807.pdf} & \includegraphics[width=0.19\linewidth, clip]{Figs/Figs-sdss/spec-1095-52521-0198-STRIPE82-0159-019110.pdf} & \includegraphics[width=0.19\linewidth, clip]{Figs/Figs-sdss/spec-1487-52964-0344-STRIPE82-0164-020337.pdf} \\
    \includegraphics[width=0.19\linewidth, clip]{Figs/Figs-sdss/spec-1492-52932-0553-STRIPE82-0010-005855.pdf} & \includegraphics[width=0.19\linewidth, clip]{Figs/Figs-sdss/spec-1493-52933-0207-STRIPE82-0011-008529.pdf} & \includegraphics[width=0.19\linewidth, clip]{Figs/Figs-sdss/spec-1494-52937-0249-STRIPE82-0013-010873.pdf} & \includegraphics[width=0.19\linewidth, clip]{Figs/Figs-sdss/spec-1498-52914-0576-STRIPE82-0026-054031.pdf} & \includegraphics[width=0.19\linewidth, clip]{Figs/Figs-sdss/spec-1499-53001-0009-STRIPE82-0027-023732.pdf} \\
    \includegraphics[width=0.19\linewidth, clip]{Figs/Figs-sdss/spec-1499-53001-0014-STRIPE82-0027-022810.pdf} & \includegraphics[width=0.19\linewidth, clip]{Figs/Figs-sdss/spec-1499-53001-0136-STRIPE82-0027-016716.pdf} & \includegraphics[width=0.19\linewidth, clip]{Figs/Figs-sdss/spec-1499-53001-0395-STRIPE82-0026-006093.pdf} & \includegraphics[width=0.19\linewidth, clip]{Figs/Figs-sdss/spec-1499-53001-0459-STRIPE82-0028-044679.pdf} & \includegraphics[width=0.19\linewidth, clip]{Figs/Figs-sdss/spec-1499-53001-0525-STRIPE82-0028-028483.pdf} \\
    \includegraphics[width=0.19\linewidth, clip]{Figs/Figs-sdss/spec-1632-52996-0497-STRIPE82-0080-018211.pdf} & \includegraphics[width=0.19\linewidth, clip]{Figs/Figs-sdss/spec-1827-53531-0289-SPLUS-n05n50-033851.pdf} & \includegraphics[width=0.19\linewidth, clip]{Figs/Figs-sdss/spec-1829-53494-0276-SPLUS-n05n53-037456.pdf} & \includegraphics[width=0.19\linewidth, clip]{Figs/Figs-sdss/spec-1829-53494-0282-SPLUS-n05n53-005761.pdf} & \includegraphics[width=0.19\linewidth, clip]{Figs/Figs-sdss/spec-1901-53261-0086-SPLUS-s02s13-032892.pdf} \\
    \includegraphics[width=0.19\linewidth, clip]{Figs/Figs-sdss/spec-1919-53240-0073-STRIPE82-0111-068928.pdf} & \includegraphics[width=0.19\linewidth, clip]{Figs/Figs-sdss/spec-1919-53240-0621-STRIPE82-0112-020655.pdf} & \includegraphics[width=0.19\linewidth, clip]{Figs/Figs-sdss/spec-2568-54153-0481-SPLUS-n01s27-050095.pdf} & \includegraphics[width=0.19\linewidth, clip]{Figs/Figs-sdss/spec-2636-54082-0584-STRIPE82-0054-038222.pdf} & \includegraphics[width=0.19\linewidth, clip]{Figs/Figs-sdss/spec-2892-54552-0550-SPLUS-n01s24-037959.pdf} \\
    \includegraphics[width=0.19\linewidth, clip]{Figs/Figs-sdss/spec-3156-54792-0288-STRIPE82-0077-052793.pdf} & \includegraphics[width=0.19\linewidth, clip]{Figs/Figs-sdss/spec-0914-52721-0431-SPLUS-n03s43-048919.pdf} & \includegraphics[width=0.19\linewidth, clip]{Figs/Figs-sdss/spec-1035-52816-0289-SPLUS-s02s08-035461.pdf} & \includegraphics[width=0.19\linewidth, clip]{Figs/Figs-sdss/spec-1079-52621-0360-STRIPE82-0030-004872.pdf} & \includegraphics[width=0.19\linewidth, clip]{Figs/Figs-sdss/spec-1095-52521-0378-STRIPE82-0160-038992.pdf} \\
    \includegraphics[width=0.19\linewidth, clip]{Figs/Figs-sdss/spec-1117-52885-0395-STRIPE82-0100-021070.pdf} & \includegraphics[width=0.19\linewidth, clip]{Figs/Figs-sdss/spec-1486-52993-0227-STRIPE82-0161-024721.pdf} & \includegraphics[width=0.19\linewidth, clip]{Figs/Figs-sdss/spec-1500-53730-0610-STRIPE82-0032-034096.pdf} & \includegraphics[width=0.19\linewidth, clip]{Figs/Figs-sdss/spec-1632-52996-0079-STRIPE82-0081-042693.pdf} & \includegraphics[width=0.19\linewidth, clip]{Figs/Figs-sdss/spec-3183-54833-0117-STRIPE82-0071-040649.pdf} \\
    \includegraphics[width=0.19\linewidth, clip]{Figs/Figs-sdss/spec-3842-55565-0582-SPLUS-n02n20-021184.pdf} & \includegraphics[width=0.19\linewidth, clip]{Figs/Figs-sdss/spec-4205-55454-0490-STRIPE82-0143-027848.pdf} & \includegraphics[width=0.19\linewidth, clip]{Figs/Figs-sdss/spec-4220-55447-0004-STRIPE82-0013-055791.pdf} & \includegraphics[width=0.19\linewidth, clip]{Figs/Figs-sdss/spec-4344-55557-0830-STRIPE82-0055-005329.pdf} & \includegraphics[width=0.19\linewidth, clip]{Figs/Figs-sdss/spec-6781-56274-0507-STRIPE82-0056-011736.pdf} \\
    \includegraphics[width=0.19\linewidth, clip]{Figs/Figs-sdss/spec-7870-57016-0292-STRIPE82-0017-021131.pdf} & \includegraphics[width=0.19\linewidth, clip]{Figs/Figs-sdss/spec-9144-57666-0046-SPLUS-s02s04-008782.pdf} & \includegraphics[width=0.19\linewidth, clip]{Figs/Figs-sdss/spec-9163-58043-0152-STRIPE82-0122-053411.pdf} & \includegraphics[width=0.19\linewidth, clip]{Figs/Figs-sdss/spec-9210-57656-0673-STRIPE82-0163-050068.pdf} & \includegraphics[width=0.19\linewidth, clip]{Figs/Figs-sdss/spec-9391-58072-0653-STRIPE82-0027-025274.pdf} \\
    \includegraphics[width=0.19\linewidth, clip]{Figs/Figs-sdss/spec-0709-52205-0048-STRIPE82-0065-009796.pdf} & \includegraphics[width=0.19\linewidth, clip]{Figs/Figs-sdss/spec-4226-55475-0425-STRIPE82-0023-017707.pdf} & \includegraphics[width=0.19\linewidth, clip]{Figs/Figs-sdss/spec-4227-55481-0648-STRIPE82-0028-008666.pdf} & \includegraphics[width=0.19\linewidth, clip]{Figs/Figs-sdss/spec-4229-55501-0142-STRIPE82-0033-040341.pdf} & \includegraphics[width=0.19\linewidth, clip]{Figs/Figs-sdss/spec-4229-55501-0512-STRIPE82-0032-003058.pdf} \\
    \includegraphics[width=0.19\linewidth, clip]{Figs/Figs-sdss/spec-4234-55478-0651-STRIPE82-0044-019857.pdf} & \includegraphics[width=0.19\linewidth, clip]{Figs/Figs-sdss/spec-4236-55479-0206-STRIPE82-0049-010030.pdf} & \includegraphics[width=0.19\linewidth, clip]{Figs/Figs-sdss/spec-4237-55478-0412-STRIPE82-0049-025567.pdf} & \includegraphics[width=0.19\linewidth, clip]{Figs/Figs-sdss/spec-4240-55455-0669-STRIPE82-0058-035027.pdf} & \includegraphics[width=0.19\linewidth, clip]{Figs/Figs-sdss/spec-4241-55450-0894-STRIPE82-0062-010390.pdf} \\
    \includegraphics[width=0.19\linewidth, clip]{Figs/Figs-sdss/spec-4288-55501-0044-STRIPE82-0156-056121.pdf} & \includegraphics[width=0.19\linewidth, clip]{Figs/Figs-sdss/spec-4302-55531-0350-STRIPE82-0012-042042.pdf} & \includegraphics[width=0.19\linewidth, clip]{Figs/Figs-sdss/spec-7872-57279-0585-STRIPE82-0020-062391.pdf} & \includegraphics[width=0.19\linewidth, clip]{Figs/Figs-sdss/spec-7873-57307-0886-STRIPE82-0023-005307.pdf} & \includegraphics[width=0.19\linewidth, clip]{Figs/Figs-sdss/spec-7876-57002-0548-STRIPE82-0028-032235.pdf} \\
    \includegraphics[width=0.19\linewidth, clip]{Figs/Figs-sdss/spec-7876-57002-0855-STRIPE82-0030-011791.pdf} & \includegraphics[width=0.19\linewidth, clip]{Figs/Figs-sdss/spec-8789-57358-0239-STRIPE82-0027-043701.pdf} & \includegraphics[width=0.19\linewidth, clip]{Figs/Figs-sdss/spec-8789-57358-0340-STRIPE82-0025-031694.pdf} & \includegraphics[width=0.19\linewidth, clip]{Figs/Figs-sdss/spec-8789-57358-0634-STRIPE82-0026-041908.pdf} & \includegraphics[width=0.19\linewidth, clip]{Figs/Figs-sdss/spec-8789-57358-0827-STRIPE82-0028-032669.pdf} \\
    \includegraphics[width=0.19\linewidth, clip]{Figs/Figs-sdss/spec-8790-57363-0208-STRIPE82-0029-026431.pdf} & \includegraphics[width=0.19\linewidth, clip]{Figs/Figs-sdss/spec-8790-57363-0451-STRIPE82-0027-022871.pdf} & \includegraphics[width=0.19\linewidth, clip]{Figs/Figs-sdss/spec-8791-57373-0951-STRIPE82-0034-031629.pdf} & \includegraphics[width=0.19\linewidth, clip]{Figs/Figs-sdss/spec-9213-57712-0731-STRIPE82-0155-024313.pdf} & \includegraphics[width=0.19\linewidth, clip]{Figs/Figs-sdss/spec-9214-57684-0636-STRIPE82-0151-054324.pdf} \\
    \includegraphics[width=0.19\linewidth, clip]{Figs/Figs-sdss/spec-9214-57684-0949-STRIPE82-0154-001350.pdf} & \includegraphics[width=0.19\linewidth, clip]{Figs/Figs-sdss/spec-9217-57934-0197-STRIPE82-0143-016256.pdf} & \includegraphics[width=0.19\linewidth, clip]{Figs/Figs-sdss/spec-9217-57934-0839-STRIPE82-0143-016137.pdf} & \includegraphics[width=0.19\linewidth, clip]{Figs/Figs-sdss/spec-9218-57724-0804-STRIPE82-0145-045634.pdf} & \includegraphics[width=0.19\linewidth, clip]{Figs/Figs-sdss/spec-9230-58021-0675-STRIPE82-0125-053114.pdf} \\
    \includegraphics[width=0.19\linewidth, clip]{Figs/Figs-sdss/spec-9238-58013-0137-SPLUS-s02s09-056401.pdf} & \includegraphics[width=0.19\linewidth, clip]{Figs/Figs-sdss/spec-9239-58018-0729-STRIPE82-0136-042690.pdf} & \includegraphics[width=0.19\linewidth, clip]{Figs/Figs-sdss/spec-9383-58097-0530-STRIPE82-0047-031879.pdf} & \includegraphics[width=0.19\linewidth, clip]{Figs/Figs-sdss/spec-9388-58074-0598-STRIPE82-0023-031284.pdf} \\
  \end{longtable}
\end{center}


\clearpage
\section{Lamost spectra}
\begin{center}
  \begin{longtable}{l l l l l }
  \caption{Espectra from LAMOST DR6 \label{tab:spec-lamost}}\
  \endfirsthead
  \caption[]{--continued}\\
  \endhead
  \hline \endfoot
    \includegraphics[width=0.19\linewidth, clip]{Figs/Figs-lamost/spec-55859-F5902_sp13-002-STRIPE82-0136-042690.pdf} & \includegraphics[width=0.19\linewidth, clip]{Figs/Figs-lamost/spec-55859-F5907_sp03-141-STRIPE82-0063-036294.pdf} & \includegraphics[width=0.19\linewidth, clip]{Figs/Figs-lamost/spec-55859-F5907_sp04-234-STRIPE82-0065-022216.pdf} & \includegraphics[width=0.19\linewidth, clip]{Figs/Figs-lamost/spec-55859-F5907_sp09-110-STRIPE82-0066-006705.pdf} & \includegraphics[width=0.19\linewidth, clip]{Figs/Figs-lamost/spec-55859-F5907_sp13-126-STRIPE82-0068-006406.pdf} \\
    \includegraphics[width=0.19\linewidth, clip]{Figs/Figs-lamost/spec-55859-F5907_sp13-145-STRIPE82-0067-049701.pdf} & \includegraphics[width=0.19\linewidth, clip]{Figs/Figs-lamost/spec-55859-F5907_sp14-024-STRIPE82-0062-010390.pdf} & \includegraphics[width=0.19\linewidth, clip]{Figs/Figs-lamost/spec-55859-F5907_sp15-171-STRIPE82-0064-031755.pdf} & \includegraphics[width=0.19\linewidth, clip]{Figs/Figs-lamost/spec-55859-F5907_sp16-143-STRIPE82-0064-069216.pdf} & \includegraphics[width=0.19\linewidth, clip]{Figs/Figs-lamost/spec-55893-F9304_sp12-121-STRIPE82-0059-052699.pdf} \\
    \includegraphics[width=0.19\linewidth, clip]{Figs/Figs-lamost/spec-55915-F5591503_sp12-098-STRIPE82-0063-036294.pdf} & \includegraphics[width=0.19\linewidth, clip]{Figs/Figs-lamost/spec-55915-F5591503_sp13-115-STRIPE82-0065-022216.pdf} & \includegraphics[width=0.19\linewidth, clip]{Figs/Figs-lamost/spec-55916-F5591602_sp01-125-STRIPE82-0042-018967.pdf} & \includegraphics[width=0.19\linewidth, clip]{Figs/Figs-lamost/spec-55916-F5591602_sp02-136-STRIPE82-0040-030039.pdf} & \includegraphics[width=0.19\linewidth, clip]{Figs/Figs-lamost/spec-55920-F5592001_sp09-085-STRIPE82-0032-003020.pdf} \\
    \includegraphics[width=0.19\linewidth, clip]{Figs/Figs-lamost/spec-55920-F5592001_sp09-232-STRIPE82-0031-003242.pdf} & \includegraphics[width=0.19\linewidth, clip]{Figs/Figs-lamost/spec-55920-F5592001_sp15-046-STRIPE82-0030-004872.pdf} & \includegraphics[width=0.19\linewidth, clip]{Figs/Figs-lamost/spec-55920-F5592001_sp16-128-STRIPE82-0030-034826.pdf} & \includegraphics[width=0.19\linewidth, clip]{Figs/Figs-lamost/spec-55973-F5597306_sp14-196-SPLUS-n02s22-039680.pdf} & \includegraphics[width=0.19\linewidth, clip]{Figs/Figs-lamost/spec-55973-F5597306_sp15-212-SPLUS-n01s23-035079.pdf} \\
    \includegraphics[width=0.19\linewidth, clip]{Figs/Figs-lamost/spec-56012-F5601204_sp14-071-SPLUS-n01s19-035613.pdf} & \includegraphics[width=0.19\linewidth, clip]{Figs/Figs-lamost/spec-56218-EG034047S021957B01_sp12-127-STRIPE82-0081-034915.pdf} & \includegraphics[width=0.19\linewidth, clip]{Figs/Figs-lamost/spec-56218-EG213945N020821B01_sp02-208-STRIPE82-0120-007072.pdf} & \includegraphics[width=0.19\linewidth, clip]{Figs/Figs-lamost/spec-56218-EG213945N020821B01_sp10-219-STRIPE82-0120-045064.pdf} & \includegraphics[width=0.19\linewidth, clip]{Figs/Figs-lamost/spec-56218-EG213945N020821M01_sp02-208-STRIPE82-0120-007072.pdf} \\
    \includegraphics[width=0.19\linewidth, clip]{Figs/Figs-lamost/spec-56218-EG213945N020821M01_sp05-072-STRIPE82-0122-053411.pdf} & \includegraphics[width=0.19\linewidth, clip]{Figs/Figs-lamost/spec-56285-HD122600N020231M01_sp08-053-SPLUS-n02n28-045059.pdf} & \includegraphics[width=0.19\linewidth, clip]{Figs/Figs-lamost/spec-56285-HD122600N020231M01_sp10-243-SPLUS-n02n26-033149.pdf} & \includegraphics[width=0.19\linewidth, clip]{Figs/Figs-lamost/spec-56591-EG012606S021203F01_sp09-053-STRIPE82-0033-008289.pdf} & \includegraphics[width=0.19\linewidth, clip]{Figs/Figs-lamost/spec-55893-F9304_sp15-076-STRIPE82-0057-019420.pdf} \\
    \includegraphics[width=0.19\linewidth, clip]{Figs/Figs-lamost/spec-56012-F5601204_sp13-173-SPLUS-n01s22-036940.pdf} & \includegraphics[width=0.19\linewidth, clip]{Figs/Figs-lamost/spec-56591-EG012606S021203F01_sp16-150-STRIPE82-0031-049213.pdf} & \includegraphics[width=0.19\linewidth, clip]{Figs/Figs-lamost/spec-56741-HD115451S012705B_sp12-016-SPLUS-n01s23-026372.pdf} & \includegraphics[width=0.19\linewidth, clip]{Figs/Figs-lamost/spec-56948-EG212551N003203M01_sp10-202-STRIPE82-0115-048674.pdf} & \includegraphics[width=0.19\linewidth, clip]{Figs/Figs-lamost/spec-57304-EG230517N011825M01_sp02-068-STRIPE82-0151-054324.pdf} \\
    \includegraphics[width=0.19\linewidth, clip]{Figs/Figs-lamost/spec-57330-EG220307N031056M01_sp07-205-STRIPE82-0132-050056.pdf} & \includegraphics[width=0.19\linewidth, clip]{Figs/Figs-lamost/spec-57454-HD120800N003716B02_sp08-055-SPLUS-n01s25-028549.pdf} & \includegraphics[width=0.19\linewidth, clip]{Figs/Figs-lamost/spec-57747-EG020648N012631M01_sp01-190-STRIPE82-0047-013896.pdf} & \includegraphics[width=0.19\linewidth, clip]{Figs/Figs-lamost/spec-56600-EG012606S021203F02_sp16-150-STRIPE82-0031-049213.pdf} & \includegraphics[width=0.19\linewidth, clip]{Figs/Figs-lamost/spec-56604-EG012606S021203F04_sp16-150-STRIPE82-0031-049213.pdf} \\
    \includegraphics[width=0.19\linewidth, clip]{Figs/Figs-lamost/spec-56632-EG012606S021203F05_sp16-150-STRIPE82-0031-049213.pdf} & \includegraphics[width=0.19\linewidth, clip]{Figs/Figs-lamost/spec-56654-HD120800N003716B01_sp02-154-SPLUS-n02s22-039680.pdf} & \includegraphics[width=0.19\linewidth, clip]{Figs/Figs-lamost/spec-56656-HD120800N003716M01_sp04-046-SPLUS-n02n24-020916.pdf} & \includegraphics[width=0.19\linewidth, clip]{Figs/Figs-lamost/spec-56656-HD120800N003716M01_sp14-201-SPLUS-n02n23-022190.pdf} & \includegraphics[width=0.19\linewidth, clip]{Figs/Figs-lamost/spec-56741-HD115451S012705B_sp01-058-SPLUS-n03s21-009182.pdf} \\
    \includegraphics[width=0.19\linewidth, clip]{Figs/Figs-lamost/spec-56741-HD115451S012705B_sp07-217-SPLUS-n03s23-001039.pdf} & \includegraphics[width=0.19\linewidth, clip]{Figs/Figs-lamost/spec-56741-HD115451S012705B_sp11-044-SPLUS-n02n22-000168.pdf} & \includegraphics[width=0.19\linewidth, clip]{Figs/Figs-lamost/spec-56741-HD115451S012705B_sp13-174-SPLUS-n02s22-039680.pdf} & \includegraphics[width=0.19\linewidth, clip]{Figs/Figs-lamost/spec-56741-HD115451S012705B_sp15-056-SPLUS-n01s21-007449.pdf} & \includegraphics[width=0.19\linewidth, clip]{Figs/Figs-lamost/spec-56749-HD123204S014620M01_sp16-101-SPLUS-n01s28-027608.pdf} \\
    \includegraphics[width=0.19\linewidth, clip]{Figs/Figs-lamost/spec-56944-EG025338N015809M01_sp01-052-STRIPE82-0062-010774.pdf} & \includegraphics[width=0.19\linewidth, clip]{Figs/Figs-lamost/spec-56944-EG025338N015809M01_sp08-159-STRIPE82-0064-069216.pdf} & \includegraphics[width=0.19\linewidth, clip]{Figs/Figs-lamost/spec-56944-EG025338N015809M01_sp08-213-STRIPE82-0064-031755.pdf} & \includegraphics[width=0.19\linewidth, clip]{Figs/Figs-lamost/spec-56945-EG233528N011847B01_sp02-123-SPLUS-s02s22-033516.pdf} & \includegraphics[width=0.19\linewidth, clip]{Figs/Figs-lamost/spec-56945-EG233528N011847B01_sp05-120-STRIPE82-0164-033376.pdf} \\
    \includegraphics[width=0.19\linewidth, clip]{Figs/Figs-lamost/spec-56948-EG212551N003203M01_sp06-078-STRIPE82-0120-000615.pdf} & \includegraphics[width=0.19\linewidth, clip]{Figs/Figs-lamost/spec-56948-EG212551N003203M01_sp13-186-STRIPE82-0120-045064.pdf} & \includegraphics[width=0.19\linewidth, clip]{Figs/Figs-lamost/spec-56976-EG215014S003621B01_sp06-085-SPLUS-s02s04-032140.pdf} & \includegraphics[width=0.19\linewidth, clip]{Figs/Figs-lamost/spec-56978-EG001639N015102M01_sp01-071-STRIPE82-0008-013742.pdf} & \includegraphics[width=0.19\linewidth, clip]{Figs/Figs-lamost/spec-56983-EG233528N011847M01_sp02-123-SPLUS-s02s22-033516.pdf} \\
    \includegraphics[width=0.19\linewidth, clip]{Figs/Figs-lamost/spec-56983-EG233528N011847M01_sp05-120-STRIPE82-0164-033376.pdf} & \includegraphics[width=0.19\linewidth, clip]{Figs/Figs-lamost/spec-56983-EG233528N011847M01_sp08-192-STRIPE82-0164-020337.pdf} & \includegraphics[width=0.19\linewidth, clip]{Figs/Figs-lamost/spec-57043-EG030739N012421M01_sp02-020-STRIPE82-0067-036513.pdf} & \includegraphics[width=0.19\linewidth, clip]{Figs/Figs-lamost/spec-57043-EG030739N012421M01_sp04-155-STRIPE82-0068-063428.pdf} & \includegraphics[width=0.19\linewidth, clip]{Figs/Figs-lamost/spec-57043-EG030739N012421M01_sp05-021-STRIPE82-0068-006406.pdf} \\
    \includegraphics[width=0.19\linewidth, clip]{Figs/Figs-lamost/spec-57308-EG000023N024031M01_sp01-121-STRIPE82-0002-034805.pdf} & \includegraphics[width=0.19\linewidth, clip]{Figs/Figs-lamost/spec-57308-EG000023N024031M01_sp07-202-STRIPE82-0004-031191.pdf} & \includegraphics[width=0.19\linewidth, clip]{Figs/Figs-lamost/spec-57313-EG220318S020919M01_sp03-071-SPLUS-s02s05-005410.pdf} & \includegraphics[width=0.19\linewidth, clip]{Figs/Figs-lamost/spec-57313-EG220318S020919M01_sp05-048-SPLUS-s03s05-005815.pdf} & \includegraphics[width=0.19\linewidth, clip]{Figs/Figs-lamost/spec-57313-EG220318S020919M01_sp10-173-SPLUS-s03s05-011958.pdf} \\
    \includegraphics[width=0.19\linewidth, clip]{Figs/Figs-lamost/spec-57313-EG220318S020919M01_sp14-138-STRIPE82-0127-009047.pdf} & \includegraphics[width=0.19\linewidth, clip]{Figs/Figs-lamost/spec-57313-EG220318S020919M01_sp15-081-STRIPE82-0129-009070.pdf} & \includegraphics[width=0.19\linewidth, clip]{Figs/Figs-lamost/spec-57313-EG220318S020919M01_sp15-198-SPLUS-s02s05-026507.pdf} & \includegraphics[width=0.19\linewidth, clip]{Figs/Figs-lamost/spec-57328-EG213118N034906M01_sp01-191-STRIPE82-0120-045064.pdf} & \includegraphics[width=0.19\linewidth, clip]{Figs/Figs-lamost/spec-57336-EG034838N001340M01_sp05-189-STRIPE82-0081-034915.pdf} \\
    \includegraphics[width=0.19\linewidth, clip]{Figs/Figs-lamost/spec-57336-EG034838N001340M01_sp09-025-STRIPE82-0084-014280.pdf} & \includegraphics[width=0.19\linewidth, clip]{Figs/Figs-lamost/spec-57336-EG034838N001340M01_sp13-222-STRIPE82-0086-013095.pdf} & \includegraphics[width=0.19\linewidth, clip]{Figs/Figs-lamost/spec-57336-EG034838N001340M01_sp14-093-STRIPE82-0080-039699.pdf} & \includegraphics[width=0.19\linewidth, clip]{Figs/Figs-lamost/spec-57336-EG222106N010205M01_sp03-208-STRIPE82-0136-036365.pdf} & \includegraphics[width=0.19\linewidth, clip]{Figs/Figs-lamost/spec-57387-EG010159N023133M01_sp02-162-STRIPE82-0020-062391.pdf} \\
    \includegraphics[width=0.19\linewidth, clip]{Figs/Figs-lamost/spec-57388-EG015238N022953M01_sp07-195-STRIPE82-0044-041254.pdf} & \includegraphics[width=0.19\linewidth, clip]{Figs/Figs-lamost/spec-57453-HD143257N033933M01_sp16-159-SPLUS-n05n50-013917.pdf} & \includegraphics[width=0.19\linewidth, clip]{Figs/Figs-lamost/spec-57453-HD143257N033933M01_sp16-196-SPLUS-n05n50-021799.pdf} & \includegraphics[width=0.19\linewidth, clip]{Figs/Figs-lamost/spec-57482-HD122600N020231B02_sp04-104-SPLUS-n02n28-039381.pdf} & \includegraphics[width=0.19\linewidth, clip]{Figs/Figs-lamost/spec-57688-EG215014S003621M01_sp04-037-STRIPE82-0125-053114.pdf} \\
    \includegraphics[width=0.19\linewidth, clip]{Figs/Figs-lamost/spec-57688-EG215014S003621M01_sp06-082-STRIPE82-0127-009047.pdf} & \includegraphics[width=0.19\linewidth, clip]{Figs/Figs-lamost/spec-57688-EG215014S003621M01_sp07-238-SPLUS-s03s04-030944.pdf} & \includegraphics[width=0.19\linewidth, clip]{Figs/Figs-lamost/spec-57688-EG215014S003621M01_sp12-086-STRIPE82-0128-050321.pdf} & \includegraphics[width=0.19\linewidth, clip]{Figs/Figs-lamost/spec-57688-EG215014S003621M01_sp13-066-STRIPE82-0129-031160.pdf} & \includegraphics[width=0.19\linewidth, clip]{Figs/Figs-lamost/spec-57688-EG215014S003621M01_sp13-120-STRIPE82-0130-010305.pdf} \\
    \includegraphics[width=0.19\linewidth, clip]{Figs/Figs-lamost/spec-57719-EG012606S021203M01_sp16-150-STRIPE82-0031-049213.pdf} & \includegraphics[width=0.19\linewidth, clip]{Figs/Figs-lamost/spec-57746-KP114249S033242B01_sp11-099-SPLUS-n02s19-006979.pdf} & \includegraphics[width=0.19\linewidth, clip]{Figs/Figs-lamost/spec-57746-KP114249S033242B01_sp12-186-SPLUS-n02s20-017833.pdf} & \includegraphics[width=0.19\linewidth, clip]{Figs/Figs-lamost/spec-57747-EG020648N012631M01_sp04-174-STRIPE82-0046-031141.pdf} & \includegraphics[width=0.19\linewidth, clip]{Figs/Figs-lamost/spec-57747-EG020648N012631M01_sp05-005-STRIPE82-0047-031879.pdf} \\
    \includegraphics[width=0.19\linewidth, clip]{Figs/Figs-lamost/spec-57751-KP114249S033242B02_sp11-099-SPLUS-n02s19-006979.pdf} & \includegraphics[width=0.19\linewidth, clip]{Figs/Figs-lamost/spec-57751-KP114249S033242B02_sp12-186-SPLUS-n02s20-017833.pdf} & \includegraphics[width=0.19\linewidth, clip]{Figs/Figs-lamost/spec-57787-HD114640S034432B01_sp16-027-SPLUS-n02s19-006979.pdf} & \includegraphics[width=0.19\linewidth, clip]{Figs/Figs-lamost/spec-57816-HD115813N032855M01_sp01-173-SPLUS-n02n22-059209.pdf} & \includegraphics[width=0.19\linewidth, clip]{Figs/Figs-lamost/spec-58078-EG224242N000415M01_sp04-100-STRIPE82-0145-047131.pdf} \\
    \includegraphics[width=0.19\linewidth, clip]{Figs/Figs-lamost/spec-58078-EG224242N000415M01_sp08-004-SPLUS-s02s13-032892.pdf} & \includegraphics[width=0.19\linewidth, clip]{Figs/Figs-lamost/spec-58078-EG224242N000415M01_sp10-165-STRIPE82-0141-015370.pdf} & \includegraphics[width=0.19\linewidth, clip]{Figs/Figs-lamost/spec-58080-S82013N00M1_sp03-057-STRIPE82-0019-043090.pdf} & \includegraphics[width=0.19\linewidth, clip]{Figs/Figs-lamost/spec-58080-S82013N00M1_sp05-163-STRIPE82-0021-026890.pdf} & \includegraphics[width=0.19\linewidth, clip]{Figs/Figs-lamost/spec-58138-S82048N00B1_sp16-083-STRIPE82-0068-063428.pdf} \\
  \end{longtable}
\end{center}

%%%%%%%%%%%%%%%%%%%%%%%%%%%%%%%%%%%%%%%%%%%%%%%%%%

% Don't change these lines
\bsp	% typesetting comment
\label{lastpage}
\end{document}

